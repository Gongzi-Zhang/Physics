哥白尼原则: 
    观察位置不特殊,从统计角度而言,一个事物存在时间越长,我们的观察点距离寿命终结点的绝对时间就越长,因为没有任何理由认为你恰巧出于最后一个时间点,最大的可能是出于中间的某个时间点,如果以10\%代表起点和终点,那么我们最有可能出于中间的80\%。(有点类似于泊松分布--你在任一时刻观察到某事件发生的可能不受观察时间点的影响)

惯性:	\\
    指一种状态,而非一个物理量。特指不受力的状态。\\

惯性质量:   \\
    指偏离惯性这种状态的难易程度。

    在牛顿力学中,用加速度作为度量,人们通常所说的质量是惯性质量。根据牛二定律,在相同作用力下,加速度与惯性质量成反比。因而惯性质量越大,加速度越小,偏离惯性状态越少。

    在相对论下,我们用动量的变化率作为偏离惯性状态的度量,这里我们关注的是惯性运动动量。惯性质量已经变为联系动量和速度的关系式中的一个参数。

