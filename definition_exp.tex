\chapter{Experiments}
High energy physical experiments:
\begin{itemize}
    \item Energy frontier: search for new particles directly
    \item Precision (luminosity) frontier: looking for tiny deviations from SM
	predictions or at phenomena that are highly-suppressed or forbidden 
	by SM symmetries. Like: Electric Dipole Moments (EDM), neutrino-less
	double beta decay ($0\nu\beta\beta$), Baryon number of lepton flavor
	violation (proton decay, $\mu 2e$), Parity-Violating electron scattering.
    \item Most precise EW experiments:
	\begin{itemize}
	    \item electron g-2: fine structure constant
	    \item muon lifetime: Fermi constant
	    \item Z line shape: mass of Z boson
	\end{itemize}
\end{itemize}

Measurement
\begin{itemize}
    \item gain: $10^5 - 10^7$
    \item rate: $10 kHz - 10 MHz/cm^2$ dependending on detectors
    \item time: $~10\ ps - ~1ns$ 
    \item spatial resolution: $~\mu m$
    \item power: $~1 - ~1000\ mW/channel$
\end{itemize}
Questions:
\begin{itemize}
    \item Do we really observe any event with final state particles (jets) total
	energy larger than half of the colliding energy in pp or ep collisions?
	If not, how do we determine the x distribution above 0.5?
    \item Why LEP2 didn't find the Higgs, now that the $\sqrt{s} = 210 \ GeV > m_H$ 
    \item Can we do something with the dumped beam? looks like waste of energy
\end{itemize}

%%%%%%%%%%%%%%%%%%%%%%%%%%%%%%%%%%%%%%%%%%%%%%%%%%%%%%%%%%%%%%%%%%%%%%%%
\section{General}
In the lab frame:
$$ \vec{p}_1 = \vec{p}_{beam}, \vec{p}_2 = 0, \vec{p}_3 = \vec{p}_{scatter}, \vec{p}_4 = \vec{p}_recoil $$
$$ s = (p_1 + p_2)^2 = (E_1 + m_2)^2 - |\vec{p}_1|^2 = m_1^2 + m_2^2 + 2E_{beam}m_2 $$
Lorentz transformation from lab to CM frame:
$$ \beta = \frac{\vec{p}_{beam}}{E_{beam} + m_2} \qquad \gamma = \frac{E_{beam} + m_2}{\sqrt{s}} $$
%%%%%%%%%%%%%%%%%%%%%%%%
\subsubsection{Beam}
Typical beam parameters: see pdg review about collider parameters

The CEBAF accelerator is special, it is for nuclear physics, rather than
the high energy physics. Though of low energy, it is capable of providing 
stable and precise energy electron beam, it can also deliver beams to 3 (now 4)
halls simultaneously, each doesn't require high beam current for nuclear 
experiments. Here are some key parameters about electron beam in CEBAF:
\begin{itemize}
    \item circumference: 1.4 km
    \item bunch repetition rate: 249.5 MHz (the srf frequency is 1497 MHz, 
	CEBAF operates at either the third or the sixth subharmonic RF frequency
	in order to create interleaved beams)
    \item charge per bunch 4 - 480 fCoul = $2.5 - 300 \times 10^5$  electrons,
	which is much smaller than that in any high energy collider ($\sim 10^{11}$).
\end{itemize}

%%%%%%%%%%%%%%%%%%%%%%%%%%%%%%%%%%%%%%%%%%%%%%%%%%%%%%%%%%%%%%%%%%%%%%%%
\section{Beam Production}

%%%%%%%%%%%%%%%%%%%%%%%%
\subsubsection{Electron Production}
Shine laser on photocathode material, which will produce electron beam. One 
problem is how to balance the left positive nuclei? What's the lifetime of 
a photocathode material?

%%%%%%%%%%%%%%%%%%%%%%%%
\subsubsection{Proton Production}
Using microwave oven to create plasma from hydrogen gas, then with an electric
field, the electron and proton will go in opposite ways, the protons will be
focused and spliced into packets to form bunches.

%%%%%%%%%%%%%%%%%%%%%%%%
\subsubsection{Positron Production}
Most external positron beams come from secondary interaction:
$$ e^- + X \rightarrow e^- + \gamma + X \rightarrow e^- + (e^+ + e^-) + X $$
here, X can be anything, usually a high density plane, which will produce 
$gamma$ ray by Bremsstrahlung radiation. If X is another electron beam, then
we create positron by electron collision:
$$ e^- + e^- \rightarrow e^- + (e^+ + e^-) + e^- $$
The higher the incoming electron beam's energy, the larger the positron 
production rate

%%%%%%%%%%%%%%%%%%%%%%%%
\subsubsection{Muon Production}
Conventional way: Proton hitting targets, which will produce $\pi$, K particles,
whose decay product is usually muon: muons arising from this way are 100\% polarized.

Noval approach: $e^+e^- \rightarrow \mu^+\mu^-$

%%%%%%%%%%%%%%%%%%%%%%%%%%%%%%%%%%%%%%%%%%%%%%%%%%%%%%%%%%%%%%%%%%%%%%%%
\section{Transport}

%%%%%%%%%%%%%%%%%%%%%%%%
\subsubsection{Synchrotron Radiation (SR)}
The synchrotron radiation matters much more for electron than heavier particles
bacause the radiation power $P \propto \gamma^4 = \lfet(\frac{E}{m}\right)^4$, 
so at the same energy, electron will emit $\left(\frac{m_p}{m_e}\right)^4 \sim 10^{13}$
more radiation than proton. Synchrotron radiation will degrade electron beam
quickly if no energy supply from RF, while its effect on proton beam is negligible.

Though of being an excellent light source, the SR is bad for accelerator itself,
there are a few damages it will caused to the accelerator:
\begin{itemize}
    \item Heat load on the beam pipe
    \item Gas load, when SR hit the beam pipe, it will disorb the gas molecules 
	on it, which will imcrease pressure, therefore reduce beam life time
	and increase background noise
    \item Emission of photoelectrons from the beam pipe through , enhance the
	formation of the electron cloud (when accelerated by the electric field 
	of a positively charged beam, the photoelectron will hit the beam pipe
	surface and produce secondary electrons, which will increase exponentially
	to form a elctron cloud around the beam bunches), leads to the beam 
	instability (electron cloud instability).
\end{itemize}

%%%%%%%%%%%%%%%%%%%%%%%%
\subsubsection{Magnets}
\begin{itemize}
    \item State of art (2019) dipole: $14 \  T$
\end{itemize}

%%%%%%%%%%%%%%%%%%%%%%%%
\subsubsection{Cooling}
For hadrons and muon beams
\begin{itemize}
    \item stochastic cooling
    \item electron cooling
    \item ionization cooling (for muon beam)
\end{itemize}

%%%%%%%%%%%%%%%%%%%%%%%%
\subsubsection{Spin maintainance}
Larmor precession: the spin precesses around $\vec{B}$ with a frequency 
proportional to $B$:


%%%%%%%%%%%%%%%%%%%%%%%%%%%%%%%%%%%%%%%%%%%%%%%%%%%%%%%%%%%%%%%%%%%%%%%%
\section{Beam Monitoring}

%%%%%%%%%%%%%%%%%%%%%%%%
\subsubsection{Luminosity}
The beam luminosity depends on beam size, beam intensity (number of particles
per bunch) and beam frequency (number of beam bunches per second). Nowadays,
the large accelerators (LHC, RHIC, HERA, CEBAF) have a beam intensity around:
$10^{10} - 10^{11}$ particles (proton, electron) per bunch with an emittance of:
$\sim \pi\mu m \cdot rad$ (and $\beta^* ~ 1 m$), which corresponding to 
$\sim 10^{13} - 10^{15} cm^{-2}$, much less than fixed target density, 
e.x.: $10^{22} cm^{-2}$ for $H_2O$

Beam Lumonosity inside a storage ring will decay along time:
\begin{figure}
    \includegraphics[width=0.6\linewidth]{RHIC_beam_luminosity}
\end{figure}

There are many reasons that will damage beam luminosity:
\begin{itemize}
    \item IntraBeam Scattering (IBS) (Coulumb Scattering): which will increase
	beam emittance and energy spread (especially for heavy-ion beam)
    \item Beam-Beam interaction: this may be important depending on bunch intensity
    \item Radiation damping (small effect)
    \item Beam-gas collision (small effect): collision with some residual gas 
	in the beam pipe
    \item Bound-free pair production (for heavy ion beams)
    \item Coulomb excitation (for EIC): the photons emitted from the electron
	beam will excite the Ion beam, which will lead to neutron emission or
	nucleus breakup.
\end{itemize}

Luminosity Measurement:
Absolute luminosity measurement by measuring the synchronic radiation and 
relative luminosity between opposite polarization.

%%%%%%%%%%%%%%%%%%%%%%%%
\subsubsection{Beam Size (Phase and Beta Function Measurement)}

%%%%%%%%%%%%%%%%%%%%%%%%
\subsubsection{Beam Position and Angle at the IP (BPM)}

%%%%%%%%%%%%%%%%%%%%%%%%
\subsubsection{Beam Energy}
How do we know the beam energy at IP? 

%%%%%%%%%%%%%%%%%%%%%%%%
\subsubsection{Spin Measurement}
For electron beam, we can use Compton polarimeter or Moller polarimeter because
the QED process is well known and we can calculate the analyzing power $A_n$, so
that by measuring the asymmetry $A_m$, we will know the beam polarization:
$$ P_b =  \frac{A_m}{A_n} $$

But for hadron beam, we can't do the same thing, because we can't calculate
the analyzing power from first principle due to complication of hadrons. 
Currently, the best method to measure hadron polarization is the process of 
elastic scattering in the Coulomb-Nuclear Interference (CNI) region, for which
there are only effective models available. We measure the polarization relatively:
$$ P_{beam} = -\frac{\epsilon_{beam}}{\epsilon_{target}g}P_{target}$$
Where $\epsilon = A_N\cdot P = \frac{N_L - N_R}{N_L + N_R}$
I don't understand why electron scattering can't work this way?
%%%%%%%%%%%%%%%%%%%%%%%%
\subsubsection{Crossing}
Cross angle at interaction point is related to particle specices. Different
particles have different trajectory inside the storage (accelerating) pipe, 

Crossing (luminors) region: For ATLAs, typical transverse (x, y) size: $\sim 10 \mu m$;
typical longitudinal (z) size: $\sim 10 mm$

%%%%%%%%%%%%%%%%%%%%%%%%
\subsubsection{Roman Pot}
They are placed as close to the beampipe as possible, they will record any protons
that are not travelling precisely along the beamline, and thus record the elastic
scattering of the proton. 
%%%%%%%%%%%%%%%%%%%%%%%%%%%%%%%%%%%%%%%%%%%%%%%%%%%%%%%%%%%%%%%%%%%%%%%%
\section{Detectors}

\begin{figure}
    \includegraphics[width=\linewidth]{charge_amplification}
    \caption{Incident particle: electron}
\end{figure}
\begin{itemize}
    \item Scintillator: $\sim 120 \gamma/MeV$, $95\% < 25\ ns$
    \item Calorimeter resolution:
	$$ \frac{\sigma(E)}{E} = \frac{a}{\sqrt{E}} + b + \frac{c}{E} $$
	For ECAL, the a term is usually a few percent, while for HCAL, the a 
	term can be a few tenth percent
    \item Multiwire-Proportional Chambers (MVPC): spatial resolution $\sim s/\sqrt{12}$;
	typical wire distance: 2mm $\Rightarrow \sigma_s = 0.6\ mm$
    \item Drift Chamber: $\sigma_s \sim 100\ \mu m$
\end{itemize}

%%%%%%%%%%%%%%%%%%%%%%%%
\subsubsection{Vertex Reconstruction}
\begin{enumerate}
    \item Track Pre-selection
    \item Seed-Finding: if no seed, finish
	\begin{itemize}
	    \item Create a weighted histogram of z0 for tracks passing selection, 
		  weight by log pT and d0
	    \item Find the interval [$z_{min}$,  $z_{max}$], containing at 
		  least a fraction f (default = 50\%) of the tracks, which has 
		  the largest weight density ( $\frac{\sum_w}{z_{max - z_{min}}}$)
		  is selected.  Use only entries in the interval just found
	    \item Seed z position is weighted average of last 2 or 3 entries
	    \item Seed (x, y) assumed to be center of luminous region
	\end{itemize}
    \item Track Assignment: if no tracks, finish;
	\begin{itemize}
	    \item Any track passing within $12\sigma$ of seed is added to list 
		of tracks to fit
		\begin{itemize}
		    \item $\sigma$ includes track measurement errors only
		    \item Vertex fit is robust against outliers
		    \item May need tightening for poorly-measured large-|$\eta$| 
			tracks as pileup increases
		\end{itemize}
	    \item If not at least two tracks passing 12 𝜎 cut, vertex-finding is finished
		\begin{itemize}
		    \item Typically occurs when only a few scattered tracks 
			remain in seed-finding
		    \item Resulting seed is far from all of them
		\end{itemize}
	\end{itemize}
    \item Vertex Reconstruction: Vertex Fit
	\begin{itemize}
	    \item An ``adaptive'' vertex fit, based on the Kalman filter, is used.
		Luminous region centroid (x, y) used as a constraint
	    \item The Kalman filter itself is equivalent to a weighted least-squares fit
	    \item Each track's weight is determined by its consistency with the 
		previous iteration of fitting
	\end{itemize}
    \item Iterate step 2-5
    \item Finish
\end{enumerate}

\begin{itemize}
    \item The ``hard-scatter'' vertex is identified as the one with largest $\sum p_T^2$
    \item Number of vertex is not exactly the same as number of pp collision ($\mu$).
	Some vertices are lost due to various reasons:
	\begin{itemize}
	    \item no associated charge particles within the ID acceptance
	    \item track reconstruction inefficiencies or errors
	    \item merge of vertices, especially when $\mu > 20$
	\end{itemize}
\end{itemize}

%%%%%%%%%%%%%%%%%%%%%%%%
\subsubsection{Multiple Coulomb Scattering (MCS)}
\begin{itemize}
    \item a statistical description arising from many small interactions with
	atomic electrons
    \item MCS alters the direction of the particle
    \item 
	 $$ \langle\theta\rangle = 0 $$
	 $$ \sigma_\theta = \frac{13.6 \v MeV}{\beta cp}z\sqrt{x/X_0}\left[ 1 + 0.038ln (x/X_0) \right]$$
	 where z is the particle charge, and $X_0$ is the material's Radiation
	 Length.
\end{itemize}

%%%%%%%%%%%%%%%%%%%%%%%%%%%%%%%%%%%%%%%%%%%%%%%%
\subsection{Spin}
\begin{itemize}
    \item A single spin-1/2 particle is always 100\% polarized along some direction: $\hat{P}$
    \item As ensemble of spin-1/2 particles (target, beam) can be fully described 
	by the average of all $\hat{P} \Rightarrow \vec{P}$ 
    \item Any observable (must be scalar) can be represented as 
	$\vec{A} \cdot \vec{P}$, where $\vec{A}$ is the analyzing power
    \item If ensemble has a quantization axis (e.g. a B-field), $\vec{P}$ will 
	be along that axis: $P_z = (N^\uparrow - N^\downarrow)/(N^\uparrow + N^\downarrow)$
    \item Spin-1 ensembles (e.g. polarized deuteron targets) have both a vector
	polarization ($P_z$) and a tensor polarization ($P_{zz}$ ???) along
	quantization axis.
\end{itemize}
\section{Physics}
\begin{figure}
    \includegraphics[width=\linewidth]{multiplicity_in_diffractive_events}
    \caption{Multiplicity in diffractive events}
\end{figure}

\begin{figure}
    \includegraphics[width=\linewidth]{hadrons_spacetime_evolution}
    \caption{Spacetime evolution}
\end{figure}



%%%%%%%%%%%%%%%%%%%%%%%%%%%%%%%%%%%%%%%%%%%%%%%%%%%%%%%%%%%%%%%%%%%%%%%%
\section{Detectors}
\begin{itemize}
    \item What's the difference between Gas Electron Multiplier (GEM) and
	Micro Mesh Gas (Micromegas)
\end{itemize}

%%%%%%%%%%%%%%%%%%%%%%%%%%%%%%%%%%%%%%%%%%%%%%%%%%%%%%%%%%%%%%%%%%%%%%%%
\section{Future Colliders}
\begin{figure}
    \includegraphics[width=\linewidth]{future_colliders_2019}
\end{figure}

%%%%%%%%%%%%%%%%%%%%%%%%%%%%%%%%%%%%%%%%%%%%%%%%
\subsection{Future Circular Collider (FCC: CERN)}

%%%%%%%%%%%%%%%%%%%%%%%%%%%%%%%%%%%%%%%%%%%%%%%%
\subsection{The Compact LInear Collider (CLIC: CERN)}

%%%%%%%%%%%%%%%%%%%%%%%%%%%%%%%%%%%%%%%%%%%%%%%%
\subsection{International Linear Collider (ILC: Japan)}

%%%%%%%%%%%%%%%%%%%%%%%%%%%%%%%%%%%%%%%%%%%%%%%%
\subsection{Circular Electron Positron Collider (CEPC: China)}

%%%%%%%%%%%%%%%%%%%%%%%%%%%%%%%%%%%%%%%%%%%%%%%%
\subsection{Muon Collider}
Motivation:
\begin{itemize}
    \item Like electron, clean physics
    \item Heavy, negligible synchrotron radiation, no beamstrahlung $\Rightarrow$
	smaller collider or larger $\sqrt{s}$; excellent energy resolution
    \item Large direct coupling to the Higgs boson (heavy)
    \item Naturally longitudinally polarized (100\%) when arised from $\pi^\pm$ 
	decay to $\mu^\pm\nu_\mu$
    \item Natural neutrino factories due to muon decay
\end{itemize}

Machine:
\begin{itemize}
    \item Driver: proton-driven ($p + C \rightarrow \pi^\pm, K^\pm$) or 
	ep driven ($e^+ + e^- \rightarrow \mu^+ + \mu^-$).
	If proton-driven, require robust target
    \item Muon collector: strong magnetic field of $20\ T$
    \item Cooling
    \item Fast acceleration and injection into circular rings
    \item High luminosity will be a challenge
\end{itemize}
