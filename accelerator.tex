\section{Accelerators}
\begin{itemize}
    \item Novosibirsk (RU) \href{https://accelconf.web.cern.ch/p95/ARTICLES/MAD/MAD04.PDF}{Accelerator Field Development at Novosibirsk (History, Status, Prospects)}: 
	\begin{itemize}
	    \item VEP-1 collider: 2x0.16 GeV, lumi: $3 \times 10^{27} cm^{-2} sec^{-1}$
	    \item VEPP-2 collider (1967): world first e-p experiment, 2x0.7 GeV, lumi: $2\times 10^{28} cm^{-2} sec^{-1}$, $\rho$, $\omega$, $\phi$
	    \item VEPP-2M collider (1975): 2x0.7 GeV, lumi: $5\times 10^{30} cm^{-2}sec^{-1}$, intensity: up to $0.8 \times 10^{11}$ and $3\times10^{11}$(BEP booster) per bunch
		\begin{itemize}
		    \item CMD-2: superconducting magnet spectrometer
		    \item SND: advanced crystal, high granularity, 3 layers EM calorimeter
		\end{itemize}
	    \item VEPP-4 collider (1980): E up to 5.5 GeV per beam, lumi: $5\times10^{30} cm^{-2}s^{-1}$
	    \item VEPP-4M collider
	    \item VEPP-5 C-$\tau$ factory
	    \item $\Phi$ factory: CP violation study
	    \item VLEPP: linear collider, designed, unfinish
	\end{itemize}
    \item ACO (Orsay, FR)
    \item CEA (Cambridge, UK)
    \item SPEAR
    \item BESSY
    \item DORIS
    \item CESR
    \item PETRA
    \item PEP
    \item TRISTAN
    \item LEPA
\end{itemize}
