\chapter{Model}
%%%%%%%%%%%%%%%%%%%%%%%%%%%%%%%%%%%%%%%%%%%%%%%%%%%%%%%%%%%%%%%%%%%%%%%%
\section{Model}

Boltzmann distribution

\subsection{Oscillation}

\subsection{Binary system}

\subsection{Scattering}

\subsection{Drude model}
\begin{equation}
    m\ddot{\bm{x}} = -e\bm{E} - \frac{m\dot{\bm{x}}}{\tau}
\end{equation}
Where $\tau$ is the relaxation time, Solve this equation, we will get
\[
    0 = -e\bm{E} - \frac{m\dot{\bm{x}}}{\tau} \Longrightarrow
    \dot{\bm{x}} = \frac{-e\bm{E}\tau}{m}
\]
This means that if electron reach velocity $\dot{x} = -eE\tau/m$, it will move at this constant velocity.

Note the difference between the Drude model and the "accelerating model", which will accelerates electron periodically, with period being the relaxation time $\tau_1$, so the average velocity will be:
\[
    \bar{v} = \frac{1}{2}\frac{-eE\tau_1}{m}
\]
So 
\[
    \dot{x} = \bar{v}  \Longrightarrow \tau = \frac{1}{2}\tau_1
\]


%%%%%%%%%%%%%%%%%%%%%%%%%%%%%%%%%%%%%%%%%%%%%%%%%%%%%%%%%%%%%%%%%%%%%%%%
\section{Approximation}
The ultimate weapon of simplifying a question (constructing a model)
is approximation. Here are some frequently used approximation.

%%%%%%%%%%%%%%%%%%%%%%%%%%%%%%%%%%%%%%%%%%%%%%%%
\subsection{RWA: Rotating Wave Approximation}
\[
    \cos(\omega t)e^{-i\omega_0 t} = \frac{1}{2}(e^{i\omega t} + e^{-i\omega t})e^{-i\omega t} = \frac{1}{2}(e^{i(\omega - \omega_0)t} + e^{-i(\omega + \omega_0)t}) \approx \frac{1}{2}e^{i(\omega - \omega_0)t}
\]
When $\omega \sim \omega_0$, we simply discard the second term because
$\omega + \omega_0 >> \delta = \omega - \omega_0$. We can think about it
as that the second term oscillates so fast that nothing can response to it
(response is small), thus the first term is more effective than the second
term.


%%%%%%%%%%%%%%%%%%%%%%%%%%%%%%%%%%%%%%%%%%%%%%%%
\subsection{Electric Dipole Approximation}
For the interaction of light with atom, the wavefunction vanish 
exponentially on the distance scale of the Bohr radius $a_0$.
\[
    |\vec{k} \cdot \vec{r}| \approx 2\pi a_0/\lambda << 1 
\]
\[
    e^{i\vec{k} \cdot \vec{r}} = 1 + i\vec{k}\cdot\vec{r} + \mathcal{O}(\vec{k}\cdot\vec{r})^2
\]

Which can be easily seen from the point that an atom is much smaller than 
the wavelength of light (visible light), so we can think that the 
electric field applied on the atom is constant at specific time.

\begin{tikzpicture}
    \draw plot[domain=0:2*pi, smooth] (\x, {sin(\x r)});
    \draw (1, {sin(1 r)}) circle (2.5pt);
    \node (atom) at (1, {sin(1 r) - 0.4}){atom};
\end{tikzpicture}



%%%%%%%%%%%%%%%%%%%%%%%%%%%%%%%%%%%%%%%%%%%%%%%%
\subsection{Slowly Varying Envelope Approximation}
The fractional change of the beam's transverse profile over a wavelength
distance to be small compared with unity, and similarly for its 
derivative, which si:
\[
    \lambda|\frac{\partial E_0}{\partial z}| << |E_0| 
    \text{and} 
    \lambda|\frac{\partial^2 E_0}{\partial z^2}| << |\frac{\partial E_0}{\partial z}| 
\]



%%%%%%%%%%%%%%%%%%%%%%%%%%%%%%%%%%%%%%%%%%%%%%%%
\subsection{Harmonic Approximation}
\[
    U = U_0 + \sum_{n, \alpha} \frac{\partial U}{\partial X_n^\alpha} X_n^\alpha + \frac{1}{2}\sum_{n \alpha; n' \alpha'}\frac{\partial^2U}{\partial X_n^\alpha \partial X_{n'}^{\alpha'}}X_n^\alpha X_{n'}^{\alpha'} + cdots
\]

Truncate the series after second order.



%%%%%%%%%%%%%%%%%%%%%%%%%%%%%%%%%%%%%%%%%%%%%%%%%%%%%%%%%%%%%%%%%%%%%%%%
\section{Distribution}



%%%%%%%%%%%%%%%%%%%%%%%%%%%%%%%%%%%%%%%%%%%%%%%%
\subsection{Boltzmann Distribution}
