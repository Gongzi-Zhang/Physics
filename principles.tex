%%%%%%%%%%%%%%%%%%%%%%%%%%%%%%%%%%%%%%%%%%%%%%%%%%%%%%%%%%%%%%%%%%%%%%%%
\section{Principles}
\textbf{What makes life easier: periodity, symmetry, linearity (constant), 
causality, conservation, static, equilibrium, main term, limitation, 
independent, binary system.}

\textbf{What makes life hard: nonlinearities, feedback, coupling (correlation),
fluctuation, perturbation, evolution.}

\bigskip
Important principles in Physics:
\begin{itemize}
    \item The smallest action principle (for evolution)
    \item The smallest energy (for static status): this doesn't look like a good
	principle, because it specialize energy, making it different from other
	physical quantaties? But on the other hand, it looks like energy is the
	only scalar quantaty 
    \item Symmetry
\end{itemize}

Application:
\begin{itemize}
    \item The Smallest action + Galileo Symmetry = Classical Mechanics
    \item The Smallest action + Lorentz Symmetry = Relativistic Mechanics
    \item The Smallest action + Lorentz Symmetry + Field = Field Theory
    \item The Smallest action + Lorentz Symmetry + Field + Gauge U(1) = Maxwell Equation
\end{itemize}
The problem is how to construct the Lagrangian?

%%%%%%%%%%%%%%%%%%%%%%%%%%%%%%%%%%%%%%%%%%%%%%%%
\subsection{Symmetry}
For a theory, study the symmetry of its Equation of Motion (EOM), and then 
specify how physical quantities transform in order to preserve the symmetry.
\begin{itemize}
    \item Continuous symmetry
	\begin{itemize}
	    \item Gauge symmetry: phase transition
	    \item Lorentz transfromation: rotation + boost
	\end{itemize}
    \item Discrete symmetry
	\begin{itemize}
	    \item Parity P
	    \item Charge Conjugate: C
	    \item Time reversal: T
	\end{itemize}
    \item Relative principle
	No matter which representation (picture) you use to describe a physical
	system (process), the physical relationship (result) should be the same
\end{itemize}




%%%%%%%%%%%%%%%%%%%%%%%%%%%%%%%%%%%%%%%%%%%%%%%%
\subsection{others}
\begin{description}[style=nextline]
    \item[correspondence principle]
	which requires that in the limit of large quantum numbers the 
	complete quantum mechanical description must agree with the 
	classical one.
    \item[Uncertainty principle]
	Conjugated variables (how do we know 2 variables are conjugated) can't 
	be identified simultaneously.  (x \& p, t \& E, $\theta$ \& L)
    \item[Practical principle]
	It does not matter that a theory is formulated in terms of infinite 
	quantities as long as observable quantities are finite (renormalization). 
	Extensive use is made of complex quantities in optics, and there is no 
	objection to that as long as the observables are real.
\end{description}

%%%%%%%%%%%%%%%%%%%%%%%%%%%%%%%%%%%%%%%%%%%%%%%%
\subsection{Rules}
\begin{itemize}
    \item (self) Consistency check
    \item Intuitive physical picture
\end{itemize}
