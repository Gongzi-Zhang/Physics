%%%%%%%%%%%%%%%%%%%%%%%%%%%%%%%%%%%%%%%%%%%%%%%%%%%%%%%%%%%%%%%%%%%%%%%%
\section{Principles}
\textbf{What makes life easier: periodity, symmetry, linearity (constant), causality, conservation, static, equilibrium, main term, limitation, independent, binary system.}

\textbf{What makes life hard: nonlinearities, feedback, coupling (correlation), fluctuation, perturbation, evolution.}

%%%%%%%%%%%%%%%%%%%%%%%%%%%%%%%%%%%%%%%%%%%%%%%%
\subsection{Symmetry}
For a theory, study the symmetry of its EOM, and then specify how physical quantities transform in order to preserve the symmetry.
\begin{itemize}
    \item Continuous symmetry
	\begin{itemize}
	    \item Gauge symmetry: phase transition
	\end{itemize}
    \item Discrete symmetry
\end{itemize}




%%%%%%%%%%%%%%%%%%%%%%%%%%%%%%%%%%%%%%%%%%%%%%%%
\subsection{others}
\begin{description}[style=nextline]
    \item[correspondence principle]
	which requires that in the limit of large quantum numbers the 
	complete quantum mechanical description must agree with the 
	classical one.
    \item[Uncertainty principle]
	Conjugated variables can't be identified simultaneously. (x \& p, t \& E, $\theta$ \& L)
    \item[Minimum Action principle]
	The action of a process being minimum.
    \item[Practical principle]
	It does not matter that a theory is formulated in terms of infinite quantities as long as observable quantities are finite. Extensive use is made of complex quantities in optics, and there is no objection to that as long as the observables are real.
\end{description}

%%%%%%%%%%%%%%%%%%%%%%%%%%%%%%%%%%%%%%%%%%%%%%%%
\subsection{Rules}
\begin{itemize}
    \item Consistency check
    \item Intuitive physical picture
\end{itemize}
