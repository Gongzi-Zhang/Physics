%%%%%%%%%%%%%%%%%%%%%%%%%%%%%%%%%%%%%%%%%%%%%%%%%%%%%%%%%%%%%%%%%%%%%%%%
\section{Principles}
\textbf{What makes life easier: periodity, symmetry, linearity (constant), causality, conservation, static, equilibrium, main term, limitation, independent, binary system.}

\textbf{What makes life hard: nonlinearities, feedback, coupling, 
fluctuation, perturbation, evolution.}

%%%%%%%%%%%%%%%%%%%%%%%%%%%%%%%%%%%%%%%%%%%%%%%%
\subsection{Symmetry}
For a theory, study the symmetry of its EOM, and then specify how physical quantities transform in order to preserve the symmetry.




%%%%%%%%%%%%%%%%%%%%%%%%%%%%%%%%%%%%%%%%%%%%%%%%
\subsection{others}
\begin{description}
    \item[correspondence principle]
	which requires that in the limit of large quantum numbers the 
	complete quantum mechanical description must agree with the 
	classical one.
    \item[Uncertainty principle]
	Conjugated variables can't be identified simultaneously. (x \& p, t \& E, $\theta$ \& L)
    \item[Minimum Action principle]
	The action of a process being minimum.
\end{description}
