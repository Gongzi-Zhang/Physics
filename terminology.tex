\chapter{Terminology}
\begin{description}
    \item [analyticity]
    \item [bilinear]	In QFT, a bilinear term means it has exactly two
	fields. Such as:
	\[ \mathcal{L}_K \supset \frac{1}{2}\phi\Box\phi,
	\frac{1}{4}F^2_{\mn}, \frac{1}{2}m^2\phi^2,
	\frac{1}{2}\phi_1\Box\phi_2, \phi_1\partial_\mu A_\mu, \dots \]
    \item [Boltzmann distribution]  $n_{i} = Ne^{-\beta E_{i}}$
    \item [causality]
    \item [cluster decomposition principle]
    \item [covariant vectors]	vectors with lower indices
    \item [contravariant vectors]   vectors with upper indices
    \item [Dirac spinors]
    \item [equipartion theorem] a body in thermal equilibrium should have
	energy equally distributed among all possible modes, (mode is a
	seperation of phase space), which means all modes have the same
	energy.
    \item [faithful representation] A representation in which each group
	element gets its own matrix is called a \emph{faithful
	representation}.
    \item [Fermi's golden rule]	$\Gamma \sim |\mathcal{M}|^{2}\delta(E_f - E_i)$
    \item [first principle]
    \item [Gauge transform] $\phi \rightarrow e^{-i\alpha}\phi$
    \item [Helicity] spin projected on the direction of motion is called the
	helicity
    \item [Legendre transformation]
    \item [Lie algebra] the generators of the Lie group form an algebra
	called its Lie algebra. 
    \item [Lie groups]	Lie groups are a class of groups, including the
	Lorentz group, with an infinite number of elements but a finite
	number of generators.
    \item [lightlike]	$V^\mu V_\mu = 0$
    \item [little group] The representation of the full \Poincare{}
	group is induced by a representation of the subgorup of the
	\Poincare{} group that holds $p^\mu$ fixed, called the
	\emph{little group}
    \item [locality]
    \item [Lorentz group] this is the generalization of the rotation group
	to include both rotations and boosts.
    \item [pseudo scalar] particles with odd \textbf{parity}
    \item [quantize]	promote x and p as operators and impose the
	canonical commutation relations:    $[x, p] = i$
    \item [quantum process]  time evolution of an open quantum system ???
    \item [representation] A set of objects that mix under a transformation
	group is called a representation of the group, though technically
	the matrix embedding is the representation.
    \item [second quantization]	canonical quantization of relativistic
	fields, 
	\[
	    H_0 = \int \frac{d^{3}p}{(2\pi)^3}\omega_p(a^{\dag}_p a_p +
	\frac{1}{2})
	\] 
	First quantization refer to the discrete mode, for
	example, of a particle in a box. Second quantization refers to the
	integer numbers of excitations of each of these modes. There are two
	features in second quantization:
	\begin{enumerate}
	    \item We have many quantum mechanical systems - one for each
		$\vec{p}$ - all at the same time.
	    \item We interpret the nth excitation of the $\vec{p}$ harmonic
		oscillator as having n \emph{particles}.
	\end{enumerate}
    \item [S-matrix]
    \item [spacelike]	$V^\mu V_\mu < 0$
    \item [SO(n)] the group of nD rotations ($det(R) = 1$)
    \item [timelike]	$V^\mu V_\mu > 0$
    \item [unitary] $\Lambda^{\dag}\Lambda = 1$
\end{description}
