\newcommand{\QM}{Quamtum Mechanics}
\newcommand{\QFT}{Quantum Field Theory}
\newcommand{\RQFT}{Relativistic Quantum Field Theory}
\newcommand{\FT}{Fourier Transform}
\newcommand{\FFT}{Fast Fourier Transform}
\newcommand{\LT}{Lorentz Transform}
\newcommand{\LI}{Lorentz Invariant}
\newcommand{\LG}{Lorentz Group}
\newcommand{\KG}{Klein-Gordon}
\newcommand{\EL}{Euler-Lagrange}
\newcommand{\sr}{special relativity}
\newcommand{\Poincare}{Poincar$\acute{\textrm{e}}$}

\chapter{QFT}

\begin{itemize}
    \item What's a field, classical field is a spatial distribution, quantum
	field is a analogy to classical one, but make up of creation and
	destruction operator.
    \item How to construct Lagrangian from a field $\leftarrow$ Lorantz
	invariant.
    \item EOM
    \item Neother theorem. Conserved current and charge
    \item Symmetry. What's each group? corresponding representation. How to
	embed particles into lorentz group and unitary group?
\end{itemize}
\section{Motivation}

\emph{particle} number is not conserved. The creation and destruction of
particls, which is possible due to the most famous eqn. of \sr{} $E =
mc^{2}$. \emph{Lorentz invariance} guides the definition of particle.

Why QFT: \fbox{quantum mechanics plus \Poincare{} symmetry}. \\
\fbox{Quantum field theory is just quantum mechanics with an infinite number
of harmonic oscillators}

The \QM{} can describe a system with a fixed number of particles
in terms of a many-body wave function. The \RQFT{}
with creation and annihilation operators was developed in order to include
processes in which the number of particles is not conserved,
and to describe the conversion of mass into energy and vice versa.
A consequence of relativity is that the number of particles isn't fixed, ???
though the converse is false: particle production can happen without relativity.

Construct $\mathcal{L}$ from field $\phi$ and its derivative under the rule
of \LI{}. How to incorporate symmetry ?

QFT is the quantum mechanics of {\Large \textit{extensive degrees of freedom}}
$\bra{x}\ket{\phi} = \phi(x)$ is a function of space, the wavefunction.
This looks like a field. It is not what we mean by field in QFT.
meaningless phrases like "second quantization" may conspire to try
to confuse you.

It is not a coincidence that the harmonic oscillator plays an important
role. After all, electromagenetic waves oscillate harmonically.

There are two common ways to quantize a field theory:
\begin{itemize}
    \item canonical quantization
    \item Feynman path integral
    \item Ohter alternatives: perturbation theory
\end{itemize}
\section{Convention}
In relativity, the symmetry refer to invariance after the transformation of
\emph{coordinate system}. So the rotation is:
\begin{equation}
    R = 
	\begin{pmatrix}
	    \cos\theta	& \sin\theta	\\
	    -\sin\theta	& \cos\theta
	\end{pmatrix}
\end{equation}
or in other way:
\[ R^{T}\mathds{1}R = \mathds{1} \]
    
\subsection{4D time-space}   
$dx^{\mu} \equiv (dt, d\vec{x})^{\mu}$ \\
$ ds^{2} = dt^{2} - d\vec{x}\cdot d\vec{x} =
\eta_{\mn}dx^{\mu}dx^{\nu}$	with 
\begin{equation}
    \eta^{\mn} = \eta_{\mn} = 
    \begin{pmatrix}
	+1  & 0	  & 0  & 0	\\
	0   & -1  & 0  & 0	\\
	0   & 0	  & -1 & 0	\\
	0   & 0	  & 0  & -1
    \end{pmatrix}_{\mn}
    \label{Minkowski metric}
\end{equation}
    
Rotation and boost in 4D time-space around x axes is:
\begin{equation*}
    \begin{pmatrix}
	1   &	&   &	\\
	    & 1 &   &	\\
	    &	& \cos\theta_x	& \sin\theta_x	\\
	    &	& -\sin\theta_x	& \cos\theta_x	\\
    \end{pmatrix},
    \begin{pmatrix}
	\cosh\beta_x	& \sinh\beta_x	&   &	\\
	\sinh\beta_x	& \cosh\beta_x	&   &	\\
	    &	& 1 &  	\\
	    &	&   & 1	\\
    \end{pmatrix}
\end{equation*}
    
\subsection{quantization}
\[ [a_k, a_p^{\dag}] = (2\pi)^{3}\delta^{3}(\vec{p}-\vec{k}), 
a_p^{\dag}\ket{0} = \frac{1}{\sqrt{2\omega_p}}\ket{\vec{p}} \]

\begin{description}
    \item [Function derivatives]
	$\frac{\delta \phi(x)}{\delta \phi(y)} = \delta(x-y)$, 
	\[ 
	\frac{\partial(\partial_\alpha A_\alpha)^2}{\partial(\partial_{\mu}A_\nu)}
	=2(\partial_{\alpha}A_\alpha)\frac{\partial(\partial_{\beta}A_\gamma)}{\partial(\partial_{\mu}A_\nu)}g_{\beta\gamma}
	=2(\partial_{\alpha}A_\alpha)g_{\beta\gamma}g_{\gamma\nu}g_{\beta\gamma}
	=2\partial_{\nu}(\partial_{\alpha}A_\alpha)
	\]
    \item [notation] $\phi$ and $\pi$ for scalar fields, $\psi, \xi,
	\chi$ for fermions, $A_{\mu}, J_{\mu}, V_{\mu}$ for vectors and
	$h_{\mu},T_{\mu}$ for tensors.
    \item [Kinetic term] Anything with just two fields of the same or
	different type can be called a kinetic term. Kinetic terms tell you
	about the free (non-interacting) behavior. Though, sometimes it is useful to
	think of a \emph{mass term} such as $m^2\phi^2$, as an interaction
	rather than a kinetic term.
    \item [Boundary conditions] we will always assume that our fields vanish
	on asymptotic boundaries. so we can integrate by part:
	\emph{\[ A\partial_\mu B = -(\partial_\mu A) B\]}
\end{description}

We define quantum fields as integrals over creation and annihilation
operators for each momentum: (Why define it this way ???)
\begin{equation}
\phi_0(\vec{x}) = \int \frac{d^{3}p}{(2\pi)^3}
\frac{1}{\sqrt{\omega_p}}(a_pe^{i\vec{p}\vec{x}} + a_p^\dag
e^{-i\vec{p}\vec{x}})
    \label{eqn:free_field}
\end{equation}
\[ \ket{x} = \phi_0(\vec{x}) \ket{0} \]

There is no physical content in the above equation. It is just a definition.
The physical content is in the algebra of $a_p$ and $a_p^{\dag}$ and in the
Hamiltonian $H_0$. Nevertheless, we will see that collections of $a_p$ and
$a_p^{\dag}$ in the form of Eq.\ref{eqn:free_field} are very useful in
quantum field theory.

Following this defition, we can get:
\[ \pi(\vec{x}) \equiv \partial_t \phi(\vec{x})|_{t=0} = 
-i \int \frac{d^{3}p}{(2\pi)^3} \sqrt{\frac{\omega_p}{2}}(a_pe^{i\vec{p}\vec{x}} - a_p^\dag e^{-i\vec{p}\vec{x}}) \]
$\pi[\phi, \dot{\phi}]$ can also be implicityly defined as:
\[ \frac{\partial \mathcal{H}[\phi, \pi ]}{\partial \pi } = \dot{\phi}\]

\subsection{Hamiltonian \& Lagrangian}
Why do we restrict to Lagrangians of the form $\mathcal{L}[\phi,
\partial_\mu \phi]$? First of all, this is the form that all "classical" 
Lagrangians had. If only first derivatives are involved, boundary 
conditions can be specified by initial positions and velocities only, in
accordance with Newton's laws. In the quantum theory, if kinetic terms 
have too many derivatives, for example $\mathcal{L} = \phi\Box^2\phi$, 
there will generally be disastrous consequences. For example, there may be
states with negative energy or negative norm, permitting the vacuum to decay. 
But interactions with multiple derivatives may occur. Actually, they must
occur due to quantum effects in all but the simplest renormalizable field
theories; for example, they are generic in all effective field theories.

\emph{Hamiltonian} and \emph{Lagrangian} density: ( How to connect field
$\phi$ with $\mathcal{L}$ or $\mathcal{H}$ ??? ).
\[ \mathcal{L}[\phi,\dot{\phi}] = \pi[\phi, \dot{\phi}]\dot{\phi} -
\mathcal{H}[\phi, \pi[\phi, \dot{\phi}]]  \]
Or inversely:
\[ \mathcal{H}[\phi,\pi] = \pi\dot{\phi}[\phi, \pi] -
\mathcal{L}[\phi, \dot{\phi}[\phi, \pi]],   
\frac{\partial \mathcal{L}[\phi, \dot{\phi}]}{\partial \dot{\phi} } = \pi\]

The Halmiltonian corresponds to a conserved quantity - the total energy of
the system - while the Lagrangian does not. The problem with Halmiltonian is
that they are not Lorentz invariant. It is the 0 component of a Lorentz
vector: $P^\mu = (H, \vec{P})$. And $\mathcal{H}$ is the 00 compnent of a
Lorentz tensor, the energy-momentum tensor $\mathcal{T}_{\mn}$. Halmiltonians
are great for non-relativistic systems, but for relativistic systems we will
almost exclusively use Lagrangians.

Time evolution is generated by a hamiltonian H.
$i\hbar\partial_{t}\ket{\phi} = H\ket{\phi}$

\subsection{Norther's theorem}
If there is such a symmetry that depends on some parameter $\alpha$ that can
be taken small (continuous), than we find:
\[ 0 = \frac{\delta\mathcal{L}}{\delta\alpha} =
\displaystyle\sum_n\left\{ \left[\frac{\partial\mathcal{L}}{\partial{\phi_n}} - 
\partial_\mu\frac{\partial\mathcal{L}}{\partial(\partial_\mu\phi_n)}\right]\frac{\delta\phi_n}{\delta\alpha} 
+
\partial_\mu\left[\frac{\partial\mathcal{L}}{\partial(\partial_\mu\phi_n)}\frac{\delta\phi_n}{\delta\alpha}\right]
\right\} \]
If the EOM is satisfied, then it reduces to $\partial\mu J_\mu = 0$, where 
\begin{equation}
    \label{Norther current}
    J_\mu =
    \displaystyle\sum_n\frac{\partial\mathcal{L}}{\partial(\partial_\mu\phi_n)}\frac{\delta\phi_n}{\delta\alpha}
\end{equation}
A vector field $J_\mu$ that satisfies $\partial_\mu J_\mu = 0$ is called
\emph{conserved current}. The total charge Q, defined as: 
\[ Q = \int d^3xJ_0 \],
satisfies 
\[ \partial_{t}Q = \int d^3x\partial_{t}J_0 = \int
d^3x\vec{\nabla}\cdot\vec{J} = 0 \]
\textbf{Neother's theorem}: If a Lagrangian has a continuous symmetry then
there exists a current associate with that symmetry that is conserved when
the equations of motion are satisfied.
\begin{itemize}
    \item continuous
    \item the current is conserved \textit{on-shell}, that is, when the EOM
	are satisfied. (The field dies out at asymptotic boundary).
    \item It works for \textit{global symmetries}, parametrized by number
	$\alpha$, not only for \textit{local(gauge) symmetries} parametrized
	by functions $\alpha(x)$.
\end{itemize}

\subsection{Energy-Momentum Tensor}
\textit{global} space-time translation $\rightarrow$ energy-momentum tensor.  \\
\[ \phi(x) \rightarrow \phi(x+\xi) = \phi(x) + \xi^\nu\partial_\nu\phi(x) + \cdots \]
With infinitesimal change:
\[ \frac{\delta\phi}{\delta\xi^\nu} = \partial_\nu\phi, 
\frac{\delta\mathcal{L}}{\delta\xi^\nu} = \partial_\nu\mathcal{L}, 
\]
So:
\[ \delta S = \int d^4x\delta\mathcal{L} = \xi^\nu\int
d^4x\partial_\nu\mathcal{L} = 0 \]
which leads to:
\[ \partial_\nu\mathcal{L} =
\frac{\delta\mathcal{L}[\phi_n,\partial_\mu\phi_n]}{\delta\xi^\nu} = 
\partial_\mu\left(
\displaystyle\sum_n\frac{\partial\mathcal{L}}{\partial(\partial_\mu\phi_n)}\frac{\delta\phi_n}{\delta\xi^\nu}
\right)
    \]
Or equivalently
\[
\partial_\mu\left(
\displaystyle\sum_n\frac{\partial\mathcal{L}}{\partial(\partial_\mu\phi_n)}\partial_\nu\phi_n
- g_{\mn}\mathcal{L}
\right) = 0
    \]
The four symmetries have produced four Noether current, one for each $\nu$:
\begin{equation}
    \mathcal{T}_{\mn} = 
    \displaystyle\sum_n\frac{\partial\mathcal{L}}{\partial(\partial_\mu\phi_n)}\partial_\nu\phi_n
    - g_{\mn}\mathcal{L}
\end{equation}
The corresponding conserved current:
\[
    Q_\nu = \int d^3x\mathcal{T}_{0\nu} \]


Electron has an inherent two-valuedness called spin, while a photon has
an inherent two-valuedness called polarization.


\section{Symmetry}

\subsection{Lorentz invariance} the symmetry group associated with
\emph{special relativity}. \\
For scalar field, the physical content of Lorentz invariance is 
that nature has a symmetry under which scalar fields do not transform. 
Take, for example, the temperature of a fluid, which can vary from 
point to point. If we change reference frames, the labels for the 
points change, but the temperature at each point stays the same.

As for vector field, the difference is that the compnents of a vector field
at the point x transform into each other as well.

The simplest Lorentz-invariant operator that we can write down involving
derivatives is the d'Alembertian:
\[ \Box = \partial_\mu^2 = \partial^2_t - \partial^2_x - \partial^2_y -
\partial^2_z \]

Objects such as $v^2 = V_\mu V^\mu, \phi, 1, \partial_\mu V^\mu$ are
\emph{Lorentz invariant}, meaning they do not depend on the Lorentz frame at
all. While objects like: $V_\mu, F_\mn, \partial_\mu, x_\mu$ are
\emph{Lorentz covariant}, meaning they do change in different frames, but
precisely as the Lorentz transformation dictates.
\subsection{Unitary} the probability should add up to \emph{1}.

\subsection{\Poincare{} Group} 
The group of translations and Lorentz transformation is called the
\textbf{\Poincare{} group}, 
ISO(1,3) (the isometry group of Minkowski space).

\section{Field}
\subsection{Scalar Field}
$$ \mathcal{L} = \frac{1}{2}\dot{\phi}^{2} -
\frac{1}{2}\vec{\nabla}\phi \cdot \vec{\nabla}\phi - \frac{1}{2}m^{2}\phi^{2} 
= \frac{1}{2}(\partial_{\mu}\phi\partial^{\mu}\phi - m^{2}\phi^{2})$$
EOM:
$$ -\partial_{t}^{2}\phi + \nabla^{2}\phi - m^{2}\phi =
(\partial_{\mu}\partial^{\mu} + m^{2}) \phi = 0 $$
$$ \pmb{\phi}(\vec{x}) = \int
\frac{d^{d}k}{(2\pi)^{d}}\sqrt{\frac{\hbar}{2\omega_{k}}}(e^{i\vec{k}\cdot\vec{x}}a_{k}+e^{-i\vec{k}\vec{x}}a_{k}^{\dag})$$
$$ \pmb{\pi}(\vec{x}) = \frac{\partial{\mathcal{L}}}{\partial_{\mu}\phi} =
\frac{1}{i} \int \frac{d^{d}k}{(2\pi)^{d}}\sqrt{\frac{\hbar\omega_{k}}{2}}(e^{i\vec{k}\cdot\vec{x}}a_{k}-e^{-i\vec{k}\vec{x}}a_{k}^{\dag}) $$
$$ [\phi(\vec{x}), \pi(\vec{x'})] = i\hbar\delta^{d}(\vec{x} - \vec{x'})$$
$$ H =\sum_{n}(p_{n}\dot{q}_n) = \int dx(\pi(x)\dot{q}(x) -
\mathcal{L})  $$


\subsection{\KG{} Field}
For a massive scalar (spin 0) and neutral (charge 0) field:
$$\mathcal{L} = \frac{1}{2} [(\partial_{\mu}\phi)(\partial^{\mu}\phi) -
m^{2} \phi^{2}]$$
The Euler-Lagrange formula:
$$ (\Box + m^{2})\phi = 0$$
This is the \KG{} equation. It was quantized by Pauli and Weisskopf in 1934.
The Klein-Gordon equation was historically rejected as a fundamental quantum
equation because it predicted negative probability density.

\subsection{Dirac Field}
Dirac was looking for an equation linear in E or in $\partial_t$. For a
massive spinor (spin $1/2$) field the Lagrangian density is:
$$ \mathcal{L} = \bar{\psi}(i\gamma^{\mu}\partial_{\mu} - m)\psi; \quad
\bar{\psi} = \psi^{*}\gamma^{0} \quad \text{Dirac adjoint} $$
The $4 \times 4$ Dirac matrices $\gamma^{\mu} (\mu = 0,1,2,3)$ satisfy the
Clifford algebra:
$$ \{\gamma^{\mu}, \gamma^{\nu}\} = \gamma^{\mu} \gamma^{\nu} +
\gamma^{\nu} \gamma^{\mu} = 2g^{\mn}$$
The corresponding EOM is the Dirac equation:
$$ (i\gamma^{\mu}\partial_{\mu} - m) \psi =0;	\quad
i(\partial_{\mu}\bar{\psi})\gamma^{\mu} + m\bar{\psi} = 0$$
The $\gamma$ matrices:
$$ \gamma^{0} = 
    \begin{pmatrix}
	I   & 0	\\
	0   & I	
    \end{pmatrix}, \quad
    \gamma^{i} = 
    \begin{pmatrix}
	0   & \sigma^{i}    \\
	-\sigma^{i} & 0	    
    \end{pmatrix}
$$
Where \textbf{$\sigma$} is the Pauli matrices:
\begin{equation}
    \sigma^{1} = 
	\begin{pmatrix}
	    0	& 1 \\
	    1	& 0 
	\end{pmatrix},	\quad
    \sigma^{2} = 
	\begin{pmatrix}
	    0	& -i \\
	    i	& 0 
	\end{pmatrix},	\quad
    \sigma^{3} = 
	\begin{pmatrix}
	    1	& 0 \\
	    0	& -1 
	\end{pmatrix}
    \label{Pauli Matrices}
\end{equation}
Quantization of the Dirac field is achieved by replacing the spinors by
field operators and using the Jordand and Wigner quantization rules.
Heisenberg's EOM for the field operator $\hat{\psi}(\vec{x}, t)$
reads:
$$ i\partial_{t}\hat{\psi}(\vec{x}, t) = [
    \hat{\psi}(\vec{x}, t), \hat{H}]$$
There are both positive and negative eigenvalues in the energy spectrum. The
later are problematic in view of Einstein’s energy of a particle at rest 
$ E = mc^2 $. Dirac’s way out of the negative energy catastrophe was to 
postulate a Fermi sea of antiparticles. This genial assumption was not 
taken seriously until the positron was discovered in 1932 by Anderson.


\subsection{Maxwell Field}
\begin{equation}
    \label{Maxwell Eqn}
\end{equation}

\[ \mathcal{L} = -\frac{1}{4}F^2_{\mn} - j_{\mu}A^{\mu} 
=-\frac{1}{4}(\partial_{\mu}A_\nu-\partial_{\nu}A_\mu)^2 - j_{\mu}A^{\mu} \\
=-\frac{1}{2}(\partial_{\mu}A_\nu) + \frac{1}{2}(\partial_{\mu}A_\mu)^2 - j_{\mu}A^{\mu}
=(E^2 - B^2)/2 - \phi V + \vec{j}\vec{A} \]
where $J_\mu$ is the external current:
\[ J_\mu(x) = \left\{ 
\begin{aligned}
    J_0(x) = \rho(x) \\
    J_i(x) = v_i(x)
\end{aligned}
\right.\]

Field strength tensor 
\[ F^{\mn} = \partial^{\mu}A^{\nu} - \partial^{\nu}A^{\mu} \] 
with components 
\[ F^{0i} = \partial^{0}A^{i} - \partial^{i}A^{0} = -E^{i}, 
\quad
F^{ij} = \partial^{i}A^{j} - \partial^{j}A^{i} = -\varepsilon^{ijk}B^{k} \]
EOM
$\frac{\partial\mathcal{L}}{\partial{A_\nu}}-\partial_\mu\frac{\partial\mathcal{L}}{\partial(\partial_{\mu}A_\nu)}=0$
\[-J_\nu-\partial_\mu(-\partial_{\mu}A_\nu)-\partial_\nu(\partial_{\mu}A_\mu)=0\]
which gives
\[\partial_{\mu}F_{\mn}=J_\nu\]
Lorentz gauge: $\partial_{\mu}A_{\mu}=0$
\[J_{\nu}=\partial_{\mu}(\partial_{\mu}A_\nu-\partial_{\nu}A_{\mu})=\Box{A_\nu}-\partial_{\nu}(\partial_{\mu}A_{\mu})
=\Box{A_{\nu}}\]
so
\[A_{\nu}(x)=\frac{1}{\Box}J_{\nu}(x)\]
\textbf{propagator}
\[\Pi_A=\frac{1}{\Box}\]
Note that the propagator has nothing to do with the source. In fact it is
entirely determined by the kinetic terms for a field.

\subsection{Proca (Massive Vector Boson) Field}
$$ \mathcal{L} = -\frac{1}{4}F_{\mn}F^{\mn} +
\frac{1}{2}m^{2}A_{\mu}A^{\mu} - j_{\mu}A^{\mu} $$
EOM:
$$ \Box A^{\nu} - \partial^{\mu}(\partial_{\mu}A^{\mu}) + m^{2}A^{\nu} =
j^{\nu}$$
