\newcommand{\QM}{Quamtum Mechanics}
\newcommand{\QFT}{Quantum Field Theory}
\newcommand{\RQFT}{Relativistic Quantum Field Theory}
\newcommand{\FT}{Fourier Transform}
\newcommand{\FFT}{Fast Fourier Transform}
\newcommand{\LT}{Lorentz Transform}
\newcommand{\LI}{Lorentz Invariant}
\newcommand{\LG}{Lorentz Group}
\newcommand{\KG}{Klein-Gordon}
\newcommand{\EL}{Euler-Lagrange}
\newcommand{\sr}{special relativity}
\newcommand{\Poincare}{Poincar$\acute{\textrm{e}}$}

\chapter{QFT}

\begin{itemize}
    \item What's a field, classical field is a spatial distribution, quantum
	field is a analogy to classical one, but make up of creation and
	destruction operator. The problem is that we \textbf{define} a field which
	collects creator and destructor operators and it works !!! What's
	the logic to define such a field ???
    \item How to construct Lagrangian from a field $\leftarrow$ Lorantz
	invariant.
    \item EOM
    \item Noether theorem. Conserved current and charge
    \item Symmetry. What's each group? corresponding representation. How to
	embed particles into lorentz group and unitary group?
    \item Couple of scalar field to $A_\mu$
\end{itemize}

%%%%%%%%%%%%%%%%%%%%%%%%%%%%%%%%%%%%%%%%%%%%%%%%%%%%%%%%%%%%%%%%%%%%%%%
\section{Motivation}

\emph{particle} number is not conserved. The creation and destruction of
particls, which is possible due to the most famous eqn. of \sr{} $E =
mc^{2}$. \emph{Lorentz invariance} guides the definition of particle.

Why QFT: \fbox{quantum mechanics plus \Poincare{} symmetry}. \\
\fbox{Quantum field theory is just quantum mechanics with an infinite number
of harmonic oscillators}

The \QM{} can describe a system with a fixed number of particles
in terms of a many-body wave function. The \RQFT{}
with creation and annihilation operators was developed in order to include
processes in which the number of particles is not conserved,
and to describe the conversion of mass into energy and vice versa.
A consequence of relativity is that the number of particles isn't fixed, ($E=mc^2$)
though the converse is false: particle production can happen without relativity.

Construct $\mathcal{L}$ from field $\phi$ and its derivative under the rule
of \LI{}. How to incorporate symmetry ?

QFT is the quantum mechanics of {\Large \textit{extensive degrees of freedom}}
$\bra{x}\ket{\phi} = \phi(x)$ is a function of space, the wavefunction.
This looks like a field. It is not what we mean by field in QFT.
meaningless phrases like "second quantization" may conspire to try
to confuse you.

It is not a coincidence that the harmonic oscillator plays an important
role. After all, electromagenetic waves oscillate harmonically.

There are two common ways to quantize a field theory:
\begin{itemize}
    \item canonical quantization
    \item Feynman path integral
    \item Ohter alternatives: perturbation theory
\end{itemize}

\textbf{Second quantization}: first quantization refers to the discrete
modes($d\vec{x}d\vec{p}\sim \hbar$), for example, of a particle in a box.
Second quantization refers to the integer numbers of excitations of each of
these modes. However, this is somewhat misleading-the fact that there are
discrete modes is a classical phenomenon. The two steps really are (1)
interpret there modes as having energy $E=\hbar\omega$ and (2) quantize each
modes as a harmonic oscillator. In that sense we are only quantizing once.

There are two new features in second quantization:
\begin{enumerate}
    \item We have many quantum mechanical systems – one for each $\vec{p}$  – all at the same time.
    \item We interpret the nth excitation of the $\vec{p}$  harmonic
	oscillator as having n \textbf{particles}.
\end{enumerate}
%%%%%%%%%%%%%%%%%%%%%%%%%%%%%%%%%%%%%%%%%%%%%%%%%%%%%%%%%%%%%%%%%%%%%%%
\section{Convention}
In relativity, the symmetry refer to invariance after the transformation of
\emph{coordinate system}. So the rotation is:
\begin{equation}
    R = 
	\begin{pmatrix}
	    \cos\theta	& \sin\theta	\\
	    -\sin\theta	& \cos\theta
	\end{pmatrix}
\end{equation}
or in other way:
\[ R^{T}\mathds{1}R = \mathds{1} \]
    
%%%%%%%%%%%%%%%%%%%%%%%%%%%%%%%%%%%%%%%%%%%%%%%%
\subsection{4D time-space}   
$dx^{\mu} \equiv (dt, d\vec{x})^{\mu}$ \\
$ ds^{2} = dt^{2} - d\vec{x}\cdot d\vec{x} =
\eta_{\mn}dx^{\mu}dx^{\nu}$	with 
\begin{equation}
    \eta^{\mn} = \eta_{\mn} = 
    \begin{pmatrix}
	+1  & 0	  & 0  & 0	\\
	0   & -1  & 0  & 0	\\
	0   & 0	  & -1 & 0	\\
	0   & 0	  & 0  & -1
    \end{pmatrix}_{\mn}
    \label{Minkowski metric}
\end{equation}
Lorentz transformation acting on 4-vectors are matrices $\Lambda$ satisfying
\[
    \Lambda^T\eta\Lambda = \eta = 
    \begin{pmatrix}
	+1  &	&   &	\\
	    & -1&   &  	\\
	    &	& -1&  	\\
	    &	&   & -1\\
    \end{pmatrix}
    \]
    
Rotation and boost in 4D time-space around x axes is:
\begin{equation*}
    \begin{pmatrix}
	1   &	&   &	\\
	    & 1 &   &	\\
	    &	& \cos\theta_x	& \sin\theta_x	\\
	    &	& -\sin\theta_x	& \cos\theta_x	\\
    \end{pmatrix},
    \begin{pmatrix}
	\cosh\beta_x	& \sinh\beta_x	&   &	\\
	\sinh\beta_x	& \cosh\beta_x	&   &	\\
	    &	& 1 &  	\\
	    &	&   & 1	\\
    \end{pmatrix}
\end{equation*}
    
\subsubsection{Lorentz transformation}
Scalar field
\[
    \phi(x^\mu)\rightarrow\phi((\Lambda^{-1})^\mu_\nu x^\nu)
    \]
Vector field
\[
    V^\mu\rightarrow\Lambda^\mu_\nu V^\nu
    \]
Tensor fields
\[
    T^{\mn}\rightarrow\Lambda^\mu_\alpha\Lambda^]\nu_\beta T^{\alpha\beta}
    \]
%%%%%%%%%%%%%%%%%%%%%%%%%%%%%%%%%%%%%%%%%%%%%%%%
\subsection{quantization}
\[ [a_k, a_p^{\dag}] = (2\pi)^{3}\delta^{3}(\vec{p}-\vec{k}), 
a_p^{\dag}\ket{0} = \frac{1}{\sqrt{2\omega_p}}\ket{\vec{p}} \]
\[
    \mathds{1}=\int\frac{d^3p}{(2\pi)^3}\frac{1}{2\omega_p}\ket{\vec{p}}\bra{\vec{p}}
    \]

\begin{description}
    \item [Function derivatives]
	$\frac{\delta \phi(x)}{\delta \phi(y)} = \delta(x-y)$, 
	\[ 
	\frac{\partial(\partial_\alpha A_\alpha)^2}{\partial(\partial_{\mu}A_\nu)}
	=2(\partial_{\alpha}A_\alpha)\frac{\partial(\partial_{\beta}A_\gamma)}{\partial(\partial_{\mu}A_\nu)}g_{\beta\gamma}
	=2(\partial_{\alpha}A_\alpha)g_{\beta\mu}g_{\gamma\nu}g_{\beta\gamma}
	\]
    \item [notation] $\phi$ and $\pi$ for scalar fields, $\psi, \xi,
	\chi$ for fermions, $A_{\mu}, J_{\mu}, V_{\mu}$ for vectors and
	$h_{\mu},T_{\mu}$ for tensors.
    \item [Kinetic term] Anything with just two fields of the same or
	different type can be called a kinetic term. Kinetic terms tell you
	about the free (non-interacting) behavior. Though, sometimes it is useful to
	think of a \emph{mass term} such as $m^2\phi^2$, as an interaction
	rather than a kinetic term.
    \item [Boundary conditions] we will always assume that our fields vanish
	on asymptotic boundaries. so we can integrate by part:
	\emph{\[ A\partial_\mu B = -(\partial_\mu A) B\]}
\end{description}

We define quantum fields as integrals over creation and annihilation
operators for each momentum: (Why define it this way ???)
\begin{equation}
\phi_0(\vec{x}) = \int \frac{d^{3}p}{(2\pi)^3}
\frac{1}{\sqrt{\omega_p}}(a_pe^{i\vec{p}\vec{x}} + a_p^\dag
e^{-i\vec{p}\vec{x}})
    \label{eqn:free_field}
\end{equation}
\[ \ket{\vec{x}} = \phi_0(\vec{x}) \ket{0} \]

There is no physical content in the above equation. It is just a definition.
The physical content is in the algebra of $a_p$ and $a_p^{\dag}$ and in the
Hamiltonian $H_0$. Nevertheless, we will see that collections of $a_p$ and
$a_p^{\dag}$ in the form of Eq.\ref{eqn:free_field} are very useful in
quantum field theory.

Following this defition, we can get:
\[ \pi(\vec{x}) \equiv \partial_t \phi(\vec{x})|_{t=0} = 
-i \int \frac{d^{3}p}{(2\pi)^3} \sqrt{\frac{\omega_p}{2}}(a_pe^{i\vec{p}\vec{x}} - a_p^\dag e^{-i\vec{p}\vec{x}}) \]
$\pi[\phi, \dot{\phi}]$ can also be implicityly defined as:
\[ \frac{\partial \mathcal{H}[\phi, \pi ]}{\partial \pi } = \dot{\phi}\]

\subsubsection{Schrodinger eqn}
\[
    \begin{aligned}
	i\partial_t\psi{x}&=i\partial_t\bra{x}\ket{\psi}=i\partial_t\bra{0}\phi(\vec{x},t)\ket{\psi}=i\bra{0}\partial_t\phi(\vec{x},t)\ket{\psi}\\
	&=\bra{0}\int\frac{d^3p}{(2\pi)^3}\frac{\sqrt{\vec{p}^2+m^2}}{2\omega_p}(a_pe^{-ipx}-a^\dag_pe^{ipx})\ket{\psi}\\
	&=\bra{0}\sqrt{m^2-\vec{\nabla}^2}\phi_0(x)\ket{\psi}\\
    \end{aligned}
    \]
So
\[
    i\partial_t\psi(x)=\sqrt{m^2-\vec{\nabla}^2}\psi(x)=\left(m-\frac{\vec{\nabla}^2}{2m}+\mathcal{O}(\frac{1}{m^2})\right)\psi(x)
    \]
To get
\[
    i\partial_t\psi(x)=-\frac{\vec{\nabla}^2}{2m}\psi(x)
    \]
%%%%%%%%%%%%%%%%%%%%%%%%%%%%%%%%%%%%%%%%%%%%%%%%
\subsection{Hamiltonian \& Lagrangian}
Why do we restrict to Lagrangians of the form $\mathcal{L}[\phi,
\partial_\mu \phi]$? First of all, this is the form that all "classical" 
Lagrangians had. If only first derivatives are involved, boundary 
conditions can be specified by initial positions and velocities only, in
accordance with Newton's laws. In the quantum theory, if kinetic terms 
have too many derivatives, for example $\mathcal{L} = \phi\Box^2\phi$, 
there will generally be disastrous consequences. For example, there may be
states with negative energy or negative norm, permitting the vacuum to decay. 
But interactions with multiple derivatives may occur. Actually, they must
occur due to quantum effects in all but the simplest renormalizable field
theories; for example, they are generic in all effective field theories.

\emph{Hamiltonian} and \emph{Lagrangian} density: ( How to connect field
$\phi$ with $\mathcal{L}$ or $\mathcal{H}$ ??? ).
\[ \mathcal{L}[\phi,\dot{\phi}] = \pi[\phi, \dot{\phi}]\dot{\phi} -
\mathcal{H}[\phi, \pi[\phi, \dot{\phi}]]  \]
Or inversely:
\[ \mathcal{H}[\phi,\pi] = \pi\dot{\phi}[\phi, \pi] -
\mathcal{L}[\phi, \dot{\phi}[\phi, \pi]],   
\frac{\partial \mathcal{L}[\phi, \dot{\phi}]}{\partial \dot{\phi} } = \pi\]

The Halmiltonian corresponds to a conserved quantity - the total energy of
the system - while the Lagrangian does not. The problem with Halmiltonian is
that they are not Lorentz invariant. It is the 0 component of a Lorentz
vector: $P^\mu = (H, \vec{P})$. And $\mathcal{H}$ is the 00 compnent of a
Lorentz tensor, the energy-momentum tensor $\mathcal{T}_{\mn}$. Halmiltonians
are great for non-relativistic systems, but for relativistic systems we will
almost exclusively use Lagrangians.

Time evolution is generated by a hamiltonian H.
$i\hbar\partial_{t}\ket{\phi} = H\ket{\phi}$

%%%%%%%%%%%%%%%%%%%%%%%%%%%%%%%%%%%%%%%%%%%%%%%%
\subsection{Norther's theorem}
If there is such a symmetry that depends on some parameter $\alpha$ that can
be taken small (continuous), than we find:
\[ 0 = \frac{\delta\mathcal{L}}{\delta\alpha} =
\displaystyle\sum_n\left\{ \left[\frac{\partial\mathcal{L}}{\partial{\phi_n}} - 
\partial_\mu\frac{\partial\mathcal{L}}{\partial(\partial_\mu\phi_n)}\right]\frac{\delta\phi_n}{\delta\alpha} 
+
\partial_\mu\left[\frac{\partial\mathcal{L}}{\partial(\partial_\mu\phi_n)}\frac{\delta\phi_n}{\delta\alpha}\right]
\right\} \]
If the EOM is satisfied, then it reduces to $\partial\mu J_\mu = 0$, where 
\begin{equation}
    \label{Norther current}
    J_\mu =
    \displaystyle\sum_n\frac{\partial\mathcal{L}}{\partial(\partial_\mu\phi_n)}\frac{\delta\phi_n}{\delta\alpha}
\end{equation}
A vector field $J_\mu$ that satisfies $\partial_\mu J_\mu = 0$ is called
\emph{conserved current}. The total charge Q, defined as: 
\[ Q = \int d^3xJ_0 \]
satisfies 
\[ \partial_{t}Q = \int d^3x\partial_{t}J_0 = \int
d^3x\vec{\nabla}\cdot\vec{J} = 0 \]
\textbf{Noether's theorem}: If a Lagrangian has a continuous symmetry then
there exists a current associate with that symmetry that is conserved when
the equations of motion are satisfied.
\begin{itemize}
    \item continuous
    \item the current is conserved \textit{on-shell}, that is, when the EOM
	are satisfied. (The field dies out at asymptotic boundary).
    \item It works for \textit{global symmetries}, parametrized by number
	$\alpha$, not only for \textit{local(gauge) symmetries} parametrized
	by functions $\alpha(x)$.
\end{itemize}

%%%%%%%%%%%%%%%%%%%%%%%%%%%%%%%%%%%%%%%%%%%%%%%%
\subsection{Energy-Momentum Tensor}
There is a very important case of Noether's theorem that applies to a global
symmetry of the action, not the Lagrangian.
\textit{global} space-time translation $\rightarrow$ energy-momentum tensor.  \\
\[ \phi(x) \rightarrow \phi(x+\xi) = \phi(x) + \xi^\nu\partial_\nu\phi(x) + \cdots \]
With infinitesimal change:
\[ \frac{\delta\phi}{\delta\xi^\nu} = \partial_\nu\phi, 
\frac{\delta\mathcal{L}}{\delta\xi^\nu} = \partial_\nu\mathcal{L}, 
\]
So:
\[ \delta S = \int d^4x\delta\mathcal{L} = \xi^\nu\int
d^4x\partial_\nu\mathcal{L} = 0 \]
which leads to:
\[ \partial_\nu\mathcal{L} =
\frac{\delta\mathcal{L}[\phi_n,\partial_\mu\phi_n]}{\delta\xi^\nu} = 
\partial_\mu\left(
\displaystyle\sum_n\frac{\partial\mathcal{L}}{\partial(\partial_\mu\phi_n)}\frac{\delta\phi_n}{\delta\xi^\nu}
\right)
    \]
Or equivalently
\[
\partial_\mu\left(
\displaystyle\sum_n\frac{\partial\mathcal{L}}{\partial(\partial_\mu\phi_n)}\partial_\nu\phi_n
- g_{\mn}\mathcal{L}
\right) = 0
    \]
The four symmetries have produced four Noether current, one for each $\nu$:
\begin{equation}
    \mathcal{T}_{\mn} = 
    \displaystyle\sum_n\frac{\partial\mathcal{L}}{\partial(\partial_\mu\phi_n)}\partial_\nu\phi_n
    - g_{\mn}\mathcal{L}
\end{equation}
The corresponding conserved current:
\[
    Q_\nu = \int d^3x\mathcal{T}_{0\nu} \]


Electron has an inherent two-valuedness called spin, while a photon has
an inherent two-valuedness called polarization.


%%%%%%%%%%%%%%%%%%%%%%%%%%%%%%%%%%%%%%%%%%%%%%%%%%%%%%%%%%%%%%%%%%%%%%%
\section{Lagrangian}
In QFT, we will use Lagrangian rather than Hamiltonian, because it is
Lorentz invariant. To construct Lorentz invariant Lagrangian from field, one
just need to add Lorentz invariant terms into it.

A vector field $A_\mu$ is just four scalars until we construct it with
$\partial_\mu$ in the Lagrangian, as in the $(\partial_{\mu}A_\mu)^2$ part
of $F_{\mn}^2$.

%%%%%%%%%%%%%%%%%%%%%%%%%%%%%%%%%%%%%%%%%%%%%%%%%%%%%%%%%%%%%%%%%%%%%%%
\section{Symmetry}
Our universe has a number of apparent symmetries that we would like our
quantum field theory to respect. One symmetry is that no place in space-time
seems any different from any other place. Thus, our theory should be
translation invariant: if we take all our fields $\phi(x)$ and replace them
by $\phi(x+a)$ for any 4-vector $a^\nu$, the observable should look the
same. Another symmetry is Lorentz invariance: physics should look the same
whether we point our measurement apparatus to the left or to the right, or
put it on a train. The group of translations and Lorentz translations is
called the \textbf{\Poincare{} group}, ISO(1,3) (the isometry group of
Minkowski space).

It is possible for two different groups to have the same algebra. For
example, the proper orthochronous Lorentz group($det(\Lambda)=+1,
\Lambda^0_0>1$) and the full Lorentz group($det(\Lambda)=\pm1,
\Lambda^0_0>1 \text{or} \Lambda^0_0<-1$)
have the same algebra, but the full Lorentz group has in addtion time
reversal and parity. The \textbf{proper} Lorentz group is the special
orthogonal group SO(1,3), which contains only the elements with determinant
1, so it excludes T and P.

Notice that different representations of the same group G can have different
dimensions. \textbf{Lie algebra} relations is the properties of G that are
inherent in all of its representations.
 
 Assume $D_R(g(\theta))$ as a representation of group G, and 
 \[
     D_R(g(\theta\sim0)) = \mathds{1} + i\theta_aT^a_R + \mathcal{O}(\theta^2)
     \]
 where $T^a_R \equiv -i\partial_{\theta_a}D_R(g(\theta))|_{\theta=0}$ 
 is the \textbf{generator} of G in the representation of R. So
 \[
     D_R(g(\theta)) = e^{i\theta_aT^a_R}
     \]
 Given two such elements $D_R(g(\theta_1)) = e^{i\theta^1_aT^a_R}$, 
 $D_R(g(\theta_2)) = e^{i\theta^2_aT^a_R}$, their product must give a third:
 \[
     D_R(g_1)D_R(g_2) = D_R(g_1g_2) = e^{i\theta^3_aT^a_R}
     \]
 Expanding the log of both hand sides to second order in the $\theta$(See
 Maggiore chapter 2.1), we get:
 \[
     \theta^3_a = \theta^1_a + \theta^2_a - \frac{1}{2}f^{bc}{}_a +\mathcal{O}(\theta^3)
     \]
 which implies that:
 \[
     [T^a, T^b] = if^{ab}{}_cT^c
     \]
 This is the \emph{Lie algebra} of G and the \emph{f} is the \emph{structure
 constants} of G. f does not depend on the representation. Note that the
 normalization of the $T^a$ is ambiguous, and rescaling T rescales f. A
 common convention is to choose an orthonormal basis:
 \[
     tr(T^aT^b) = \frac{1}{2}\delta^{ab}
     \]

 The Lie algebra is defined in the neighborhood of the identity element, but
 by conjugating by finite transformations, the tangent space to any point on
 the group has the same structure, so it determines the local structure. 
 It doesn't know about global, discrete issues, like disconnected components, 
 so different groups can have the same Lie algebra.

 A \emph{casimir} of the algebra is an operator made from the generators
 which commutes with all of them.

 The notation of "axis of rotation" is (d=3)-centric. More generally, 
 a rotaion is specified by a (2D) \emph{plane} of rotation. 
 In d=3, we can specify a plane by its normal
 direction, the one that's left out, $J^i \equiv \epsilon^{ijk}J^{jk}$, in
 terms of which the so(3) lie algebra is 
 \[
     [J^{ij}, J^{kl}] = i(\delta^{jk}J^{il}+\delta^{il}J^{jk}-\delta^{ik}J^{jl}-\delta^{jl}J^{ik})
     \]
 The vector representation is 
 \[
     \left(J^{ij}_{(1)}\right)^k{}_l = i(\delta^{ik}\delta^j_l - \delta^{jk}\delta^i_l)
     \]
 The spinor representation is 
 \[
     J^{ij}_{(\frac{1}{2})} = \epsilon^{ijk}\frac{1}{2}\sigma^k =
     \frac{i}{4}[\sigma^i, \sigma^j]
     \]
 For general \emph{d}, we can make a spinor representation of dimension k
 (k=2J+1) if we find d $k\times k$ matrices $\gamma^i$ which satisfy the
 \emph{Clifford algebra} $\{\gamma^i, \gamma^j\} = 2\delta^{ij}$.

 For (faithful) representations of \emph{non-compact} groups, 'unitary' and
 'finite-dimensional' are mutually exclusive.
%%%%%%%%%%%%%%%%%%%%%%%%%%%%%%%%%%%%%%%%%%%%%%%%
\subsection{Lorentz group} 

the symmetry group associated with \emph{special relativity}. \\

For scalar field, the physical content of Lorentz invariance is 
that nature has a symmetry under which scalar fields do not transform. 
Take, for example, the temperature of a fluid, which can vary from 
point to point. If we change reference frames, the labels for the 
points change, but the temperature at each point stays the same.

As for vector field, the difference is that the compnents of a vector field
at the point x transform into each other as well.

The simplest Lorentz-invariant operator that we can write down involving
derivatives is the d'Alembertian:
\[ \Box = \partial_\mu^2 = \partial^2_t - \partial^2_x - \partial^2_y -
\partial^2_z \]

Objects such as $v^2 = V_\mu V^\mu, \phi, 1, \partial_\mu V^\mu$ are
\emph{Lorentz invariant}, meaning they do not depend on the Lorentz frame at
all. While objects like: $V_\mu, F_\mn, \partial_\mu, x_\mu$ are
\emph{Lorentz covariant}, meaning they do change in different frames, but
precisely as the Lorentz transformation dictates.

The Lorentz group is sometimes called O(1,3). This is orghogonal (preserves
a metric) group corresponding to a metric with (1,3) signature.

The irreducible representations of the Lorentz group can be constructed from
irreducible representations of SU(2). 
\[
    J^+_i \equiv \frac{1}{2}(J_i + iK_i), 
    J^-_i \equiv \frac{1}{2}(J_i - iK_i)
\]
which satisfy
\[
    \begin{aligned}
	[J^+_i, J^+_j] &= i\epsilon_{ijk}J^+_k    \\
	[J^-_i, J^-_j] &= i\epsilon_{ijk}J^-_k    \\
	[J^+_i, J^-_j] &= 0 
    \end{aligned}
\]

%%%%%%%%%%%%%%%%%%%%%%%%
\subsubsection{Lorentz algebra}
Lorentz algebra, so(1,3)
\[
    \begin{aligned}
	[J_i, J_j] &= i\epsilon_{ijk}J_k    \\
	[J_i, K_j] &= i\epsilon_{ijk}K_k    \\
	[K_i, K_j] &= -i\epsilon_{ijk}J_k    \\
    \end{aligned}
\]
Note that $ [J_i, J_j] = i\epsilon_{ijk}J_k $ is also the algebra for
rotations, SO(3), and in fact the $J_i$ generate the 3D rotation subgroup of
the Lorentz group.

Similar to rotation group, we can generalize to other SO(1,d) by collecting
the generators into an antisymmetric matrix $J^{\mu}$ with components
$J^{ij} = \epsilon^{ijk}J^k, J^{0i} = K^I = -J^{i0}$(exactly as $\vec{E},
\vec{B}$ are collected into $F^{\mu}$). This object satisfies:
\[
    J^{\mn}=
    \begin{aligned}
	0   & K_1   & K_2   & K_3   \\
	-K_1	& 0 & J_3   & -J_2  \\
	-K_2	& -J_3	& 0 & J_1   \\
	-K_3	& J_2	& -J_1	& 0 \\
    \end{aligned}
    \]
\[
    [J^{\mn}, J^{\rho\sigma}]=
    i( \eta^{\nu\rho}J^{\mu\sigma}
    +\eta^{\mu\sigma}J^{\nu\rho} 
    -\eta^{\mu\rho}J^{\nu\sigma}
    -\eta^{\nu\sigma}J^{\mu\rho}
    )
    \]
The fundamental (d+1 Dimentional vector) representation matrices solving
this equation are 
\[
    (J^{\mn})^\rho{}_\sigma = i(\eta^{\nu\rho}\delta^{\mu}_\sigma -
    \eta^{\mu\rho}\delta^{\nu}_{\sigma})
    \]
\[
    [S^{\mn}, S^{\rho\sigma}]=
    i( g^{\nu\rho}S^{\mu\sigma}
    -g^{\mu\rho}S^{\nu\sigma}
    -g^{\nu\sigma}S^{\mu\rho}
    +g^{\mu\sigma}S^{\nu\rho} )
    \]
The Lie algebra is independent of any concrete representation, though we
derive them from 4-vector representation. They holds for any representation.

%%%%%%%%%%%%%%%%%%%%%%%%
\subsubsection{represention}
The Dirac representation of the Lorentz group is reducible; it is the direct
sum of a left-handed and a right-hadned spinor representation.

\subsubsection{$\gamma^5$}
\begin{itemize}
    \item $(\gamma^5)^2 = \mathcal{1}$
    \item $\left{\gamma^5,\gamma^\mu\right}=0$
    \item Extended Clifford algebra: $\left{\gamma^M,\gamma^N\right}=2g^{MN}$, with $\gamma^M =
	\gamma^0,\gamma^1,\gamma^2,\gamma^3,i\gamma^5$, and $g^{MN}=diag(1, -1, -1, -1, -1)$ 
\end{itemize}

%%%%%%%%%%%%%%%%%%%%%%%%%%%%%%%%%%%%%%%%%%%%%%%%
\subsection{Unitary group} 
the probability should add up to \emph{1}.
\textbf{U(N)} is defined by its N dimensional representation as $N\times N$
complex unitary matrices $\mathds{1} = M^{\dag}M = MM^{\dag}$. This one
doesn't arise as a spacetime symmetry, but is crucial in the study of gauge
theory. 

Unitary representation, which means under the representation, the group
element $\mathcal{P}$ has $\mathcal{P}^\dag\mathcal{P}=1$. But why the
unitarity depends on representation? shouldn't it be representation
independent.

%%%%%%%%%%%%%%%%%%%%%%%%
\subsubsection{Majorana represention}
\begin{equation}
    \gamma^{0} = 
    \begin{pmatrix}
	0   &	\sigma^2    \\
	\sigma^2    & 0     \\
    \end{pmatrix},
    \gamma^{1} = 
    \begin{pmatrix}
	i\sigma^3    & 0     \\
	0   &	i\sigma^3    \\
    \end{pmatrix},
    \gamma^{2} = 
    \begin{pmatrix}
	0   &	-\sigma^2    \\
	\sigma^2    & 0     \\
    \end{pmatrix},
    \gamma^{3} = 
    \begin{pmatrix}
	-i\sigma^1    & 0     \\
	0   &	-i\sigma^1    \\
    \end{pmatrix}
\end{equation}
The Majorana is another $(\frac{1}{2}, 0)\bigoplus (0, \frac{1}{2})$
representation of the Lorentz group that is phhysically equivalent to the
Weyl representation.
%%%%%%%%%%%%%%%%%%%%%%%%%%%%%%%%%%%%%%%%%%%%%%%%
\subsection{\Poincare \ Group} 
The group of translations and Lorentz transformation is called the
\textbf{\Poincare{} group}, 
ISO(1,3) (the isometry group of Minkowski space).

\textbf{Particle} can be defined as a set of states that mix only among
themselves under \Poincare{} transformations.

Particles transform under irreducible unitary representations of the
\Poincare{} group.

There are \emph{no finite-dimensional unitary representations of the
\Poincare{} group}

%%%%%%%%%%%%%%%%%%%%%%%%%%%%%%%%%%%%%%%%%%%%%%%%
\subsection{Gauge symmetry}
Under a gauge transformation, $\phi$ can transform as 
\[
    \phi\rightarrow e^{-i\alpha(x)}\phi
    \]
But the derivatives $|\partial_\mu \phi|^2$ is not invariant. we can make
the kinetic term gauge invariant using something called a covariant
derivative. Let
\[
    A_\mu \rightarrow A_\mu + \frac{1}{e}\partial_\mu \alpha
    \]
So
\[
    (\partial_\mu+ieA_\mu)\phi\rightarrow(\partial_\mu+ieA_\mu+i\partial_\mu\alpha)e^{-i\alpha(x)}\phi=e^{-i\alpha(x)}(\partial_\mu+ieA_\mu)\phi
    \]
This leads to the definition of \emph{covariant derivative}
\[
    D_\mu\phi\equiv(\partial_\mu+ieA_\mu)\phi\rightarrow{e^{-i\alpha(x)}D_\mu\phi}
    \]
%%%%%%%%%%%%%%%%%%%%%%%%%%%%%%%%%%%%%%%%%%%%%%%%%%%%%%%%%%%%%%%%%%%%%%%
\section{Field}
%%%%%%%%%%%%%%%%%%%%%%%%%%%%%%%%%%%%%%%%%%%%%%%%
\subsection{Scalar Field}
$$ \mathcal{L} = \frac{1}{2}\dot{\phi}^{2} -
\frac{1}{2}\vec{\nabla}\phi \cdot \vec{\nabla}\phi - \frac{1}{2}m^{2}\phi^{2} 
= \frac{1}{2}(\partial_{\mu}\phi\partial^{\mu}\phi - m^{2}\phi^{2})$$
EOM:
$$ -\partial_{t}^{2}\phi + \nabla^{2}\phi - m^{2}\phi =
(\partial_{\mu}\partial^{\mu} + m^{2}) \phi = 0 $$
$$ \pmb{\phi}(\vec{x}) = \int
\frac{d^{d}k}{(2\pi)^{d}}\sqrt{\frac{\hbar}{2\omega_{k}}}(e^{i\vec{k}\cdot\vec{x}}a_{k}+e^{-i\vec{k}\vec{x}}a_{k}^{\dag})$$
$$ \pmb{\pi}(\vec{x}) = \frac{\partial{\mathcal{L}}}{\partial_{\mu}\phi} =
\frac{1}{i} \int \frac{d^{d}k}{(2\pi)^{d}}\sqrt{\frac{\hbar\omega_{k}}{2}}(e^{i\vec{k}\cdot\vec{x}}a_{k}-e^{-i\vec{k}\vec{x}}a_{k}^{\dag}) $$
$$ [\phi(\vec{x}), \pi(\vec{x'})] = i\hbar\delta^{d}(\vec{x} - \vec{x'})$$
$$ H =\sum_{n}(p_{n}\dot{q}_n) = \int dx(\pi(x)\dot{q}(x) -
\mathcal{L})  $$


%%%%%%%%%%%%%%%%%%%%%%%%%%%%%%%%%%%%%%%%%%%%%%%%
\subsection{\KG{} Field}
For a massive scalar (spin 0) and neutral (charge 0) field:
$$\mathcal{L} = \frac{1}{2} [(\partial_{\mu}\phi)(\partial^{\mu}\phi) -
m^{2} \phi^{2}]$$
The Euler-Lagrange formula:
$$ (\Box + m^{2})\phi = 0$$
This is the \KG{} equation. It was quantized by Pauli and Weisskopf in 1934.
The Klein-Gordon equation was historically rejected as a fundamental quantum
equation because it predicted negative probability density.

%%%%%%%%%%%%%%%%%%%%%%%%%%%%%%%%%%%%%%%%%%%%%%%%
\subsection{Dirac Field}
Dirac was looking for an equation linear in E or in $\partial_t$. For a
massive spinor (spin $1/2$) field the Lagrangian density is:
$$ \mathcal{L} = \bar{\psi}(i\gamma^{\mu}\partial_{\mu} - m)\psi; \quad
\bar{\psi} = \psi^{*}\gamma^{0} \quad \text{Dirac adjoint} $$
The $4 \times 4$ Dirac matrices $\gamma^{\mu} (\mu = 0,1,2,3)$ satisfy the
Clifford algebra:
\[ 
    \{\gamma^{\mu},\gamma^{\nu}\}=\gamma^{\mu}\gamma^{\nu}+\gamma^{\nu}\gamma^{\mu}=2g^{\mn} 
\]

Define:
\[
    \sigma^{\mn} \equiv \frac{i}{2}[\gamma^{\mu}, \gamma^{\nu}]
    \]
Any set of four matrices satisfying this equation can be combined to form a
4d representation of so(1,3) in the form
\[
    J^{\mn}_{Dirac} \equiv \frac{i}{4}[\gamma^\mu, \gamma^\nu]
    \]
The corresponding EOM is the Dirac equation:
$$ (i\gamma^{\mu}\partial_{\mu} - m) \psi =0;	\quad
i(\partial_{\mu}\bar{\psi})\gamma^{\mu} + m\bar{\psi} = 0$$
To be explicit, this is shorthand for
\[
    (i\gamma^\mu_{\alpha\beta}\partial_\mu-m\delta_{\alpha\beta})\psi_\beta = 0
    \]
    The Dirac ($\gamma$) matrices:
\[
    \gamma^{\mu} = 
    \begin{pmatrix}
	0   & \sigma^{\mu}  \\
	\bar{\sigma}^{\mu}  & 0	\\
    \end{pmatrix}
\]
\[
    \gamma^5\equiv{i}\gamma^0\gamma^1\gamma^2\gamma^3
    \]
where
\[
    \sigma^\mu\equiv(\mathds{1},\vec{\sigma}),
    \bar{\sigma}^\mu\equiv(\mathds{1},-\vec{\sigma}),
\]
\[ 
    \gamma^{0} = 
    \begin{pmatrix}
	0   & I	\\
	I   & 0	\\	
    \end{pmatrix}, \quad
    \gamma^{i} = 
    \begin{pmatrix}
	0   & \sigma^{i}    \\
	-\sigma^{i} & 0	    \\
    \end{pmatrix}
\]
Where \textbf{$\sigma$} is the Pauli matrices:
\begin{equation}
    \sigma^{1} = 
	\begin{pmatrix}
	    0	& 1 \\
	    1	& 0 
	\end{pmatrix},	\quad
    \sigma^{2} = 
	\begin{pmatrix}
	    0	& -i \\
	    i	& 0 
	\end{pmatrix},	\quad
    \sigma^{3} = 
	\begin{pmatrix}
	    1	& 0 \\
	    0	& -1 
	\end{pmatrix}
    \label{Pauli Matrices}
\end{equation}
\[  \vec{\sigma} = (\sigma_1, \sigma_2, \sigma_3)   \]
\[  \vec{\sigma}^\dag=\vec{\sigma}\]
Commutation relation is
\[  
    \{\sigma^i, \sigma^j\} = 2\delta^{ij}, 
    [\sigma^i, \sigma^j] = 2i\epsilon^{ijk}\sigma^k	\]
In the Weyl basis, the Dirac equation is:
\[
    \begin{pmatrix}
	-m  & i\sigma^\mu\partial_\mu	\\
	i\bar{\sigma}^\mu\partial_\mu	& -m	\\
    \end{pmatrix}
    \begin{pmatrix}
	\psi_L	\\
	\psi_R	\\
    \end{pmatrix} = 0
\]
Quantization of the Dirac field is achieved by replacing the spinors by
field operators and using the Jordand and Wigner quantization rules.
Heisenberg's EOM for the field operator $\hat{\psi}(\vec{x}, t)$
reads:
$$ i\partial_{t}\hat{\psi}(\vec{x}, t) = [
    \hat{\psi}(\vec{x}, t), \hat{H}]$$
There are both positive and negative eigenvalues in the energy spectrum. The
later are problematic in view of Einstein’s energy of a particle at rest 
$ E = mc^2 $. Dirac’s way out of the negative energy catastrophe was to 
postulate a Fermi sea of antiparticles. This genial assumption was not 
taken seriously until the positron was discovered in 1932 by Anderson.

\subsection{Weyl spinor}
\[  \sigma^\mu\partial_\mu\psi=\mathds{1}\partial_t\psi-\partial_i\sigma_i\psi=0    \]
Dirac equation for a Weyl spinor:
\[  \sigma^\mu\parital_\mu\psi=0    \]

%%%%%%%%%%%%%%%%%%%%%%%%%%%%%%%%%%%%%%%%%%%%%%%%
\subsection{Maxwell Field}
\begin{equation}
    \label{Maxwell Eqn}
\end{equation}

\[ 
\begin{aligned}
    \mathcal{L} &= -\frac{1}{4}F^2_{\mn} - j_{\mu}A^{\mu} 
    =-\frac{1}{4}(\partial_{\mu}A_\nu-\partial_{\nu}A_\mu)^2-j_{\mu}A^{\mu}\\
    &=-\frac{1}{2}(\partial_{\mu}A_\nu)^2+\frac{1}{2}(\partial_{\mu}A_\mu)^2-j_{\mu}A^{\mu}\\
    &=(E^2 - B^2)/2 - \phi V + \vec{j}\vec{A} 
\end{aligned}
    \]
Here we use
\[
    (\partial_{\mu}A_\nu)^2=(\partial_{\nu}A_\mu)^2
    \]
where $J_\mu$ is the external current:
\[ J_\mu(x) = \left\{ 
\begin{aligned}
    J_0(x) = \rho(x) \\
    J_i(x) = v_i(x)
\end{aligned}
\right.\]

Field strength tensor 
\[ F^{\mn} = \partial^{\mu}A^{\nu} - \partial^{\nu}A^{\mu} \] 
with components 
\[ F^{0i} = \partial^{0}A^{i} - \partial^{i}A^{0} = -E^{i}, 
\quad
F^{ij} = \partial^{i}A^{j} - \partial^{j}A^{i} = -\varepsilon^{ijk}B^{k} \]
EOM
$\frac{\partial\mathcal{L}}{\partial{A_\nu}}-\partial_\mu\frac{\partial\mathcal{L}}{\partial(\partial_{\mu}A_\nu)}=0$
\[-J_\nu-\partial_\mu(-\partial_{\mu}A_\nu)-\partial_\nu(\partial_{\mu}A_\mu)=0\]
which gives
\[\partial_{\mu}F_{\mn}=J_\nu\]
Lorentz gauge: $\partial_{\mu}A_{\mu}=0$
\[J_{\nu}=\partial_{\mu}(\partial_{\mu}A_\nu-\partial_{\nu}A_{\mu})=\Box{A_\nu}-\partial_{\nu}(\partial_{\mu}A_{\mu})
=\Box{A_{\nu}}\]
so
\[A_{\nu}(x)=\frac{1}{\Box}J_{\nu}(x)\]
\textbf{propagator}
\[\Pi_A=\frac{1}{\Box}\]
Note that the propagator has nothing to do with the source. In fact it is
entirely determined by the kinetic terms for a field.

%%%%%%%%%%%%%%%%%%%%%%%%%%%%%%%%%%%%%%%%%%%%%%%%
\subsection{Proca (Massive Vector Boson) Field}
$$ \mathcal{L} = -\frac{1}{4}F_{\mn}F^{\mn} +
\frac{1}{2}m^{2}A_{\mu}A^{\mu} - j_{\mu}A^{\mu} $$
EOM:
$$ \Box A^{\nu} - \partial^{\mu}(\partial_{\mu}A^{\mu}) + m^{2}A^{\nu} =
j^{\nu}$$

\subsubsection{spinors}
\begin{description}
    \item[Dirac spinors] have both left- and right-handed compnents. Theny
	can be massive or massless.
    \item[Weyl spinors] are always massless and can be left- or
	right-handed. When embedded in Dirac spinors they satisfy the
	constraint $\gamma_5\psi=\pm\psi$.
    \item[Majorana spinors] are left- or right handed. When embedded in
	Dirac spinors they satisfy the constraint
	$\psi=\psi_C=-i\gamma_2\psi^*$
\end{description}
\section{Terminology}
\begin{description}
    \item [free spinors]
    \item [Dirac spinors]
    \item [Weyl representation]
    \item [spinor representation]
    \item [$(\frac{1}{2}, \frac{1}{2})$ representation]
    \item [projective representation]
    \item [Lorentz generators]
	\[
	    S^{\mn} = \frac{i}{4}[\gamma^\mu, \gamma^\nu]
	\]
\end{description}


Lorentz group O(1,3); tensor representations $\phi,A_\mu, h_{\mn}$, spinor
represontations: Weyl spinors $\phi_L, \phi_R$. A Dirac spinor $\psi$
transforms in the reducible $(\frac{1}{x},0)\oplus(0, \frac{1}{2})$
representaion.
The next step towards quantizing a thory with spinors is to use these
Lorentz group representaions to generate irreducible unitray representaions
of the \Poincare{} group.
