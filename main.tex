\documentclass[12pt]{report}

\usepackage{amsmath, amssymb}
\usepackage{bm}	% mathematical bold font
\usepackage{mathtools}	% formatting text above or below rightarrow
\usepackage{siunitx}	% comprehensive SI units; like angstrom

\usepackage{dsfont}
\usepackage{enumitem}
\usepackage{geometry}
\usepackage[colorlinks]{hyperref}	
\hypersetup{
    colorlinks = true,
    linkcolor = black,
    urlcolor = blue,
}
\usepackage{physics}	% ket, bra
\usepackage{slashed}	% slashed symbols in Dirac equation
\usepackage{imakeidx}	
\usepackage{xcolor}
\usepackage{tikz}
    \usetikzlibrary{arrows, shapes, chains, positioning}
    \usetikzlibrary{decorations.markings}

\title{Physcis}
\author{Weibin Zhang}
\date{March 14, 2018}

\makeindex

\newcommand{\ann}[1]{%
    \begin{tikzpicture}[remember picture, baseline=-0.75ex]%
	\node[coordinate] (inText) {};%
    \end{tikzpicture}%
    \marginpar{%
	\renewcommand{\baselinestretch}{1.0}%
	\begin{tikzpicture}[remember picture]%
	    \definecolor{orange}{rgb}{1,0.5,0}%
	    \draw node[fill=red!20,rounded corners,text width=\marginparwidth] (inNote){\footnotesize#1};%
	\end{tikzpicture}%
    }%
    \begin{tikzpicture}[remember picture, overlay]%
    \draw[draw = orange, thick]
	([yshift=-0.2cm] inText)
	    -| ([xshift=-0.2cm] inNote.west)
	    -| (inNote.west);%
	\end{tikzpicture}%
    }%


\begin{document}
\maketitle

\tableofcontents


%%% Index setting 
\newcommand{\mn}{\mu\nu}
\newcommand{\nm}{\nu\mu}
\newcommand{\ab}{\alpha\beta}
\newcommand{\ba}{\beta\alpha}
\newcommand{\bg}{\beta\gamma}
\newcommand{\gb}{\gamma\beta}
\newcommand{\rd}{\rho\delta}
\newcommand{\dr}{\delta\rho}

%%% QFT 
\newcommand{\QM}{Quamtum Mechanics}
\newcommand{\QFT}{Quantum Field Theory}
\newcommand{\RQFT}{Relativistic Quantum Field Theory}
\newcommand{\FT}{Fourier Transform}
\newcommand{\FFT}{Fast Fourier Transform}
\newcommand{\LT}{Lorentz Transform}
\newcommand{\LI}{Lorentz Invariant}
\newcommand{\LG}{Lorentz Group}
\newcommand{\KG}{Klein-Gordon}
\newcommand{\EL}{Euler-Lagrange}
\newcommand{\sr}{special relativity}
\newcommand{\Poincare}{Poincar$\acute{\textrm{e}}$}
\newcommand{\SR}{special relativity}


%%% Display
% add text above approx symbol
\newcommand\defeq{\stackrel{\smash{\scriptscriptstyle\mathrm{def}}}{=}}



% part 0
%%%%%%%%%%%%%%%%%%%%%%%%%%%%%%%%%%%%%%%%%%%%%%%%%%%%%%%%%%%%%%%%%%%%%%%%
\section{Principles}
\textbf{What makes life easier: periodity, symmetry, linearity (constant), causality, conservation, static, equilibrium, main term, limitation, independent, binary system.}

\textbf{What makes life hard: nonlinearities, feedback, coupling (correlation), fluctuation, perturbation, evolution.}

%%%%%%%%%%%%%%%%%%%%%%%%%%%%%%%%%%%%%%%%%%%%%%%%
\subsection{Symmetry}
For a theory, study the symmetry of its EOM, and then specify how physical quantities transform in order to preserve the symmetry.
\begin{itemize}
    \item Continuous symmetry
	\begin{itemize}
	    \item Gauge symmetry: phase transition
	\end{itemize}
    \item Discrete symmetry
\end{itemize}




%%%%%%%%%%%%%%%%%%%%%%%%%%%%%%%%%%%%%%%%%%%%%%%%
\subsection{others}
\begin{description}[style=nextline]
    \item[correspondence principle]
	which requires that in the limit of large quantum numbers the 
	complete quantum mechanical description must agree with the 
	classical one.
    \item[Uncertainty principle]
	Conjugated variables can't be identified simultaneously. (x \& p, t \& E, $\theta$ \& L)
    \item[Minimum Action principle]
	The action of a process being minimum.
    \item[Practical principle]
	It does not matter that a theory is formulated in terms of infinite quantities as long as observable quantities are finite. Extensive use is made of complex quantities in optics, and there is no objection to that as long as the observables are real.
\end{description}

%%%%%%%%%%%%%%%%%%%%%%%%%%%%%%%%%%%%%%%%%%%%%%%%
\subsection{Rules}
\begin{itemize}
    \item Consistency check
    \item Intuitive physical picture
\end{itemize}

% part 1
\part{Introduction}
\chapter{Introduction}
% \section{Introduction}

EOM: one can derive it from action (the smallest action principle). As for the action, it should obey some symmetric rules, for example, in relativity, the action should be \emph{Lorentz invariant} and \emph{Gauge invariant}. 

In terms of $\dot{q} = \frac{\partial{q(t)}}{\partial t}$, one can also write $q(t) = \int \dot{q}dt$, so $\dot{q}$ and $q$ are equivalent, not one base on another, both can be independent.

The tools of QM: Perturbation theory (P.T. both time-dependent and time-independent)

Two keys objects in QM: operators and wavefunctions, if you keep operators invariant and wavefunction change with time, you get \emph{Schordinger Picture}; on the other hand, if you keep wavefunction invariant and change operators with time, you get \emph{Heisenburg Picture}. No matter in which picture, both operators and wavefunctions dependents on \emph{representations}, usual representations include r rep., k rep., Bloch state rep. etc.

QFT is just QM with an infinite number of Harmonic Oscillator (H.O.)

In QM (QFT), the basic idea behind renormalization -- infinities can appear in intermediate calculations, but they must drop out of physical observables.


% I want to organized this book in such a way that I can find any concepts quickly and as detailed as I wanted:
% 1. index of concepts in alphabetical order within each different subfields
% 2. explanation of each concept (logically ordered)
% 3. More context: motivation, proof, explanation [dimension analysis], physical implication, reltaed concepts, application
    \newcommand{\DOF}{degree of freedom}

\section{Doubts}
Some doubt about physics

\subsection{Questions}
\begin{itemize}
    \item gravitational redshift: will light interact with gravity? When shine 
	a beam of light into the sky, the light doesn't slow down, but gravity 
	does take away some of its energy? (how? light has no mass) -- but this 
	i observed in S2 (a bright, giant star that orbits the black hole 
	SgrA* every 16 years) in 2018
    \item Does mass continuous? Does it has a minimum unit?
\end{itemize}

\subsection{Classical Physics}
\begin{description}[style=nextline]
    \item [Newton's Second Law]	Why it is relationship between acceleration
	and force, rather than velocity and force?
    \item [Action] $S = \int Ldt$, why L, but not other quantities, e.g. H.
    \item [Lagrangian] What's the intuitive physical meaning of 
	$L = E_{k} - E_{p}$
\end{description}

\subsection{Statistical Physics}
\begin{description}[style=nextline]
    \item Now that we have smallest energy principle, does the assumption that
	each state is equivalent in SM violate the smallest energy principle?
    \item [Enthalpy, Free energy, Gibbs Free energy] What do H,F,G stand for? 
	Physical meaning?
    \item [Entropy] How do we know the entropy defined in $\Delta \Gamma =
	\frac{d\Gamma}{dE}\Delta E = e^{S}$ is the same one as we defined in
	$dE = TdS - PdV$ ?
\end{description}


\section{Notation}
\subsection{Electrodynamics}

The reason a conductor puts boundary conditions on the EM field is that the
electrons move around to compensate for an applied field. But there is a
limit on how fast the electrons can move. The resulting cutoff frequency is
called the \textit{plasma frequency}
\begin{description}[style=nextline]
    \item [Maxwell Eqn] 
	\begin{equation}
	    \left.
	    \begin{aligned}
		\vec{\nabla} \times (\vec{\nabla} \varphi) &= 0 \\
		\vec{\nabla} \cdot (\vec{\nabla} \times \vec{A}) &= 0
	    \end{aligned}
	    \right\} 
	    \Rightarrow 
	    \left.
	    \begin{aligned}
		\vec{\nabla} \cdot \vec{E} &= \rho \\
		\vec{\nabla} \times \vec{B} &=
		\frac{1}{c} \vec{j} + \frac{1}{c}\partial_{t}\vec{E}   \\
		\vec{\nabla} \cdot \vec{B} &= 0	\\
		-\vec{\nabla} \times \vec{E} &=
		\frac{1}{c} \partial_{t}\vec{B}   \\
	    \end{aligned}
	    \right\}	
	\end{equation}
	With $\vec{E}$ and $\vec{B}$ still connected
	with each other, there is one \textbf{DOF}.
\end{description}

Now that EM wave can only transport transversed wave ($\vec{k}\cdot\vec{E} =0$), then
how does the static electric potential $\varphi$ could produce $\vec{E}$ in
the direction of $\vec{r}$? So the formation of a static electric field is 
different from the propagation of EM wave.


\subsection{Statistical Physics}
\begin{description}[style=nextline]
    \item [Partisian func] When count from particle (one particle can occupy
	how many states), remember the $\frac{1}{N!}$ factor! When count from
	state (how many particles in one state), don't need $\frac{1}{N!}$.
\end{description}

\subsection{Quantum Mechanics}
\begin{description}[style=nextline]
    \item [Baker-Campbell-Hausdorff]	
	if $[A,B] = \mathbb{C}$, then:	\\
	$$ e^{A}e^{B} = e^{B}e^{A}e^{[A,B]}, \qquad 
	e^{A}e^{B} = e^{A+B+\frac{1}{2}[A,B]}, \qquad
	e^{A+B} = e^{B+A} $$
    \item [Angular Momentum] Rep. of J in the Hilbert Space are discrete
	(finite) $\Leftarrow$ subspace.
	$$Y^m_l(\theta,\phi) = \bra{\theta,\phi}\ket{l,m} =
	\sqrt{\frac{2l+1}{4\pi}\frac{(l-m)!}{(l+m)!}}P^m_l(cos\theta)e^{im\phi} $$
	Where $P_l^m$ is the associate Legendre polynomial:
    \item [Harmonic Oscillator]	What's the physical meaning of the energy
	eigenstate $\ket{\phi}$, what's its relationship to $\ket{x}$ state?
\end{description}

\chapter{Constants}

%%%%%%%%%%%%%%%%%%%%%%%%%%%%%%%%%%%%%%%%%%%%%%%%%%%%%%%%%%%%%%%%%%%%%%%%
\section{Constants}
\renewcommand{\arraystretch}{2}
\begin{table}[h]
    \caption{Symbols}
    \label{tab:symbols}
    \begin{center}
	\begin{tabular}{c}
	    \hline  
	    $\displaystyle \nabla^2 = \frac{1}{r^2}\partial_r(r^2 \partial_r) +
	    \frac{1}{r^2}\frac{1}{\sin\theta} \partial_{\theta}(\sin\theta\partial_{\theta}) +
	    \frac{1}{r^2\sin\theta^2}\partial^2_{\phi}$	\\ 

	    $\displaystyle \nabla^2 = \frac{1}{r}\partial_r (r\partial_r) +
	    \frac{1}{r^2}\partial^2_{\theta} + \partial^2_z$ \\
	    \hline
	\end{tabular}
    \end{center}
\end{table}

\begin{table}[h]
    \caption{Constants}
    \label{tab:constants}
    \begin{tabular}{c | l l}
	    \hline
	    Constant    & S.I.  & Gauss \\
	    \hline
	    h	& $6.626\times10^{-34}$ $J \cdot s$	& $4.1357 \times 10^{-15}$ $eV \cdot s$   \\
	    \hline
	    $\displaystyle \hbar=\frac{h}{2\pi}$	&   & $6.5821 \times 10^{-16} $ $eV \cdot s$	\\
	    \hline
	    $\hbar c$	&   & 200 $MeV \cdot fm$	\\
	    \hline
	    $\epsilon_0$    & 
		$\begin{array}{l}
		    8.854 \times 10^{-12} $ $F \cdot m^{-1} \\
		    8.854 \times 10^{-12} $ $C^{2} \cdot N^{-1} \cdot m^{-2}
		\end{array}$
		&   \\
	    \hline
	    $\displaystyle \frac{1}{4\pi\epsilon_0}$	&
		$\begin{array}{l}
		    8.9876 \times 10^{9}$ $V \cdot m \cdot C^{-1}
		\end{array}$
		&   \\
	    $\displaystyle \frac{e}{4\pi\epsilon_0}$	&
		$\begin{array}{l}
		    1.438 \times 10^{-9}$ $V \cdot m
		\end{array}$
		&   \\
	    $\displaystyle \frac{e^2}{4\pi\epsilon_0}$	&
		$\begin{array}{l}
		    2.301 \times 10^{-28}$ $C \cdot V \cdot m
		\end{array}$
		& $1.438 \times 10^{-9}$ $eV \cdot m$   \\
	    \hline
	    $\mu_0$ &
		$\begin{array}{l}
		    4\pi \times 10^{-7}$ $H \cdot m^{-1}    \\
		    1.257 \times 10^{-6}$ $N \cdot A^{-2}    \\
		    1.257 \times 10^{-6}$ $T \cdot m \cdot A^{-1}    \\
		    1.257 \times 10^{-6}$ $Wb \cdot A^{-1} \cdot m^{-1}	\\
		    1.257 \times 10^{-6}$ $V \cdot s \cdot A^{-1} \cdot m^{-1}\\
		\end{array}$
		&   \\
	    \hline
	    Boltzmann's const k	& $1.308 \times 10^{-8} $ $J \cdot K^{-1}$	&   $8.617 \times 10^{-5}$ $eV \cdot K^{-1}$	\\
	    \hline
	    Stefan-Boltzmann $\sigma$	& $5.67 \times 10^{-8}$ $W \cdot m^{-2} \cdot K^{-4}$& \\
	    \hline
	    Larmor radiation ($v << c$) & $\displaystyle P = \frac{2}{3}\frac{q^2a^2}{4\pi\epsilon_0c^3}$   & $\displaystyle P = \frac{2}{3}\frac{q^2a^2}{c^3}$	\\
	    \hline
    \end{tabular}
\end{table}


\begin{table}[h]
    \caption{Frequently used constants}
    \label{tab:constants}
    \begin{center}
	\begin{tabular}{l|c}
	    \hline
	    Bohr radius	    & $R_n = n^2 a_0$;  $a_0 = 4\pi\epsilon_0 \hbar^2/me^2 = 0.529\times 10^{-8} cm$	\\
	    Bohr Energe	    & $\displaystyle E_n = -\frac{1}{2}\frac{e^2}{4\pi\epsilon_0 a_0} \frac{1}{n^2} = -\frac{1}{2}\frac{\hbar^2}{ma_0^2}\frac{1}{n^2}$ \\
	    Bohr magneton   & $\displaystyle \mu_B=\frac{e}{2m_e}$ \\
	    Fine structure  & $\displaystyle \alpha=\frac{e^2}{4\pi}=\frac{1}{137}$	\\
	    Compton wavelength	&   $\displaystyle \lambda = \frac{\hbar}{mc}$  \\
	    k(photon)	    & $\displaystyle \frac{10^4}{\lambda[\mu m]}[cm^{-1}]$	\\
	    E(photon)	    & $\displaystyle \frac{1.242}{\lambda[\mu m]}[eV]$	\\
	    \hline
	\end{tabular}
    \end{center}
\end{table}



%%%%%%%%%%%%%%%%%%%%%%%%%%%%%%%%%%%%%%%%%%%%%%%%%%%%%%%%%%%%%%%%%%%%%%%%
\section{Scale}
\begin{table}[h]
    \centering
    \caption{atom and lattice}
    \label{tab:atom}
    \begin{tabular}{l | c}
	\hline
	description & value \\
	\hline
	Electron frequency in H.O. model    & $\omega_0 = \sqrt{\frac{k}{m}} \sim 10^{16}$ rad/s \\
	\hline
	Atomic excited state lifetime  & 2 ns\\
	\hline
	lattice constant & a $\sim\si{\angstrom}$, $\frac{e^2}{a} \sim eV, v \sim \sqrt{\frac{e^2}{am}} \sim 10^{8}$ $cm/s$  \\ 
	\hline
	Debye temp. & $\sim 300 K$  \\
	\hline
	plasmon	& $\hbar \omega_p = \hbar \left(\frac{4\pi ne^2}{m}\right)^{\frac{1}{2}} \sim 10 eV$	\\
	\hline
	minimum band gap    & C:$5.5 eV$, Si:$1.1eV$, Ge:$0.76eV$
    \end{tabular}
\end{table}

\subsection{optics}
\begin{table}[h]
    \centering
    \caption{optics}
    \label{tab:optics}
    \begin{tabular}{l | c}
	\hline
	description & value \\
	\hline
	f of visible light  & $\omega \sim 10^{15}$ Hz   \\
	\hline
    \end{tabular}
\end{table}

\subsection{Resistivity}
\begin{table}[h]
    \centering
    \caption{Classification of solids using resistivity}
    \label{tab:resistivity}
    \begin{tabular}{l  l  l}
	\hline
	class	& typical R ($\Omega \cdot cm$)	& example   \\
	\hline
	metal	    & $10^{-6}$ & copper	\\
	semimetal   & $10^{-3}$	& bismuth	\\
	semiconductor	& $10^{-2}-10^{9}$ &	silicon	\\
	insulator   & $10^{14}-10^{22}$	    & diamond	\\
	\hline
    \end{tabular}
\end{table}


\subsection{Superconductivity}
\begin{table}[h]
    \centering
    \caption{Superconductivity}
    \label{tab:superconductivity}
    \begin{tabular}{l | l}
	\hline
	description &	value	\\
	\hline
	transition energy   & $10^{-5}-10^{-2} eV$  \\
	\hline
    \end{tabular}
\end{table}
%%%%%%%%%%%%%%%%%%%%%%%%%%%%%%%%%%%%%%%%%%%%%%%%%%%%%%%%%%%%%%%%%%%%%%%%
\section{Collections}

\newpage
\begin{table}
    \caption{Wavelengths of Common Lasers}
    \label{tab:common_lasers}
    \begin{center}
	\begin{tabular}{l|c}
	    \hline
	    Source	& nm	\\
	    \hline
	    ArF	& 93	\\
	    \hline
	    KrF	& 48	\\
	    \hline
	    Nd:YAG(4)	& 66	\\
	    \hline
	    XeCl	& 08	\\
	    \hline
	    HeCd	& 25, 441.6	\\
	    \hline
	    N2	& 37.1, 427	\\
	    \hline
	    XeF	& 51	\\
	    \hline
	    Nd:YAG(3)	& 54.7	\\
	    \hline
	    Ar	& 88, 514.5, 351.1, 363.8	\\
	    \hline
	    Cu	& 10.6, 578.2	\\
	    \hline
	    Nd:YAG(2)	& 32	\\
	    \hline
	    HeNe	& 32.8, 543.5, 594.1, 611.9, 1153, 1523	\\
	    \hline
	    Kr	& 47.1, 676.4	\\
	    \hline
	    Ruby	& 94.3	\\
	    \hline
	    Nd:Glass	& 060	\\
	    \hline
	    Nd:YAG	& 064, 1319	\\
	    \hline
	    Ho:YAG	& 100	\\
	    \hline
	    Er:YAG	& 940	\\
	    \hline
	\end{tabular}
    \end{center}
\end{table}

\renewcommand{\arraystretch}{1}

% \subsection{}
% \begin{table}[h]
%     \centering
%     \caption{}
%     \label{tab:}
%     \begin{tabular}{}
% 	\hline
% 	\hline
%     \end{tabular}
% \end{table}
% constants and scale
\chapter{Math}

%%%%%%%%%%%%%%%%%%%%%%%%%%%%%%%%%%%%%%%%%%%%%%%%%%%%%%%%%%%%%%%%%%%%%%%%
\section{Useful Equations}
\begin{itemize}
    \item Associated Laguerre polynomials:
	$$ L_n{}^m(x) = (-1)^m\frac{d^m}{dx^m}L_{n+m}(x) \qquad \text{for } x \ge 0$$
    \item Assocated Legendre polynomials:
	$$ P_l{}^m(x) = (1-x^2)^{m/2}\frac{d^m}{dx^m}P_{l}(x) \qquad \text{for } x \ge 0$$
    \item Beta function
	$$ B(x,y) = \frac{\Gamma(x)\Gamma(y)}{\Gamma(x+y)} $$
    \item Complete elliptic integral of the first kind:
	$$ K(k) = F(k,\frac{\pi}{2}) = \int_0^{\frac{\pi}{2}}\frac{d\theta}{\sqrt{1-k^2\sin^2\theta}}$$
    \item Complete elliptic integral of the second kind:
	$$ K(k,\frac{\pi}{2}) = \int_0^{\frac{\pi}{2}}\sqrt{1-k^2\sin^2\theta}\ d\theta$$
    \item Complete elliptic integral of the third kind:
	$$ \Pi(\nu,k,\frac{\pi}{2}) = \int_0^{\frac{\pi}{2}}\frac{d\theta}{(1-\nu\sin^2\theta)\sqrt{1-k^2\sin^2\theta}}$$
    \item Confluent hypergeometric function:
	$$ F(a,c,x)= \frac{\Gamma(c)}{\Gamma(a)}\sum_{n=0}^\infty\frac{\Gamma(a+n)x^n}{\Gamma(c+n)n!}$$
    \item Regular modified cylindrical Bessel function:
	$$ I_\nu(x) = i^{-\nu} J_\nu(ix) = \sum_{k=0}^{\infty}\frac{(x/2)^{\nu+2k}}{k!\Gamma(\nu+k+1)} \qquad \text{for } x \ge 0$$
    \item Cylindrical Bessel function of the first kind:
	$$ J_\nu(x) = \sum_{k=0}^{\infty}\frac{(-1)^k(x/2)^{\nu+2k}}{k!\Gamma(\nu+k+1)} \qquad \text{for } x \ge 0$$
    \item Irregular modified cylindrical Bessel function: 
	$$ K_\nu(x) = \frac{\pi}{2}i^{\nu+1}(J_\nu(ix) + iN_\nu(ix)) = 
	    \begin{cases}
		\frac{I_{-\nu}(x) - I_\nu(x)}{\sin\nu\pi}   \qquad \text{for } x\ge0 \text{ and } \nu \notin Z \\
		\displaystyle \frac{\pi}{2}\lim_{\mu\rightarrow\nu}\frac{I_{-\nu}(x) - I_\nu(x)}{\sin\mu\pi}	\qquad \text{for } x<0 \text{ and } \nu \in Z \\
	    \end{cases}
	    $$
    \item Cylindrical Neumann function of the first kind
	$$ N_\nu(x) =
	    \begin{cases}
		\frac{J_{\nu}(x)\cos\nu x - J_\mu(x)}{\sin\nu\pi} \qquad \text{for } x\ge0 \text{ and } \nu \notin Z \\
		\displaystyle \lim_{\mu\rightarrow\nu}\frac{J_{\mu}(x)\cos\mu\pi - J_{-\nu}(x)}{\sin\mu\pi} \qquad \text{for } x<0 \text{ and } \nu \in Z \\
	    \end{cases}
	    $$
	\item Incomplete elliptic integral of the first kind:
	    $$ F(k,\phi) = \int_0^{\phi}\frac{d\theta}{\sqrt{1-k^2\sin^2\theta}}	\qquad \text{for } |k| \le 1$$
	\item Incomplete elliptic integral of the second kind:
	    $$ E(k,\phi) = \int_0^{\phi}\sqrt{1-k^2\sin^2\theta}\ d\theta	\qquad \text{for } |k| \le 1$$
	\item Incomplete elliptic integral of the thrid kind:
	    $$ \Pi(k,\nu,\phi) = \int_0^{\phi}\frac{d\theta}{(1-\nu\sin^2\theta)\sqrt{1-k^2\sin^2\theta}}	\qquad \text{for } |k| \le 1$$
	\item Exponential integral:
	    $$ Ei(x) = -\int_{-x}^\infty \frac{e^{-t}}{t}dt $$
	\item Hermite polynomials:
	    $$ H_n(x) = (-1)^n e^{x^2} \frac{d^n}{dx^n}e^{-x^2} $$
	\item Hygergeometric series:
	    $$ F(a,b,c,x) = \frac{\Gamma(c)}{\Gamma(a)\Gamma(b)}\sum_{n=0}^{\infty}\frac{\Gamma(a+n)\Gamma{b+n}}{\Gamma(c+n)}\frac{x^n}{n!}$$
	\item Laguerre polynomials:
	    $$ L_n(x) = \frac{e^x}{n!} \frac{d^n}{dx^n}\left(x^n e^{-x} \right)	\qquad \text{for } x\ge 0$$
	\item Legendre polynomials:
	    $$ P_l(x) = \frac{1}{2^l l!} \frac{d^l}{dx^l}(x^2 - 1)^l	\qquad \text{for } |x| \le 1$$
	\item Riemann zeta function:
	    $$ Z(x) = 
	    \begin{cases}
		\sum_{k=1}^{\infty} k^{-x}  \qquad \text{for } x > 1\\
		2^x \pi^{x-1}\sin\left( \frac{x\pi}{2} \right)\Gamma(1-x)\zeta(1-x) \qquad \text{for } x < 1
	    \end{cases}
	    $$
	\item Spherical Bessel functions of the first kind:
	    $$ j_n(x) = \sqrt{\frac{\pi}{2x}} J_{n+1/2}(x)  \qquad \text{for } x \ge 0$$
	\item Spherical assocated Legendre function:
	    $$ Y_l^m(\theta, \phi) = (-1)^m \left[ \frac{2l+1}{4\pi} \frac{(l-m)!}{(l+m)!}\right]^{\frac{1}{2}} P_l^m(\cos\theta)e^{im\phi} \qquad \text{for } |m| \le l$$
	\item Spherical Neumann functions, Spherical Bessel functions of the second kind:
	    $$ n_n(x) = \left(\frac{\pi}{2x} \right)^{\frac{1}{2}} N_{n+\frac{1}{2}}(x)	\qquad \text{for } x \ge 0 $$
\end{itemize}
%%%%%%%%%%%%%%%%%%%%%%%%%%%%%%%%%%%%%%%%%%%%%%%%%%%%%%%%%%%%%%%%%%%%%%%%
Number of independent solution to a derivative equation ???
\section{Definition}

%%%%%%%%%%%%%%%%%%%%%%%%
\subsubsection{Levi-Civita}
$\epsilon_{ijk}$ is an antisymmetric tensor, which is defined as: 
\begin{equation}
    \label{eqn:math::Levi-Civita}
    \epsilon_{123} = 1, \epsilon_{ijj} = 0, \epsilon_{ijk} = -\epsilon_{jik}
\end{equation}

\subsubsection{Fourier Transform}
\begin{equation}
    \label{eqn:math::FT}
    \begin{gathered}
    \displaystyle \tilde{f}(k) = \int_{-\infty}^{\infty}dx e^{ikx}f(x)	\\
    \displaystyle f(x) = \int_{-\infty}^{\infty}\frac{dk}{2\pi}e^{-ikx}\tilde{f}(k)
    \end{gathered}
\end{equation}

If $f(x) = f(x+R)$, then 
\begin{equation}
    \tilde{f}(k) = \int dx e^{ikx}f(x) = e^{-ikR} \int d(x+R) e^{ik(x+R)}f(x+R) = e^{-ikR} \tilde{f}(k)
\end{equation}
So, $\tilde{f}(k)$ is non-zero only at point with $kR = 2n\pi$, or $k = \frac{2\pi}{R}n$.
\begin{equation}
    \displaystyle f(x) = \sum_{k} e^{-ikx}\tilde{f}(k)
\end{equation}

\section{Vector Operation}
\begin{equation}
    \nabla\times(\nabla\times\vec{A}) = \nabla(\nabla\vec{A}) - \nabla^2\vec{A} 
\end{equation}

Two point Fourier transformation:
\begin{equation}
    \displaystyle f(\bm{r,r'}) = \frac{1}{\Omega}\sum_{\bm{qq'}}e^{i\bm{q\cdot r}}f(\bm{q,q'})e^{-i\bm{q'\cdot r'}}
\end{equation}

%%%%%%%%%%%%%%%%%%%%%%%%%%%%%%%%%%%%%%%%%%%%%%%%%%%%%%%%%%%%%%%%%%%%%%%%
\section{Formulas}

%%%%%%%%%%%%%%%%%%%%%%%%%%%%%%%%%%%%%%%%%%%%%%%%
\subsection{Fourier transform}
\[
    \delta^3(\vec{x})=\int\frac{d^3k}{(2\pi)^3}e^{i\vec{k}\vec{x}}
    \]
%%%%%%%%%%%%%%%%%%%%%%%%%%%%%%%%%%%%%%%%%%%%%%%%
\subsection{Limitation}
For 
\[
    f(\omega) = \frac{\sin(N\omega)T}{\sin(\omega T)}   \quad (\omega > 0)
    \]
when $N\rightarrow\infty$, 
we can find that the dominant value lies in 
\[
    \omega = \frac{m\pi}{T} \qquad  m=1,2,3 \cdots
    \]
with $\omega=\frac{m\pi}{T}+\Delta\omega$, we can get:
\begin{equation}
    \displaystyle \lim_{N\rightarrow\infty}f(\omega)\approx\frac{(-1)^m\sin(N\Delta\omega T)}{(-1)^m\Delta\omega T}
\end{equation}
Note: $N\Delta\omega T$ is not infinity small, so we need to keep $\sin()$. \\
We also find that:
\begin{equation}
    \int_{-\epsilon}^{\epsilon}d(\Delta\omega)\frac{\sin(N\Delta\omega T)}{\Delta\omega T}
    =\frac{1}{T}\int_{-N\epsilon T}^{N\epsilon T} dx\frac{x}{x} = \frac{\pi}{T}
\end{equation}
So: 
\begin{equation}
    \displaystyle \lim_{N\rightarrow\infty}f(\omega)\approx\frac{(-1)^m\sin(N\Delta\omega T)}{(-1)^m\Delta\omega T}
    =\displaystyle \sum_{m=1}\frac{\pi}{T}\delta({\omega-\omega_m})
\end{equation}

Similarly:
\begin{equation}
    \displaystyle \lim_{N\rightarrow\infty}f_1(\omega)=\left(\frac{\sin(N\omega T)}{\omega T}\right)^2
    \approx\left(\frac{(-1)^m\sin(N\Delta\omega T)}{(-1)^m\Delta\omega T}\right)^2
\end{equation}

\begin{equation}
    \int_{-\epsilon}^{\epsilon}d(\Delta\omega)\left(\frac{\sin(N\Delta\omega T)}{\Delta\omega T}\right)^2
    =\frac{N}{T}\int_{-N\epsilon T}^{N\epsilon T} dx\frac{x}{x}
    =\frac{N\pi}{T}
\end{equation}
\begin{equation}
    f_1(\omega)=\displaystyle \sum_{m=1}\frac{N\pi}{T}\delta({\omega-\omega_m})
\end{equation}


\begin{equation}
    \displaystyle lim_{\alpha\rightarrow 0^{+}}\frac{1}{x \pm i\alpha} = P\left(\frac{1}{x}\right) \mp i\pi\delta(x)
\end{equation}
%%%%%%%%%%%%%%%%%%%%%%%%%%%%%%%%%%%%%%%%%%%%%%%%
\subsection{Integral}
\[
    I=\int_{-\infty}^{\infty}dk\frac{e^{ikr}-e^{-ikr}}{k}
    \]
Note that the integrand does not blow up as $k \rightarrow 0$, so
\[
    I=\displaystyle\lim_{\delta\rightarrow{0}}\left[\int_{-\infty}^{\infty}dk\frac{e^{ikr}-e^{-ikr}}{k+i\delta}\right]
    \]
So we have pole at $k = -i\delta$. For $e^{ikr}$ we must close the contour
up to get exponential decay at large $\mathit{k}$. This misses the pole, so
this term gives zero. So
\[
    I=\int_{-\infty}^{\infty}dk\frac{-e^{-ikr}}{k+i\delta}=-(2\pi{i})(-e^{-\delta{r}})=2\pi{i}e^{-i\delta{r}}
    \]

\subsubsection{Useful Integrals}
\[
    \int_0^\infty e^{-\alpha x}x^ndx = \frac{n!}{\alpha^{n+1}} 
    \]
%%%%%%%%%%%%%%%%%%%%%%%%%%%%%%%%%%%%%%%%%%%%%%%%%%%%%%%%%%%%%%%%%%%%%%%%
\section{Functions}

%%%%%%%%%%%%%%%%%%%%%%%%%%%%%%%%%%%%%%%%%%%%%%%%
\subsection{Interesting functions}
%%%%%%%%%%%%%%%%%%%%%%%%
\subsubsection{Weierstrass function}
\begin{equation}
    f(x) = \sum_{n=0}^{\infty}a^{n}cos(b^{n}\pi x)
\end{equation}
Continuous but not derivative everywhree.


%%%%%%%%%%%%%%%%%%%%%%%%%%%%%%%%%%%%%%%%%%%%%%%%
\subsection{Frequently used functions}

%%%%%%%%%%%%%%%%%%%%%%%%
\subsubsection{Legendre Polynomials and spherical harmonics}
\begin{equation}
    Y^m_l(\theta,\phi)=Ne^{im\phi}P^m_l(\cos\theta)
\end{equation}
which satisfies
\[
    r^2\nabla^2Y^m_l(\theta,\phi)=-l(l+1)Y^m_l(\theta,\phi)
    \]

Standard convention :
\[
    P^{-m}_l(\cos\theta)=(-1)^m\frac{(l-m)!}{(l+m)!}P^m_l(\cos\theta),
\]
In QM:
\[
    Y^m_l(\theta,\phi)=(-1)^msqrt{\frac{2l+1}{4\pi}\frac{(l-m)!}{(l+m)!}}P_{lm}(\cos\theta)e^{im\phi}
\]
Where $P_{lm}$ are associated Legendre polynomials without the
Condon-Shortley phase.

Though of different definition, all of them satisfy
\[ 
    Y_l^{m*}(\theta,\phi)=(-1)^{m}Y_{l}^{-m}(\theta,\phi) 
    \]


Associate Legendre Polynomials:

%%%%%%%%%%%%%%%%%%%%%%%%
\subsection{Bessel Function}
%%%%%%%%%%%%%%%%%%%%%%%%%%%%%%%%%%%%%%%%%%%%%%%%
\subsection{Differential Equations}

%%%%%%%%%%%%%%%%%%%%%%%%
\subsubsection{First Order Equation}
\[ 
\frac{dV}{dt} + \eta(t)V(t) = f(t)  
\]
so:
\[
    \frac{1}{I}\frac{d(I(t)V(t))}{dt} = \frac{dV(t)}{dt} + \frac{I'(t)}{I(t)}V(t)
    \]
if: 
\[ 
    \frac{I'(t)}{I(t)} = \eta(t)	\rightarrow I(t) = exp(\int^{t}dt'\eta(t'))
    \]
so:
\[ \frac{d(I(t)V(t))}{I dt} = f(t)\]

\subsubsection{inhomogeneous 2nd order differential equation}
\begin{equation}
    \frac{d^{2}x}{dt^2} + a(t)\frac{dx}{dt} + b(t)x = f(t)
\end{equation}
Its solution is given by
\begin{equation}
    \displaystyle x(t) = c_{1}\phi_{1}(t) + c_{2}\phi_{2}(t) + \int_{t_0}^{t}\frac{\phi_{1}(\xi)\phi_{2}(t) - \phi_{2}(\xi)\phi_{1}(t)}{W(\xi)} f(\xi)d\xi
\end{equation}
Where $\phi_{1}$ and $\phi_{2}$ are the 2 orthogonal solutions to the homogeneous 2nd differential equation, and $W(t)$ is the Wronskian.
\begin{equation}
    W(t) = 
	\begin{vmatrix}
	    \phi_1(t)	    & \phi_2(t) \\
	    \phi_1^{'}(t)   & \phi_2^{'}(t)	\\
	\end{vmatrix}
\end{equation}


\subsubsection{Pendulum Equation}
\begin{equation}
    \begin{gathered}
	\frac{d^2}{dz^2}\psi + ucos\psi = 0 \\
	\psi(z) = \psi(0) + \psi'(0)z - u\int_0^z dz_1 \int_0^{z_1}cos\psi(z_2)dz_2
    \end{gathered}
\end{equation}
%%%%%%%%%%%%%%%%%%%%%%%%
\subsubsection{Floquet theory}
\begin{equation}
    \ddot{x} + \omega_{0}^{2}(1+\mu cos(\nu t))x = 0
\end{equation}
Analogy to QM(Schrodinger Equation):	
\[
    -\frac{\hbar^{2}}{2m}{\psi''} + U(x)\psi = E\psi
    \]
where $U(x) = -U_{0}cos\frac{2\pi x}{a}$.
So let 
\[
    k^{2} = \frac{2mE}{\hbar^{2}}, u=\frac{U_0}{E}, \nu=\frac{2\pi}{a}
    \]
$u$ is small, we will get   
\[ 
    \psi''+k^{2}\psi = -k^{2}ucos(\nu x)\psi
    \]

%%%%%%%%%%%%%%%%%%%%%%%%%%%%%%%%%%%%%%%%%%%%%%%%
\subsection{Integrated functions}
%%%%%%%%%%%%%%%%%%%%%%%%
\subsubsection{$\Gamma$ function}
\[\Gamma(z) \equiv \int_{0}^{+\infty}dt e^{-t} t^{z-1} =
2\int_{0}^{+\infty}dx e^{-x^2}x^{2z-1}\]

\[\Gamma(z)\Gamma(1-z) = \frac{\pi}{sin\pi z}\]

%%%%%%%%%%%%%%%%%%%%%%%%%%%%%%%%%%%%%%%%%%%%%%%%%%%%%%%%%%%%%%%%%%%%%%%%
\section{Group}

\subsection{Representation of Group}
A \textbf{representation} of a group G on a vector space V over a field K is
a group homomorphism from G to GL(V), the 

%%%%%%%%%%%%%%%%%%%%%%%%%%%%%%%%%%%%%%%%%%%%%%%%%%%%%%%%%%%%%%%%%%%%%%%%
\section{Probability Distribution Functions}

%%%%%%%%%%%%%%%%%%%%%%%%%%%%%%%%%%%%%%%%%%%%%%%%
\subsection{Uniform}

%%%%%%%%%%%%%%%%%%%%%%%%%%%%%%%%%%%%%%%%%%%%%%%%
\subsection{Exponential}

%%%%%%%%%%%%%%%%%%%%%%%%%%%%%%%%%%%%%%%%%%%%%%%%
\subsection{Gauss}

%%%%%%%%%%%%%%%%%%%%%%%%%%%%%%%%%%%%%%%%%%%%%%%%
\subsection{BreitWigner}

%%%%%%%%%%%%%%%%%%%%%%%%%%%%%%%%%%%%%%%%%%%%%%%%
\subsection{Poisson}

%%%%%%%%%%%%%%%%%%%%%%%%%%%%%%%%%%%%%%%%%%%%%%%%%%%%%%%%%%%%%%%%%%%%%%%%
\section{Direvative Functions}

%%%%%%%%%%%%%%%%%%%%%%%%
\subsubsection{RC Oscillation}
\begin{equation}
    \dot{Q} + \frac{Q}{RC} = I(t) = I_0e^{-t/\tau}
\end{equation}
to get:
\begin{equation}
    Q(t) = A (e^{-t/\tau} - e^{-t/\tau_0})
\end{equation}
where: $$ \tau_0 = RC $$ 

If $$ \tau = \tau_0 $$, then the solution is different:
\begin{equation}
    Q(t) = I_0 te^{-t/\tau_0} + Ae^{-t/\tau_0}
\end{equation}
% tools
\chapter{Model}
%%%%%%%%%%%%%%%%%%%%%%%%%%%%%%%%%%%%%%%%%%%%%%%%%%%%%%%%%%%%%%%%%%%%%%%%
\section{Model}

Boltzmann distribution

\subsection{Oscillation}

\subsection{Binary system}

\subsection{Scattering}

\subsection{Drude model}
\begin{equation}
    m\ddot{\bm{x}} = -e\bm{E} - \frac{m\dot{\bm{x}}}{\tau}
\end{equation}
Where $\tau$ is the relaxation time, Solve this equation, we will get
\[
    0 = -e\bm{E} - \frac{m\dot{\bm{x}}}{\tau} \Longrightarrow
    \dot{\bm{x}} = \frac{-e\bm{E}\tau}{m}
\]
This means that if electron reach velocity $\dot{x} = -eE\tau/m$, it will move at this constant velocity.

Note the difference between the Drude model and the "accelerating model", which will accelerates electron periodically, with period being the relaxation time $\tau_1$, so the average velocity will be:
\[
    \bar{v} = \frac{1}{2}\frac{-eE\tau_1}{m}
\]
So 
\[
    \dot{x} = \bar{v}  \Longrightarrow \tau = \frac{1}{2}\tau_1
\]


%%%%%%%%%%%%%%%%%%%%%%%%%%%%%%%%%%%%%%%%%%%%%%%%%%%%%%%%%%%%%%%%%%%%%%%%
\section{Approximation}
The ultimate weapon of simplifying a question (constructing a model)
is approximation. Here are some frequently used approximation.

%%%%%%%%%%%%%%%%%%%%%%%%%%%%%%%%%%%%%%%%%%%%%%%%
\subsection{RWA: Rotating Wave Approximation}
\[
    \cos(\omega t)e^{-i\omega_0 t} = \frac{1}{2}(e^{i\omega t} + e^{-i\omega t})e^{-i\omega t} = \frac{1}{2}(e^{i(\omega - \omega_0)t} + e^{-i(\omega + \omega_0)t}) \approx \frac{1}{2}e^{i(\omega - \omega_0)t}
\]
When $\omega \sim \omega_0$, we simply discard the second term because
$\omega + \omega_0 >> \delta = \omega - \omega_0$. We can think about it
as that the second term oscillates so fast that nothing can response to it
(response is small), thus the first term is more effective than the second
term.


%%%%%%%%%%%%%%%%%%%%%%%%%%%%%%%%%%%%%%%%%%%%%%%%
\subsection{Electric Dipole Approximation}
For the interaction of light with atom, the wavefunction vanish 
exponentially on the distance scale of the Bohr radius $a_0$.
\[
    |\vec{k} \cdot \vec{r}| \approx 2\pi a_0/\lambda << 1 
\]
\[
    e^{i\vec{k} \cdot \vec{r}} = 1 + i\vec{k}\cdot\vec{r} + \mathcal{O}(\vec{k}\cdot\vec{r})^2
\]

Which can be easily seen from the point that an atom is much smaller than 
the wavelength of light (visible light), so we can think that the 
electric field applied on the atom is constant at specific time.

\begin{tikzpicture}
    \draw plot[domain=0:2*pi, smooth] (\x, {sin(\x r)});
    \draw (1, {sin(1 r)}) circle (2.5pt);
    \node (atom) at (1, {sin(1 r) - 0.4}){atom};
\end{tikzpicture}



%%%%%%%%%%%%%%%%%%%%%%%%%%%%%%%%%%%%%%%%%%%%%%%%
\subsection{Slowly Varying Envelope Approximation}
The fractional change of the beam's transverse profile over a wavelength
distance to be small compared with unity, and similarly for its 
derivative, which si:
\[
    \lambda|\frac{\partial E_0}{\partial z}| << |E_0| 
    \text{and} 
    \lambda|\frac{\partial^2 E_0}{\partial z^2}| << |\frac{\partial E_0}{\partial z}| 
\]



%%%%%%%%%%%%%%%%%%%%%%%%%%%%%%%%%%%%%%%%%%%%%%%%
\subsection{Harmonic Approximation}
\[
    U = U_0 + \sum_{n, \alpha} \frac{\partial U}{\partial X_n^\alpha} X_n^\alpha + \frac{1}{2}\sum_{n \alpha; n' \alpha'}\frac{\partial^2U}{\partial X_n^\alpha \partial X_{n'}^{\alpha'}}X_n^\alpha X_{n'}^{\alpha'} + cdots
\]

Truncate the series after second order.



%%%%%%%%%%%%%%%%%%%%%%%%%%%%%%%%%%%%%%%%%%%%%%%%%%%%%%%%%%%%%%%%%%%%%%%%
\section{Distribution}



%%%%%%%%%%%%%%%%%%%%%%%%%%%%%%%%%%%%%%%%%%%%%%%%
\subsection{Boltzmann Distribution}


% part 2: general definition and conclusion without detailed proof
\part{Definition and Conclusion}
\chapter{Definition}
Definition and Conclusion.

%%%%%%%%%%%%%%%%%%%%%%%%%%%%%%%%%%%%%%%%%%%%%%%%%%%%%%%%%%%%%%%%%%%%%%%%
\section{General}

%%%%%%%%%%%%%%%%%%%%%%%%%%%%%%%%%%%%%%%%%%%%%%%%
\subsection{index}
Entries $(\mathcal{L})_{\mn}$ of a matrix are labelled by rows ($\mu$) 
and columns ($\nu$).
You are \textbf{free} to move this row and column index up and down at 
will -- what matters is the order of index:
\[
    (\mathcal{L})_{\mn} = (\mathcal{L}^T)_{\mn} = {(\mathcal{L})^\mu}_\nu = {(\mathcal{L}^T)_\nu}^\mu = {L^\mu}_\nu
    \]
The much \textbf{preferred} placemnet of the indices surrounding the 
matrix is just a visual reminder of the individual entries ${L^\mu}_\nu$ 
which together form the matrix ($\mathcal{L}$) and ($\mathcal{L}^T$), 
and that's all.

The indices labelling ${L^\mu}_\nu$ can not be raised or lowered randomly,
but are raised and lowered with the metric tensor:
\[
    (g\mathcal{L})_{\mn} = g_{\mu\rho}{L^\rho}_{\nu} \equiv L_{\mn} 
    \]
and 
\[
    {(g\mathcal{L}g)_\mu}^\nu = g_{\mu\rho}{L^\rho}_\sigma g^{\sigma\nu} \equiv {L_\mu}^\nu
    \]

%%%%%%%%%%%%%%%%%%%%%%%%
\subsubsection{Vector Index}
\begin{equation}
    \begin{gathered}
	x^i = (x, y, z); \quad 
	x_i = 
	\begin{pmatrix}
	    x   \\
	    y   \\
	    z   \\
	\end{pmatrix}   \\
	x^\mu = (t, x, y, z); \quad 
	x_\mu = 
	\begin{pmatrix}
	    t	\\
	    x   \\
	    y   \\
	    z   \\
	\end{pmatrix}   \\
    \end{gathered}
\end{equation}



\subsection{Cross Section}
\begin{equation}
    \label{eqn:xsection}
    \sigma = \frac{\text{scattered power}}{\text{incident intensity}}
\end{equation}
quasi-periodic:
\begin{equation}
    \label{quasi-periodic}
    \phi(x+a) = e^{ika}\phi(x)
\end{equation}
There is a phase shift, so it is not rigorously periodic.


Rotation:
The algebraic characterization $R^TR=\mathcal{1}$ is a much more useful
definition of the group than the explicit form of the rotation matrices 
as a function of $\theta$.  \\
Another way to define the rotation group is as the set of linear 
transformations on $\mathcal{R}^n$ preserving the inner product 
$x^ix_i=\delta_{ij}x^ix^j$:
\begin{equation}
    \label{eqn:rotation}
    R_{ki}R_{lj}\delta_{kl} = (R^T)_{ik}\delta_{kl}R_{lj} = [(R^T)\mathcal{1}(R)]_{ij} = (R^TR)_{ij}=\delta_{ij}
\end{equation}


For an oscillating field $E(t) = E_0 e^{i(\vec{k}\cdot\vec{x}-\omega t}$, the time average is:
\begin{equation}
    <E> = \frac{1}{T}\int_0^TE(t)dt = \frac{1}{2}E_0e^{i\vec{k}\cdot\vec{x}}
\end{equation}

A plain wave $E = E_0 e^Pi(\vec{k}\cdot\vec{x}-\omega t)$ (such as the 
Fourier mode of a $\delta$ function) has \textbf{no} starting or ending 
point and it fulfills all the space. So understanding the scatter of a 
plain wave is some tricky thing.

Velocity: 
\begin{equation}
    \vec{v} = \nabla_{\vec{p}}E(|\vec{p}|)
\end{equation}

\textbf{Lagrangian (Hamiltonian)} is the Legendre transform of the Hamiltonian (Lagrangian)
\begin{equation}
    L(q, \dot{q}) = p(q, \dot{q})\dot{q} - H(q,p(q, \dot{q})), \quad 
    H(q, p) = p\dot{q}(q,p) - L(q,\dot{q}(q,p))
\end{equation}
where $p(q, \dot{q})$ is defined as $\frac{\partial H}{\partial p} = \dot{q}$
and $\dot{q}(q,p)$ is defined as $\frac{\partial L(q,\dot{q})}{\partial \dot{q}}=p$

%%%%%%%%%%%%%%%%%%%%%%%%%%%%%%%%%%%%%%%%%%%%%%%%
\subsection{Energy}
\begin{description}
    \item [Power]
	\begin{equation}
	    \frac{dP}{d\Omega} = \frac{dW}{dt\cdot d\Omega} = r^2\frac{dW}{dt dA}
	\end{equation}
\end{description}

%%%%%%%%%%%%%%%%%%%%%%%%%%%%%%%%%%%%%%%%%%%%%%%%
\subsection{Potential}
\begin{description}
    \item [Yukawa potential (screened Coulomb potential)]
	\begin{equation}
	    \label{eqn:Yukawa}
	    V = V_0\frac{e^{-\mu r}}{r}
	\end{equation}
\end{description}

%%%%%%%%%%%%%%%%%%%%%%%%%%%%%%%%%%%%%%%%%%%%%%%%
\subsection{Monte Carlo}
\begin{description}
    \item [Monte Carlo Method]
	A method to search for solutions to mathematical problem using a statistical sampling with random numbers.
\end{description}

%%%%%%%%%%%%%%%%%%%%%%%%%%%%%%%%%%%%%%%%%%%%%%%%
\subsection{Dirac Delta}
\begin{description}
    \item [Dimension] 
	Note that $\delta^3(x)$ and $\delta(x)$ have different dimensions.
	\begin{equation}
	    [\delta^3(x)] = L^{-3}, \qquad [\delta(x)] = L^{-1}
	\end{equation}
\end{description}

\subsection{Acronyms}
h.c.	\\
    Hermit conjugate	\\


%%%%%%%%%%%%%%%%%%%%%%%%%%%%%%%%%%%%%%%%%%%%%%%%%%%%%%%%%%%%%%%%%%%%%%%%
\section{CM}


%%%%%%%%%%%%%%%%%%%%%%%%%%%%%%%%%%%%%%%%%%%%%%%%%%%%%%%%%%%%%%%%%%%%%%%%
\section{QM}
Heisenberg EOM:
\begin{equation}
    \label{eqn:Heisenberg EOM}
    i\partial_t\phi(x) = [\phi, H]
\end{equation}

S-matrix
\begin{equation}
    \label{S-matrix}
    \bra{f}S\ket{i}_{heisernberg} = \bra{f;\infty}\ket{i;-\infty}_{Schrodinger}
\end{equation}

\subsubsection{Harmonic basis}
$\hat{\epsilon}_{-1} = \frac{\hat{x} - i\hat{y}}{\sqrt{2}}$
$\hat{\epsilon}_{0} = \hat{z}$
$\hat{\epsilon}_{1} = -\frac{\hat{x} + i\hat{y}}{\sqrt{2}}$

$\hat{\epsilon}_0$ refers to light linearly polarized in the z-direction
($\pi-$light). The polarizations $\hat{\epsilon_{\pm 1}}$ refer to 
circularly polarized light with $-1$ for left-handed ($\sigma^-$-light) 
and $+1$ for right-handed ($\sigma^+$-light)

\[
    \hat{\epsilon}_q \cdot \vec{r} = \sqrt{\frac{4\pi}{3}}rY_{1q}(\theta, \phi)
\]
%%%%%%%%%%%%%%%%%%%%%%%%%%%%%%%%%%%%%%%%%%%%%%%%
\subsection{Symmetry}

%%%%%%%%%%%%%%%%%%%%%%%%%%%%%%%%%%%%%%%%%%%%%%%%
\subsection{Perturbation Theory}

%%%%%%%%%%%%%%%%%%%%%%%%%%%%%%%%%%%%%%%%%%%%%%%%
\subsection{Scattering}
\begin{description}
    \item [Lippmann-Schwinger Equation]
	\begin{equation}
	    \label{eqn:qm:LSEqn}
	    \left\{
		\begin{aligned}
		    i\hbar\partial_t\psi^\dag(t,\vec{x}) &= H\psi^\dag(t, \vec{x})   \\
		    \psi^\dag(t=-\inf, x) &= \phi(t,x) = \frac{1}{(2\pi)^3/2}e^{-\frac{i}{\hbar}Et + i\vec{k}\cdot\vec{x}} 
		\end{aligned}
		\right.
	\end{equation}
\end{description}

\section{Relativity}

Postulates:
\begin{itemize}
    \item All inertial observers have the same equations of motion and the same physical laws.
    \item The speed of light is constant for all inertial frames.
\end{itemize}
Assuming internial frame K' moves with velocity $\vec{v} = v_0\hat{x}$ with respected to
rest internial frame K.
\subsection{Definition}
\begin{description}
    \item {4-vector}
	\begin{equation}
	    \label{eqn:sr::4vectors}
		x^\mu \equiv (ct, x^1, x^2, x^3) = (ct, \vec{x})
	\end{equation}

	In calculation, it's usually regarded as a column vector:
	\begin{equation}
	    x^\mu = 
	    \begin{pmatrix}
		ct	\\
		x^1	\\
		x^2	\\
		x^3	\\
	    \end{pmatrix}
	\end{equation}

    \item{4-velocity}
	\begin{equation}
	    \label{eqn:sr::4velocity}
	    U^\mu \equiv \frac{dx^\mu}{d\tau} = (\gamma c, \gamma \vec{v})
	\end{equation}

    \item[metric tensor] 
	\begin{equation}
	    \label{eqn:sr::metricTensor}
	    g^{\mn} = 
	    \begin{pmatrix}
		-1  & 0	& 0 & 0	\\
		0   & 1	& 0 & 0 \\
		0   & 0	& 1 & 0 \\
		0   & 0	& 0 & 1 \\
	    \end{pmatrix}
	\end{equation}
    \item [$\beta$, $\gamma$]
	\begin{equation}
	    \label{eqn:sm::betaGamma}
	    \begin{gathered}
		\beta \equiv \frac{v_0}{c} \\
		\gamma	\equiv \frac{1}{\sqrt{1-\beta^2}}	\Rightarrow \gamma \ge 1
	    \end{gathered}
	\end{equation}
	Play of $\beta$ and $\gamma$
	\begin{equation}
	    \begin{gathered}
		\gamma^2 - \beta^2\gamma^2 = \gamma^2(1-\beta^2) = 1	\\
		1-\beta^2 = (1+\beta)(1-\beta) \xRightarrow{\beta\sim 1} 2(1-\beta)  \\
		\beta \sim 0 \Rightarrow \gamma \sim 1  \\
	    \end{gathered}
	\end{equation}

    \item [Proper time]
	\begin{equation}
	    \label{eqn:sr::properTime}
	    d\tau = \frac{1}{\gamma}dt
	\end{equation}
\end{description}
\subsection{Transformation}
\begin{description}
    \item [Spacetime]
	\begin{equation}
	    \label{eqn:sr::spacetime}
	    \begin{pmatrix}
		ct' \\
		x'  \\
		y'  \\
		z'  \\
	    \end{pmatrix}
	    =
	    \begin{pmatrix}
		\gamma	& -\bg	&   0	& 0 \\
		-\bg	& \gamma    & 0	& 0 \\
		0   & 0	& 1 & 0	\\
		0   & 0	& 0 & 1	\\
	    \end{pmatrix}
	    \begin{pmatrix}
		ct \\
		x  \\
		y  \\
		z  \\
	    \end{pmatrix}
	\end{equation}

    \item [Velocity]
	\begin{equation}
	    \label{eqn:sr::velocity}
	    \begin{gathered}
		c = c	\\
		v'^x = \frac{v^x - v_0}{1-v^xv_0/c^2}	\\
		v'^y = \frac{v^y}{\gamma(1-v^xv_0/c^2)}	\\
		v'^z = \frac{v^z}{\gamma(1-v^xv_0/c^2)}	\\
	    \end{gathered}
	\end{equation}
\end{description}

\subsection{Examples}
A object moves in velocity $\vec{v} = v_0\hat{x}$, then in the 
object rest frame K', we measure its length as $l_0$; and the
proper time for an evnet as $d\tau$; then in the lab frame, what 
we meaure are $L$ and $dt$:
\begin{description}
    \item [length contract effect]
	\begin{equation}
	L = \frac{l_0}{\gamma}
	\end{equation}

    \item [time dilation]
	\begin{equation}
	    dt = \gamma d\tau
	\end{equation}
\end{description}

%%%%%%%%%%%%%%%%%%%%%%%%%%%%%%%%%%%%%%%%%%%%%%%%%%%%%%%%%%%%%%%%%%%%%%%%
\section{QFT}

%%%%%%%%%%%%%%%%%%%%%%%%%%%%%%%%%%%%%%%%%%%%%%%%%%%%%%%%%%%%%%%%%%%%%%%%
\section{Nucleus}
\begin{description}
    \item [Radiation length]
    \item [Mean free path]
    \item [$dE/dx$]
\end{description}

\begin{description}
    \item [Decay rate] 
	\begin{equation}
	    \label{eqn:nu::decayRate}
	    \Gamma = 1/\tau
	\end{equation}
\end{description}

%%%%%%%%%%%%%%%%%%%%%%%%%%%%%%%%%%%%%%%%%%%%%%%%%%%%%%%%%%%%%%%%%%%%%%%%
\section{HEP}
Fermi's golden rule: the transition rate between two states is proportional
to the matrix element squared:
\begin{equation}
    \label{Fermi's golden rule}
    \Gamma \approx |\mathcal{M}|^2\delta(E_f-E_i)
\end{equation}
The matrix element
\begin{equation}
    \label{matrix element}
    \mathcal{M} = \bra{f}H_{int}\ket{i}
\end{equation}
%%%%%%%%%%%%%%%%%%%%%%%%%%%%%%%%%%%%%%%%%%%%%%%%%%%%%%%%%%%%%%%%%%%%%%%%
\section{Miscellaneous}
\begin{description}
    %%%%%%%%% A %%%%%%%%%%% 
    \item [adjoint representation]
    \item [analyticity]
    \item [associativity of addition and multiplication]    
	$a+(b+c) = (a+b)+c$   , and $a\cdot(b\cdot c)=(a\cdot b)\cdot c$
    \item [bilinear]	In QFT, a bilinear term means it has exactly two
	fields. Such as:
	\[ \mathcal{L}_K \supset \frac{1}{2}\phi\Box\phi,
	\frac{1}{4}F^2_{\mn}, \frac{1}{2}m^2\phi^2,
	\frac{1}{2}\phi_1\Box\phi_2, \phi_1\partial_\mu A_\mu, \dots \]

    %%%%%%%%% B %%%%%%%%%%% 
    \item [Boltzmann distribution]  $n_{i} = Ne^{-\beta E_{i}}$

    %%%%%%%%% C %%%%%%%%%%% 
    \item [causality]
    \item [charge conjugation C] taking particles to antiparticles
	\[  C:\quad\psi\rightarrow-i\gamma_2\psi^*\equiv\psi_C	\]
	In the Weyl basis, $\gamma_2^*=-\gamma_2$ and $\gamma_2^T=\gamma_2$, so
	\[  C:\quad\psi^*\rightarrow-i\gamma_2\psi  \]
    \item [chirality] The handedness of a spinor is referred to as its
	chirality. The \textbf{left-handed} and \textbf{right-handed} 
	refer to the ($\frac{1}{2},0$) or (0,$\frac{1}{2}$) representations
	of the Lorentz group. This concept only exists for spinors, or more
	precisely for (A,B) representations of the Lorentz group with
	$A\neq B$. Almost always, chirality means that a theory is not
	symmetric between left-handed Weyl spinors $\psi_L$ and right-handed
	spinors $\psi_R$
    \item [cluster decomposition principle]
    \item [commutativity of addition and multiplication] 
	$a+b=b+a$ and $a\cdot b = b\cdot a$
    \item [contravariant vectors]   vectors with upper indices
	$$ p^\mu = (E, \vec{p}) = (E, p_x, p_y, p_z) $$
    \item [covariant vectors]	vectors with lower indices
	$$ p_\mu = g_{\mu\nu}p^\mu = (E, -\vec{p}) = (E, -p_x, -p_y, -p_z) $$

	using convention: $g_{\mu\nu} = g^{\mu\nu} = diag(1, -1, -1, -1)$
    %%%%%%%%% D %%%%%%%%%%% 
    \item [Dirac spinors]
    \item [Dirac Lagrangian] 
	\[
	    \mathcal{L}=\bar{\psi}(i\slashed{\partial}-m)\psi
	\]
    \item [Dirac equation]
	\[  
	    (i\slashed{D} -m)\psi =0	
	\]
    \item [Dirac Mass]
	\[  
	    \mathcal{L}_{\text{Dirac mass}} = m(\psi_L^\dag\psi_R + \psi^\dag_R\psi_L) 
	\]
    \item [Distributivity of multiplication over addition]
	$a\cdot(b+c)=(a\cdot b) + (a\cdot c)$

    %%%%%%%%% E %%%%%%%%%%% 
    \item [equipartion theorem] a body in thermal equilibrium should have
	energy equally distributed among all possible modes, (mode is a
	seperation of phase space), which means all modes have the same
	energy.

    %%%%%%%%% F %%%%%%%%%%% 
    \item [faithful representation] A representation in which each group
	element gets its own matrix is called a \emph{faithful
	representation}.
    \item [Fermi's golden rule]	$\Gamma \sim |\mathcal{M}|^{2}\delta(E_f - E_i)$
    \item [first principle]

    %%%%%%%%% G %%%%%%%%%%% 
    \item [Gauge transform] $\phi \rightarrow e^{-i\alpha}\phi$
    \item [good quantum number] which doesn't change along time. $[Q, H] = 0$
    \item [Grassmann Numbers] For Majorana masses to be non-trivial, fermion 
	compnents cannot be regular numbers, they must be anticommuting numbers. 
	Such things are called Grassmann numbers.

    %%%%%%%%% H %%%%%%%%%%% 
    \item [Helicity] spin projected on the direction of motion is called the
	helicity. $\hat{h}=\frac{\vec{\sigma}\cdot\vec{p}}{|\vec{p}|}$

    %%%%%%%%% L %%%%%%%%%%% 
    \item [Legendre transformation]
    \item [Levi-Civita] $\epsilon_{ijk} = \left\{
	\begin{aligned}
	    1\quad  normal order(123,231,312) \\
	    -1\quad reverse order(132,213,321)	\\
	    0\quad  otherwise	\\
	\end{aligned} \right.$
    \item [Lie algebra] the generators of the Lie group form an algebra
	called its Lie algebra. 
    \item [Lie groups]	Lie groups are a class of groups, including the
	Lorentz group, with an infinite number of elements but a finite
	number of generators.
    \item [lightlike]	$V^\mu V_\mu = 0$
    \item [Lippmann-Schwinger eqn] 
	In P.T.
	\[
	    H = H_0 + V (correction)
	    \]
	If 
	\[
	    H_0\ket{\phi} = E\ket{\phi}, \quad and \quad 
	    H\ket{\psi} = E\ket{\psi}
	    \]
	Then
	\[
	    \ket{\psi} = \ket{\phi} + \frac{1}{E-H_0}V\ket{\psi}
	    \]
	This is called the \textbf{Lippmann-Schwinger equation}. And the inverted object 
	is a kind of Green's function known as the \textbf{Lippmann-Schwinger kernel}:
	\[
	    \prod_{LS} = \frac{1}{E-H_0}
	    \]
	Define operator T, so that
	\[
	    V\ket{\psi} = T\ket{\phi}
	    \]
	So
	\[
	    \ket{\psi} = \ket{\phi} + \frac{1}{E-H_0}T\ket{\phi}
	    \]
	and 
	\[
	    T = V + V\frac{1}{E-H_0}T
	    \]
	Solve it perturbatively in V:
	\[
	    \begin{aligned}
		T &= V + V\frac{1}{E-H_0}V + V\frac{1}{E-H_0}V\frac{1}{E-H_0}V + \cdots
		  &= V + V\prod_{LS}V + V\prod_{LS}V\prod_{LS}V + \cdots
	    \end{aligned}
	    \]

    \item [little group] The representation of the full \Poincare{}
	group is induced by a representation of the subgorup of the
	\Poincare{} group that holds $p^\mu$ fixed, called the
	\emph{little group}. When $P_\mu$ is massive, the little group is
	SO(3); when $p_\mu$ is massless, the little group is ISO(2).
    \item [locality]
    \item [Lorentz group] this is the generalization of the rotation group
	to include both rotations and boosts.

    %%%%%%%%% M %%%%%%%%%%% 
    \item [Majorana spinor] A spinor whose antiparticle is itself. 
    \item [Majorana masses]
	\[
	    \mathcal{L}=i\psi^\dag_L\sigma_\mu\partial_\mu\psi_L+i\frac{m}{2}(\psi^\dag_L\sigma_2\psi^*_L-\psi^T_L\sigma_2\psi_L)
	\]
	The mass terms in this Lagrangian are called \textbf{Majorana masses}.
    \item [Majorana fermions]
	in Dirac spinors:
	\[
	    \psi=\begin{pmatrix}
		\psi_L	\\
		i\sigma_2\psi^*_L   \\
	    \end{pmatrix}
	\]

    %%%%%%%%% P %%%%%%%%%%% 
    \item [pseudo scalar] particles with odd \textbf{parity}

    %%%%%%%%% Q %%%%%%%%%%% 
    \item [quantize]	promote x and p as operators and impose the
	canonical commutation relations:    $[x, p] = i$
    \item [quantum process]  time evolution of an open quantum system ???

    %%%%%%%%% R %%%%%%%%%%% 
    \item [Rarita-Schwinger field]  spin-$\frac{3}{2}$
    \item [representation] A set of objects that mix under a transformation
	group is called a representation of the group, though technically
	the matrix embedding is the representation. A representation is a 
	particular embedding of group elements into operators that act on a 
	vector space. For finite-dimensional representations, this means an 
	embedding of the $g_i$ into matrices. 

    %%%%%%%%% S %%%%%%%%%%% 
    \item [second quantization]	canonical quantization of relativistic
	fields, 
	\[  H_0 = \int \frac{d^{3}p}{(2\pi)^3}\omega_p(a^{\dag}_p a_p +
	\frac{1}{2}) \] 
	First quantization refer to the discrete mode, for
	example, of a particle in a box. Second quantization refers to the
	integer numbers of excitations of each of these modes. There are two
	features in second quantization:
	\begin{enumerate}
	    \item We have many quantum mechanical systems - one for each
		$\vec{p}$ - all at the same time.
	    \item We interpret the nth excitation of the $\vec{p}$ harmonic
		oscillator as having n \emph{particles}.
	\end{enumerate}
    \item [S-matrix]
    \item [spacelike]	$V^\mu V_\mu < 0$
    \item [SO(n)] the group of nD rotations ($det(R) = 1$)
	
    %%%%%%%%% T %%%%%%%%%%% 
    \item [time dilation] In relativity, time dilation is a difference in 
	the elapsed time measured by two observers.
    \item [timelike]	$V^\mu V_\mu > 0$

    %%%%%%%%% U %%%%%%%%%%% 
    \item [unitary] $\Lambda^{\dag}\Lambda = 1$

    %%%%%%%%% W %%%%%%%%%%% 
    \item [Weyl spinor] Irreducible unitary spin-$\frac{1}{2}$
	representations of the \Poincare{} group.

    %%%%%%%%% $\gamma$ %%%%%%%%%%% 
    \item [$\gamma$ matrix]
	In the Weyl basis:
	\[  \gamma_\mu = 
	    \begin{pmatrix}
		0   & \sigma_\mu    \\
		\bar{\sigma}_\mu  & 0	\\
	    \end{pmatrix} 
	\]
    \item [$\gamma^5$]
	\[
	    \gamma^5 \equiv i\gamma^0\gamma^1\gamma^2\gamma^3
	\]
	In the Weyl representation:
	\[
	    \gamma^5 = 
	    \begin{pmatrix}
		-\mathcal{1}	&   \\
		    & \mathcal{1}   \\
	    \end{pmatrix}
	\]
\end{description}

\include{definition_cm}
%%%%%%%%%%%%%%%%%%%%%%%%%%%%%%%%%%%%%%%%%%%%%%%%%%%%%%%%%%%%%%%%%%%%%%%%
\section{EM}

%%%%%%%%%%%%%%%%%%%%%%%%%%%%%%%%%%%%%%%%%%%%%%%%
\subsection{General}
\textbf{degree of polarization}:    \\
\begin{equation}
    \text{degree of pol} \equiv \frac{|\sigma_{\perp}-\sigma_{//}|}{\sigma_{\perp}+\sigma_{//}}
\end{equation}

%%%%%%%%%%%%%%%%%%%%%%%%
\subsubsection{Gauge}
\begin{description}
    \item [Coulomb Gauge]
	\begin{equation}
	    \label{eqn:em::CoulombGauge}
	    \nabla\cdot\vec{A} = 0
	\end{equation}

    \item [Lorentz Gauge]
	\begin{equation}
	    \label{eqn:em::LorentzGauge}
	    \nabla\cdot\vec{A} + \partial_t\varphi = 0
	\end{equation}
\end{description}

%%%%%%%%%%%%%%%%%%%%%%%%
\subsubsection{Maxwell Equation}
\begin{equation}
    \label{eqn:Maxwell}
    \left\{
    \begin{aligned}
	\nabla\cdot{E} &= \frac{\rho}{\epsilon}	\\
	\nabla\times{B} &= \frac{1}{c}\left(J + J_D + J_{ind}\right)	\\
	\nabla\cdot{B} &= 0  \\
	\nabla\times{E} &= -\frac{1}{c}\partial_{t}B	\\
    \end{aligned}
    \right.
    \longrightarrow
    \left\{
    \begin{aligned}
	\nabla\cdot{D} &= \rho	\\
	\nabla\times{H} &= \frac{1}{c}J +\frac{1}{c}\partial_{t}D    \\
	\nabla\cdot{B} &= 0  \\
	\nabla\times{E} &= -\frac{1}{c}\partial_{t}B	\\
    \end{aligned}
    \right.
\end{equation}

In vacuum environment ($\epsilon_0 = \mu_0  = 1$):
\begin{equation}
    \label{eqn:Maxwell_vacuum}
    \left\{
    \begin{aligned}
	\nabla\cdot{\vec{E}} &= \rho	\\
	\nabla\times{\vec{B}} &= \frac{1}{c}\vec{J} +\frac{1}{c}\partial_{t}\vec{E}    \\
	\nabla\cdot{\vec{B}} &= 0  \\
	\nabla\times{\vec{E}} &= -\frac{1}{c}\partial_{t}\vec{B}	\\
    \end{aligned}
    \right.
\end{equation}
\begin{equation*}
    \begin{gathered}
	\nabla\cdot\vec{B} = 0 \Rightarrow \vec{B} = \nabla\times\vec{A}    \\
	\nabla\times\vec{E} = -\frac{1}{c}\partial_t\vec{B} \Rightarrow \vec{E} = -\frac{1}{c}\partial_t\vec{A} - \nabla\varphi
    \end{gathered}
\end{equation*}

The source equation
\begin{equation*}
    \begin{gathered}
	-\frac{1}{c}\partial_t(\nabla\cdot\vec{A}) - \nabla^2\varphi = \rho    \\
	\nabla(\nabla\cdot\vec{A} + \frac{1}{c}\partial_t\varphi) + \frac{1}{c^2}\partial^2_t\vec{A} - \nabla^2\vec{A} = \frac{\vec{j}}{c}
    \end{gathered}
\end{equation*}

\textbf{Coulomb Gauge}:	$\nabla\cdot\vec{A} = 0$
\begin{equation*}
    \begin{gathered}
	-\nabla^2\varphi = \rho	\\
	\nabla(\frac{1}{c}\partial_t\varphi) + \Box\vec{A} = \frac{\vec{j}}{c}	\\
    \end{gathered}
\end{equation*}

\indent If $\vec{j} = 0$: 
\begin{equation*}
    \begin{gathered}
	-\nabla^2\varphi = \rho	\\
	\partial_t\varphi = 0	\\
    \end{gathered}
\end{equation*}
Green function:
\begin{equation*}
    \varphi = \frac{q}{4\pi|\vec{r} - \vec{r_0}|}
\end{equation*}

\indent If $\rho = 0$:    
\begin{equation*}
    \begin{gathered}
	\varphi = 0 \\
	\Box\vec{A} = {\vec{j}}/{c}
    \end{gathered}
\end{equation*}

\textbf{Lorentz Gauge}:	$\nabla\cdot\vec{A} + \frac{1}{c}\partial_t\varphi = 0$
\begin{equation*}
    \begin{gathered}
	\Box\varphi = \rho	\\
	\Box\vec{A} = {\vec{j}}/{c}
    \end{gathered}
\end{equation*}

Far field ($r >> r_0$, retarded time T):
\begin{equation*}
    \begin{gathered}
	\varphi(t,\vec{x}) = \frac{1}{4\pi r}\int d\vec{r}_0\rho(T, \vec{r}_0)    \\
	\vec{A}(t, \vec{x}) = \frac{1}{4\pi r}\int d\vec{r}_0\vec{j}(T,\vec{r}_0)/c    \\
    \end{gathered}
\end{equation*}

%%%%%%%%%%%%%%%%%%%%%%%%%%%%%%%%%%%%%%%%%%%%%%%%
\subsection{Radiation}
\begin{description}
    \item [Retarded Time] 
	\begin{equation}
	    \label{eqn:em::retardedTime}
	    T = t-\frac{|\vec{r} - \vec{r}_0|}{c} \stackrel{r>>r_0}{\approx} t - \frac{r}{c} + \frac{\vec{n}\cdot\vec{r}_0}{c}
	\end{equation}


    \item [Radiation Power]
	\begin{equation}
	    \label{eqn:em::radiationPower}
	    \begin{gathered}
		\frac{dP}{dA} = \vec{S} = c\vec{E}\cdot\vec{B} = \frac{e^2}{16\pi^2}\frac{a^2}{c^3}\cos^2{\theta}
	    \end{gathered}
	\end{equation}
\end{description}

\begin{equation}
    \vec{E}_{rad} = -\frac{1}{c}\partial_t\vec{A}_{rad} - \nabla\phi_{rad} = \hat{n}\times\hat{n}\times\frac{1}{c}\partial_t\vec{A}_{rad}
    = \frac{e}{4\pi r c^2}\hat{n}\times\hat{n}\times\vec{a}
\end{equation}

%%%%%%%%%%%%%%%%%%%%%%%%%%%%%%%%%%%%%%%%%%%%%%%%
\subsection{Relativity}

\textbf{covariant} vector: lower indices;    \\
\textbf{contravatiant} vector: upper indices.	\\
\textbf{d'Alembertian} 
\begin{equation}
    \label{eqn:d'Alembertian}
    \Box = \partial^2_\mu = \partial_t^2-\partial_x^2-\partial_y^2-\partial_z^2.
\end{equation}

\textbf{timelike}:  $V^\mu V_\mu > 0$, for example (t, 0, 0, 0)	\\
\textbf{lightlike}:  $V^\mu V_\mu = 0$, for example (p, 0, 0, p)	\\
\textbf{spacelike}:  $V^\mu V_\mu < 0$, for example (0, x, 0, 0)
%%%%%%%%%%%%%%%%%%%%%%%%%
\subsubsection{4-Vector}
\begin{itemize}
    \item $x^\mu \equiv (ct, \vec{x})$
    \item $\partial_\mu = \frac{\partial}{\partial x^\mu} = (\partial_t, \partial_x, \partial_y, \partial_z)$
    \item $U^\mu = (u^0, \vec{u}) \equiv \frac{dx^\mu}{d\tau} = (\gamma c, \gamma\vec{v})$
    \item $p^\mu = (E/c, p_x, p_y, p_z)$
    \item $J^\mu \equiv (c\rho, \vec{J})$
    \item $A^{\mu} \equiv (\varphi, \vec{A})$
    \item $K^\mu = (\omega/c, \vec{k})$
\end{itemize}



%%%%%%%%%%%%%%%%%%%%%%%%%
\subsubsection{E and B}
\begin{equation}
    \label{eqn:em::sr:::E&B}
    \begin{aligned}
	E'_{//} &= E_{//}   \qquad&   B'_{//} &= B_{//}	\\
	E'_{\perp} &= \gamma (E_{\perp} + \beta B_{\perp})   \qquad& B'_{\perp} &= \gamma(B_{\perp} - \beta E_{\perp})
    \end{aligned}
\end{equation}
\begin{description}
    \item [Retarded Field]
	\begin{equation}
	    \label{eqn:em::retardedField}
	    \left\{
	    \begin{aligned}
		\Box\phi &= \rho	\\
		\Box \vec{A} &= \vec{j}	\\
	    \end{aligned}
            \right.
	    \Rightarrow
	    \left\{
	    \begin{aligned}
		\phi(t,\vec{r}) &= \int d^3r_0 \frac{1}{4\pi r}\frac{\rho(T,\vec{r}_0)}{1-\vec{\beta}(T)\cdot\vec{n}}   \\
		\vec{A}(t,\vec{r})  &= \int d^3r_0 \frac{1}{4\pi r}\frac{\vec{j}(T,\vec{r}_0)}{1-\vec{n}\cdot\vec{\beta}(T)}   \\ 
	    \end{aligned}
            \right.
	\end{equation}
\end{description}

%%%%%%%%%%%%%%%%%%%%%%%%%
\subsubsection{Radiation}
\begin{equation}
    \vec{E(t)} = \hat{n}\times\hat{n}\times\frac{1}{c}\partial_t\vec{A}_{ret} = \frac{q}{4\pi rc^2}\left[\frac{\hat{n}\times(\hat{n}-\vec{\beta})\times\vec{a}}{(1-\hat{n}\cdot\vec{\beta})^3}\right]_{ret}
\end{equation}

\chapter{SM}
%%%%%%%%%%%%%%%%%%%%%%%%%%%%%%%%%%%%%%%%%%%%%%%%%%%%%%%%%%%%%%%%%%%%%%%%
\section{General}
Equipartition theorem: a body in thermal equilibrium should have energy 
equally distributed among all possible modes. (Similar to the equal 
probability principle)

In equilibrium, the number density is determined by Boltzmann dist.
\begin{equation}
    \label{Boltzmann distribution}
    n_i = Ne^{-\beta E_i}
\end{equation}
\subsection{Blackbody Radiation}
Quantum explanation: In mode $\omega_n$, it can be excited an integer 
number j times, giving energy $jE_n = j(\hbar\omega_n)$ in that mode.
The probability of find that much energy in that mode is the same 
as the probability of finding energy in anything, proportional to 
the Boltzmann weight exp(-energy/$k_BT$). Thus, the expectation value of
energy in each mode is
\begin{equation}
    <E_n> = \frac{\sum_{j=o}^\infty(jE_n)e^{-jE_n\beta}}{\sum_{j=o}^\infty e^{-jE_n\beta}} = \frac{-\frac{d}{d\beta}\frac{1}{1-e^{-\hbar\omega_n\beta}}}{\frac{1}{1-e^{-\hbar\omega_n\beta}}} = \frac{\hbar\omega_n}{e^{\hbar\omega_n\beta}-1}
\end{equation}
where $\beta=1/k_BT$

\include{definition_qm}

% part 3: detailed proofs
\part{Proofs}
\chapter{Classical Mechanics}


\section{Oscillation}
\subsection{Forced Oscillation}
\begin{equation}
    \label{eqn:forced oscillation}
    x^{''} + \omega_0^2x = F\cos\omega t
\end{equation}

\chapter{Electramegnetic Mechanics}

\section{EM in relativity}
\begin{equation}
    F_{\mn} \equiv 
	\begin{pmatrix}
	    0	& E_x	& E_y	& E_z	\\
	    -E_x    & 0	& -B_z	& B_y	\\
	    -E_y    & B_z   & 0	& -B_x	\\
	    -E_z    & -B_y  & B_x   & 0	\\
	\end{pmatrix}
	= \partial_\mu A_\nu - \partial_\nu A_\mu
\end{equation}
The Maxwell Eqn:
\begin{gather}
    \partial_\mu F_{\mn} = \Box A_\nu - \partial_\nu(\partial_mu A_\mu) =
    \Box A_\mu = 0, \notag\\
    \partial_\mu F_{\nu\rho} + \partial_\nu F_{\rho\mu} + \partial_\rho F_{\mn} = 0
\end{gather}

\include{SM}
\chapter{Quantum Mechanics}
%%%%%%%%%%%%%%%%%%%%%%%%%%%%%%%%%%%%%%%%%%%%%%%%
\subsection{Harmonic Oscillator}
\[
    \bra{n'}x\ket{n}=\sqrt{\hbar/2m\omega}(\sqrt{n+1}\delta_{n',n+1}+\sqrt{n}\delta_{n',n-1})
    \]
For $V(r)=\frac{1}{2}m\omega^2r^2=\frac{1}{2}kr^2$
\[
    \phi_0(r)=[(\frac{\alpha}{\sqrt{\pi}})^{1/2}e^{-\frac{1}{2}\alpha^2x^2}]
    [(\frac{\alpha}{\sqrt{\pi}})^{1/2}e^{-\frac{1}{2}\alpha^2y^2}]
    [(\frac{\alpha}{\sqrt{\pi}})^{1/2}e^{-\frac{1}{2}\alpha^2z^2}]
    \]
where $\alpha=(\frac{mk}{\hbar^2})^{1/4}=(\frac{m\omega}{\hbar})^{1/2}$
%%%%%%%%%%%%%%%%%%%%%%%%%%%%%%%%%%%%%%%%%%%%%%%%%%%%%%%%%%%%%%%%%%%%%%%%

\section{Rotation}

%%%%%%%%%%%%%%%%%%%%%%%%%%%%%%%%%%%%%%%%%%%%%%%%
\subsection{Vector}
In QM, we demand that the expectation value of a vector operator transforms
under rotation like a classical vector.
\begin{gather*}
    \ket{\alpha}\rightarrow\ket{\alpha}^R=D(R)\ket{\alpha}\\
    ^{R}\bra{\alpha}V_i\ket{\alpha}^R=\bra{\alpha}D^{\dag}V_{i}D\ket{\alpha}=R_{ij}\bra{\alpha}V_j\ket{\alpha}\\
    D^{\dag}V_{i}D=R_{ij}V_j
\end{gather*}
With infinitesimal transform, we will find:
\begin{equation}
    [V_i,J_j]=i(\epsilon_{ijk}V_k
\end{equation}

%%%%%%%%%%%%%%%%%%%%%%%%%%%%%%%%%%%%%%%%%%%%%%%%
\subsection{Tensor}
We define spherical tensor of rank k with $(2k+1)$ components labelled by q
as
\begin{equation}
    D^{\dag}(R)T^{(k)}_{q}D(R)=\displaystyle\sum_{q^\prime=-k}^{k}D^{(k)*}_{qq^\prime}T^{(k)}_{q^\prime}
\end{equation}
%%%%%%%%%%%%%%%%%%%%%%%%%%%%%%%%%%%%%%%%%%%%%%%%
\subsection{Spin-$1/2$} 
\[ 
D(R_{\hat{n}}(\phi)) = e^{-i\frac{\phi}{\hbar}\hat{n}\cdot\vec{S}} = e^{-i\frac{\phi}{2}\hat{n}\cdot\vec{\sigma}} = 
cos(\frac{\phi}{2}) - i \hat{n} \cdot \vec{\sigma} sin(\frac{\phi}{2}) = 
\begin{pmatrix}
    \cos(\frac{\phi}{2}) - in_{z}\sin(\frac{\phi}{2}) & (-i n_{x} - n_{y})\sin(\frac{\phi}{2}) \\ 
    (-i n_{x} + n_{y})\sin(\frac{\phi}{2}) &  \cos(\frac{\phi}{2}) + in_{z}\sin(\frac{\phi}{2})  
\end{pmatrix}
\]

%%%%%%%%%%%%%%%%%%%%%%%%%%%%%%%%%%%%%%%%%%%%%%%%
\subsection{Spin-1}
\begin{equation}
    S_x=\frac{\hbar}{\sqrt{2}}
    \begin{pmatrix}
	0   & 1	& 0 \\
	1   & 0	& 1 \\
	0   & 1	& 0 \\
    \end{pmatrix}\quad
    S_y=\frac{\hbar}{\sqrt{2}}
    \begin{pmatrix}
	0   & -i    & 0 \\
	i   & 0	& -i \\
	0   & i	& 0 \\
    \end{pmatrix}\quad
    S_z=\hbar
    \begin{pmatrix}
	1   & 0	& 0 \\
	0   & 0	& 0 \\
	0   & 0	& -1 \\
    \end{pmatrix}\quad
\end{equation}

%%%%%%%%%%%%%%%%%%%%%%%%
\subsubsection{Wigner-Echart theorem}
The matrix elements of tensor operators with respect to angular momentum
eigenstates.
\begin{equation}
    \bra{\alpha^\prime,j^{\prime}m^\prime}T^{(k)}_{q}\ket{\alpha,jm}
    =\frac{\bra{jk;mq}\ket{jk;j^{\prime}m^\prime}}{\sqrt{2j+1}}{\bra{\alpha^\prime,j^\prime}T^{(k)}\ket{\alpha,j}}
\end{equation}

%%%%%%%%%%%%%%%%%%%%%%%%
\subsubsection{Zeeman-effect}
This is effect only in \textbf{weak magnetism}

%%%%%%%%%%%%%%%%%%%%%%%%
\subsubsection{Paschen-Back effect}
This happen with \textbf{strong magnetism}($\frac{\Phi}{\Phi_0}\gg\alpha^2\sim10^{-4}$).


%%%%%%%%%%%%%%%%%%%%%%%%%%%%%%%%%%%%%%%%%%%%%%%%%%%%%%%%%%%%%%%%%%%%%%%%
\section{Symmetry}

%%%%%%%%%%%%%%%%%%%%%%%%%%%%%%%%%%%%%%%%%%%%%%%%
\subsection{Parity} 
Parity of Spherical Harmonics:
\[
    r\rightarrow{}r,\quad\theta\rightarrow\pi-\theta,\quad\phi\rightarrow\phi+\pi
    \]
So
\[
    Y_{ll}(\theta,\phi)\xrightarrow{P}Y_{ll}(\pi-\theta,\phi+\pi)=e^{il\pi}e^{il\phi}\sin^l(\pi-\theta)=(-1)^{l}Y_{ll})\theta,\phi
    \]
Because $L_\_$ is an \emph{axial vector}, so $P(L_\_)=1$.
\[
    L_\_Y_{ll}\xrightarrow{P}(-1)^{l}L_\_Y_{ll}
    \]
To get
\[
    Y_{lm}(\theta,\phi)\xrightarrow{P}Y_{lm}(\pi-\theta,\phi+\pi)=(-1)^{l}L_\_Y_{l,m+1}(\theta,\phi)
    \]

\subsection{Time reversal}
\[
    \ket{jm}\xrightarrow{T}(-1)^m\ket{j,-m}
    \]

\chapter{Special Relativity}
%%%%%%%%%%%%%%%%%%%%%%%%%%%%%% Relativity %%%%%%%%%%%%%%%%%%%%%%%%%%%%%%
Basic assumption:
\begin{itemize}
    \item Laws of physics should be invariant under special relativity 
	transformations: rotations and velocity boosts	\eqnum\label{SM:assumption1}
    \item Speed of light is constant	\eqnum\label{SM:assumption2}
\end{itemize}

According to \ref{SM:assumption1} 
(the distance travelled by light within time
interval $dt$ is $dl = \sqrt{dx_idx_i}$, therefore $c^2dt^2 - dx_idx_i = c^2dt'^2 - dx'_idx'_i = 0$), 
generalize it, we will get the length $ds^2$ and it should be invariant under
transformations:
\begin{equation}
    ds^2 = c^2dt^2 - dx_idx_i = c^2dt'^2 - dx'_i dx'_i
\end{equation}
define Lorentz 4-vector $dx^\mu = (dt, dx^1, dx^2, dx^3)$ and the metrix tensor:
$$ g_{\mu\nu} = diag(1,-1, -1, -1) $$
then $ds^2 = g_{\mu\nu} dx^\mu dx^\nu$
so:
\begin{equation}
    x^\mu x_\mu = x^\mu g_{\mu\nu}x^\nu = x'^\mu x'_\mu
\end{equation}
let the transformation be: $x'^\mu = \Lambda^\mu_\nu x^\nu$ or $x' = \Lambda x$
then:
\begin{equation}
    x^T g x = x'^T g x' = (\Lambda x)^T g (\Lambda x) = x^T (\Lambda^T g \Lambda) x
\end{equation}
we get:
\begin{equation}
    \label{SR:LG}
 g = \Lambda^T g \Lambda
\end{equation}
if $g = E$ where $E = diag(1, 1, 1,1 )$ is the identity matrix, then we get:
$$ E = \Lambda^T \Lambda \Rightarrow \Lambda^T = \Lambda^{-1}$$
This defines the special orthogonal matrices (SO(4)), which must have $det\Lambda = \pm 1$

\subsubsection{Lorentz Group, SO(3, 1)}
However, our metric tensor g is not the identity matrix, but rather a mixed 
metrix with 3 -1 entries and 1 +1 entry. We call all elements that satisfy
\ref{SR:LG} as Lorentz Group SO(3, 1).

Here are a few examples of elements in SO(3, 1):
\begin{equation*}
    \Lambda_R = 
    \begin{pmatrix}
	1   & 0	\\
	0   & R
    \end{pmatrix}
    \quad \text{where R are the 3x3 rotation matrix SO(3)}
\end{equation*}

\begin{equation*}
    \Lambda_{B_x} = 
    \begin{pmatrix}
	\cosh\eta   & -\sinh\eta    & 0	& 0 \\
	-\sinh\eta   & \cosh\eta    & 0	& 0 \\
	0   & 0 & 1 & 0 \\
	0   & 0 & 0 & 1 \\
    \end{pmatrix}
    \text{velocity boost in the x direction}
\end{equation*}
where $\cosh\eta = \gamma$ and $\sinh\eta = \beta\gamma$, with $\gamma = \frac{1}{\sqrt{1-v^2/c^2}}$
and $\beta = v/c$

\section{Relativistic generalization from Classical results}
The power of radiation is (cgs units):
\[
    P = \frac{2}{3}\frac{q^2a^2}{c^3}
\]
Genelize it to \textbf{covariant} form:
\[
    P = \frac{2}{3}\frac{q^2}{m^2c^3}|\dot{\vec{P}}|^2
\]
The power P should be \emph{Lorentz invariant}, so $|\dot{\vec{P}}|^2$ 
should include the Lorentz scalar by taking the inner product of the 
four-acceleration $a^\mu = dp^\mu/d\tau$ with itself, so:
\[
    P = -\frac{2}{3}\frac{q^2}{m^2c^3}\frac{dp_\mu}{d\tau}\frac{dp^\mu}{d\tau}
\]
where:
\[
    \frac{dp_\mu}{d\tau}\frac{dp^\mu}{d\tau} = \beta^2 \left( \frac{dp_0}{d\tau} \right)^2 - \left( \frac{d\vec{p}}{d\tau} \right)^2
\]

The above inner product can also be written in terms of $\vec{\beta}$ and
its time derivative:
\[
    P = \frac{2q^2\gamma^6}{3c}\left[ (\dot{\vec{\beta}})^2 - (\vec{\beta} \times \dot{\vec{\beta}})^2 \right]
\]



%%%%%%%%%%%%%%%%%%%%%%%%%%%%%%%%%%%%%%%%%%%%%%%%%%%%%%%%%%%%%%%%%%%%%%%%
\section{Relativity}
Two conditions of \SR{}. 
\begin{itemize}
    \item general physical rules, which means in different flames, we can 
	observe the same EOM.
    \item constant c
\end{itemize}
let $X^\mu=(ct, \vec{x})$, so in order to make c constant, we have
\[
    -(ct')^2+\vec{x'}^2 = (ct)^2 +\vec{x}^2
\]
The \textbf{simplest} way to preserve the length is to construct a \textbf{linear map}:
\[
    X'^\mu = {(\mathcal{L})^\mu}_{\nu}X^\nu
\]
Where ${(\mathcal{L})^\mu}_{\nu} = {(\mathcal{L}^T)_\nu}^{\mu} = {\Lambda^\mu}_\nu$.
Note that for a matrix, what important is the order of index, not their position, so
\[
    {\mathcal{L}^{\mu}}_{\nu}={\mathcal{L}_{\mu}}^{\nu}={(\mathcal{L}^T)_{\nu}}^{\mu}={(\mathcal{L}^T)^{\nu}}_{\mu}
\]
Let 
\[
    A = \begin{pmatrix}
	A^0 \\
	A^1 \\
	A^2 \\
	A^3 \\
    \end{pmatrix}\quad
    B = \begin{pmatrix}
	B^0 \\
	B^1 \\
	B^2 \\
	B^3 \\
    \end{pmatrix}
\]
and 
\[
    A_\mu  \equiv g_{\mn}A^\nu
\]
where 
\begin{equation}
    g_{\mn} = diag(-1, 1, 1, 1)
\end{equation}
\[
    g^{\mn}g_{\nu\sigma}={g^{\mu}}_\sigma={\delta^{\mu}}_\sigma 
\]
So, we can define:
\[
    A\cdot B \equiv -A^0B^0 + A^1B^1 + A^2B^2 + A^3B^3 = A^\mu B_\mu = A^\mu g_{\mn}B^\nu
\]
Inserting $\mathcal{L}\mathcal{L}^T$ into the above equation:
\[
    \begin{aligned}
	A_\mu B^\mu= A^\mu g_{\mn}B^\nu = (\mathcal{L}^{-1}A')^{\mu}g_{\mn}(\mathcal{L}^{-1})^\nu \\
	= (\mathcal{L}^{-1})^\mu_{\nu}A'^{\nu}g_{\mn}(\mathcal{L}^{-1})^\nu_{\rho}B'^\rho    \\
	= A'^{\nu}(\mathcal{L}^{-1})^\mu_{\nu}g_{\mn}(\mathcal{L}^{-1})^\nu_{\rho}B'^\rho    \\
	= A'^{\nu}({\mathcal{L}^{-1}}^T)_\nu^{\mu}g_{\mn}(\mathcal{L}^{-1})^\nu_{\rho}B'^\rho    \\
	= A'^{\nu}g'_{\nu\rho}B'^\rho
    \end{aligned}
\]
In \SR{}, we require that:
\[
    g'_{\nu\rho}=({\mathcal{L}^{-1}}^T)_\nu^{\mu}g_{\mn}(\mathcal{L}^{-1})^\nu_{\rho}=g_{\nu\rho}
\]
So we get
\[
    {\mathcal{L}^{-1}}^Tg\mathcal{L}^{-1}=g \rightarrow {\mathcal{L}^{-1}}^T=g\mathcal{L}g \rightarrow \mathcal{L}^{-1}=g\mathcal{L}^Tg
\]
With these conditions, we can easily find out $\mathcal{L}$. For 2D-coordinate $(ct, x)$, we have
\[
    \mathcal{L}(v) = 
	\begin{pmatrix}
	    \gamma  &	-\beta\gamma	\\
	    -\beta\gamma    & \gamma	\\
	\end{pmatrix}
\]
Where $\beta=\frac{v}{c}$ and $\gamma=\frac{1}{\sqrt{1-\beta^2}}$. 
Usually, we would like to use another quantity in \SR{}, which is rapidity:
\[
    \tanh{y}\equiv\frac{v}{c} \rightarrow y=\frac{1}{2}ln(\frac{1+\beta}{1-\beta})
\]
And correspondingly
\[
    \gamma=\cosh{y}, \quad \beta\gamma\sinh{y}
\]



%%%%%%%%%%%%%%%%%%%%%%%%%%%%%%%%%%%%%%%%%%%%%%%%%%%%%%%%%%%%%%%%%%%%%%%%
\section{General Relativity}
Einstein's relativity thoery says that no info travels faster than the speed the light. 
On the other hand, Neuton's familiar law of gravity says the force of gravity
acts instantaneously on distant bodies. To resolve this parabox, Einstein 
proposed that matter bends and warps space and time, giving rise to gravity. 

GR successfully explain the movement of Mercury, whose orbit is not a perfect ellipse.
Each time it revolves around the sun, it always comes back very slightly ahead of 
the ellipse. The effect is extremely small, scientists had noted this effect
before Einstein, but could not account for it with Neutro's thoery of gravity. The
anomalous 43 arcseconds/century was explained by Einstein's theory.


\newcommand{\QM}{Quamtum Mechanics}
\newcommand{\QFT}{Quantum Field Theory}
\newcommand{\RQFT}{Relativistic Quantum Field Theory}
\newcommand{\FT}{Fourier Transform}
\newcommand{\FFT}{Fast Fourier Transform}
\newcommand{\LT}{Lorentz Transform}
\newcommand{\LI}{Lorentz Invariant}
\newcommand{\LG}{Lorentz Group}
\newcommand{\KG}{Klein-Gordon}
\newcommand{\EL}{Euler-Lagrange}
\newcommand{\sr}{special relativity}
\newcommand{\Poincare}{Poincar$\acute{\textrm{e}}$}

\chapter{QFT}

Questions:
\begin{itemize}
    \item What's a field, classical field is a spatial distribution, quantum
	field is a analogy to classical one, but make up of creation and
	destruction operator. The problem is that we \textbf{define} a field which
	collects creator and destructor operators and it works !!! What's
	the logic to define such a field ???
    \item How to construct Lagrangian from a field $\leftarrow$ Lorantz
	invariant.
    \item EOM
    \item Noether theorem. Conserved current and charge
    \item Symmetry. What's each group? corresponding representation. How to
	embed particles into lorentz group and unitary group?
    \item Couple of scalar field to $A_\mu$
    \item Anticommuting spinors:
	\[  -(\gamma_0)_{\alpha\beta}\psi_\alpha\psi^*_\beta=(\gamma_0)_{\alpha\beta\psi^*_\beta\psi_\alpha}	\]
\end{itemize}

%%%%%%%%%%%%%%%%%%%%%%%%%%%%%%%%%%%%%%%%%%%%%%%%%%%%%%%%%%%%%%%%%%%%%%%
\section{Motivation}

\emph{particle} number is not conserved. The creation and destruction of
particls, which is possible due to the most famous eqn. of \sr{} $E =
mc^{2}$. \emph{Lorentz invariance} guides the definition of particle.

Why QFT: \fbox{quantum mechanics plus \Poincare{} symmetry}. \\
\fbox{Quantum field theory is just quantum mechanics with an infinite number
of harmonic oscillators}

The \QM{} can describe a system with a fixed number of particles
in terms of a many-body wave function. The \RQFT{}
with creation and annihilation operators was developed in order to include
processes in which the number of particles is not conserved,
and to describe the conversion of mass into energy and vice versa.
A consequence of relativity is that the number of particles isn't fixed, ($E=mc^2$)
though the converse is false: particle production can happen without relativity.

Construct $\mathcal{L}$ from field $\phi$ and its derivative under the rule
of \LI{}. How to incorporate symmetry ?

QFT is the quantum mechanics of {\Large \textit{extensive degrees of freedom}}
$\bra{x}\ket{\phi} = \phi(x)$ is a function of space, the wavefunction.
This looks like a field. It is not what we mean by field in QFT.
meaningless phrases like "second quantization" may conspire to try
to confuse you.

It is not a coincidence that the harmonic oscillator plays an important
role. After all, electromagenetic waves oscillate harmonically.

There are two common ways to quantize a field theory:
\begin{itemize}
    \item canonical quantization
    \item Feynman path integral
    \item Ohter alternatives: perturbation theory
\end{itemize}

\textbf{Second quantization}: first quantization refers to the discrete
modes($d\vec{x}d\vec{p}\sim \hbar$), for example, of a particle in a box.
Second quantization refers to the integer numbers of excitations of each of
these modes. However, this is somewhat misleading-the fact that there are
discrete modes is a classical phenomenon. The two steps really are (1)
interpret there modes as having energy $E=\hbar\omega$ and (2) quantize each
modes as a harmonic oscillator. In that sense we are only quantizing once.

There are two new features in second quantization:
\begin{enumerate}
    \item We have many quantum mechanical systems – one for each $\vec{p}$  – all at the same time.
    \item We interpret the nth excitation of the $\vec{p}$  harmonic
	oscillator as having n \textbf{particles}.
\end{enumerate}
%%%%%%%%%%%%%%%%%%%%%%%%%%%%%%%%%%%%%%%%%%%%%%%%%%%%%%%%%%%%%%%%%%%%%%%
\section{Convention}
In relativity, the symmetry refer to invariance after the transformation of
\emph{coordinate system}. So the rotation is:
\begin{equation}
    R = 
	\begin{pmatrix}
	    \cos\theta	& \sin\theta	\\
	    -\sin\theta	& \cos\theta
	\end{pmatrix}
\end{equation}
or in other way:
\[ R^{T}\mathds{1}R = \mathds{1} \]
    
%%%%%%%%%%%%%%%%%%%%%%%%%%%%%%%%%%%%%%%%%%%%%%%%
\subsection{4D time-space}   
$dx^{\mu} \equiv (dt, d\vec{x})^{\mu}$ \\
$ ds^{2} = dt^{2} - d\vec{x}\cdot d\vec{x} =
\eta_{\mn}dx^{\mu}dx^{\nu}$	with 
\begin{equation}
    \eta^{\mn} = \eta_{\mn} = 
    \begin{pmatrix}
	+1  & 0	  & 0  & 0	\\
	0   & -1  & 0  & 0	\\
	0   & 0	  & -1 & 0	\\
	0   & 0	  & 0  & -1
    \end{pmatrix}_{\mn}
    \label{Minkowski metric}
\end{equation}
Lorentz transformation acting on 4-vectors are matrices $\Lambda$ satisfying
\[
    \Lambda^T\eta\Lambda = \eta = 
    \begin{pmatrix}
	+1  &	&   &	\\
	    & -1&   &  	\\
	    &	& -1&  	\\
	    &	&   & -1\\
    \end{pmatrix}
    \]
    
Rotation and boost in 4D time-space around x axes is:
\begin{equation*}
    \begin{pmatrix}
	1   &	&   &	\\
	    & 1 &   &	\\
	    &	& \cos\theta_x	& \sin\theta_x	\\
	    &	& -\sin\theta_x	& \cos\theta_x	\\
    \end{pmatrix},
    \begin{pmatrix}
	\cosh\beta_x	& \sinh\beta_x	&   &	\\
	\sinh\beta_x	& \cosh\beta_x	&   &	\\
	    &	& 1 &  	\\
	    &	&   & 1	\\
    \end{pmatrix}
\end{equation*}
    
\subsubsection{Lorentz transformation}
Scalar field
\[
    \phi(x^\mu)\rightarrow\phi((\Lambda^{-1})^\mu_\nu x^\nu)
    \]
Vector field
\[
    V^\mu\rightarrow\Lambda^\mu_\nu V^\nu
    \]
Tensor fields
\[
    T^{\mn}\rightarrow\Lambda^\mu_\alpha\Lambda^]\nu_\beta T^{\alpha\beta}
    \]
%%%%%%%%%%%%%%%%%%%%%%%%%%%%%%%%%%%%%%%%%%%%%%%%
\subsection{quantization}
\[ [a_k, a_p^{\dag}] = (2\pi)^{3}\delta^{3}(\vec{p}-\vec{k}), 
a_p^{\dag}\ket{0} = \frac{1}{\sqrt{2\omega_p}}\ket{\vec{p}} \]
\[
    \mathds{1}=\int\frac{d^3p}{(2\pi)^3}\frac{1}{2\omega_p}\ket{\vec{p}}\bra{\vec{p}}
    \]

\begin{description}
    \item [Function derivatives]
	$\frac{\delta \phi(x)}{\delta \phi(y)} = \delta(x-y)$, 
	\[ 
	\frac{\partial(\partial_\alpha A_\alpha)^2}{\partial(\partial_{\mu}A_\nu)}
	=2(\partial_{\alpha}A_\alpha)\frac{\partial(\partial_{\beta}A_\gamma)}{\partial(\partial_{\mu}A_\nu)}g_{\beta\gamma}
	=2(\partial_{\alpha}A_\alpha)g_{\beta\mu}g_{\gamma\nu}g_{\beta\gamma}
	\]
    \item [notation] $\phi$ and $\pi$ for scalar fields, $\psi, \xi,
	\chi$ for fermions, $A_{\mu}, J_{\mu}, V_{\mu}$ for vectors and
	$h_{\mu},T_{\mu}$ for tensors.
    \item [Kinetic term] Anything with just two fields of the same or
	different type can be called a kinetic term. Kinetic terms tell you
	about the free (non-interacting) behavior. Though, sometimes it is useful to
	think of a \emph{mass term} such as $m^2\phi^2$, as an interaction
	rather than a kinetic term.
    \item [Boundary conditions] we will always assume that our fields vanish
	on asymptotic boundaries. so we can integrate by part:
	\emph{\[ A\partial_\mu B = -(\partial_\mu A) B\]}
\end{description}

We define quantum fields as integrals over creation and annihilation
operators for each momentum: (Why define it this way ???)
\begin{equation}
\phi_0(\vec{x}) = \int \frac{d^{3}p}{(2\pi)^3}
\frac{1}{\sqrt{\omega_p}}(a_pe^{i\vec{p}\vec{x}} + a_p^\dag
e^{-i\vec{p}\vec{x}})
    \label{eqn:free_field}
\end{equation}
\[ \ket{\vec{x}} = \phi_0(\vec{x}) \ket{0} \]

There is no physical content in the above equation. It is just a definition.
The physical content is in the algebra of $a_p$ and $a_p^{\dag}$ and in the
Hamiltonian $H_0$. Nevertheless, we will see that collections of $a_p$ and
$a_p^{\dag}$ in the form of Eq.\ref{eqn:free_field} are very useful in
quantum field theory.

Following this defition, we can get:
\[ \pi(\vec{x}) \equiv \partial_t \phi(\vec{x})|_{t=0} = 
-i \int \frac{d^{3}p}{(2\pi)^3} \sqrt{\frac{\omega_p}{2}}(a_pe^{i\vec{p}\vec{x}} - a_p^\dag e^{-i\vec{p}\vec{x}}) \]
$\pi[\phi, \dot{\phi}]$ can also be implicityly defined as:
\[ \frac{\partial \mathcal{H}[\phi, \pi ]}{\partial \pi } = \dot{\phi}\]

\subsubsection{Schrodinger eqn}
\[
    \begin{aligned}
	i\partial_t\psi{x}&=i\partial_t\bra{x}\ket{\psi}=i\partial_t\bra{0}\phi(\vec{x},t)\ket{\psi}=i\bra{0}\partial_t\phi(\vec{x},t)\ket{\psi}\\
	&=\bra{0}\int\frac{d^3p}{(2\pi)^3}\frac{\sqrt{\vec{p}^2+m^2}}{2\omega_p}(a_pe^{-ipx}-a^\dag_pe^{ipx})\ket{\psi}\\
	&=\bra{0}\sqrt{m^2-\vec{\nabla}^2}\phi_0(x)\ket{\psi}\\
    \end{aligned}
    \]
So
\[
    i\partial_t\psi(x)=\sqrt{m^2-\vec{\nabla}^2}\psi(x)=\left(m-\frac{\vec{\nabla}^2}{2m}+\mathcal{O}(\frac{1}{m^2})\right)\psi(x)
    \]
To get
\[
    i\partial_t\psi(x)=-\frac{\vec{\nabla}^2}{2m}\psi(x)
    \]
%%%%%%%%%%%%%%%%%%%%%%%%%%%%%%%%%%%%%%%%%%%%%%%%
\subsection{Hamiltonian \& Lagrangian}
Why do we restrict to Lagrangians of the form $\mathcal{L}[\phi,
\partial_\mu \phi]$? First of all, this is the form that all "classical" 
Lagrangians had. If only first derivatives are involved, boundary 
conditions can be specified by initial positions and velocities only, in
accordance with Newton's laws. In the quantum theory, if kinetic terms 
have too many derivatives, for example $\mathcal{L} = \phi\Box^2\phi$, 
there will generally be disastrous consequences. For example, there may be
states with negative energy or negative norm, permitting the vacuum to decay. 
But interactions with multiple derivatives may occur. Actually, they must
occur due to quantum effects in all but the simplest renormalizable field
theories; for example, they are generic in all effective field theories.

\emph{Hamiltonian} and \emph{Lagrangian} density: ( How to connect field
$\phi$ with $\mathcal{L}$ or $\mathcal{H}$ ??? ).
\[ \mathcal{L}[\phi,\dot{\phi}] = \pi[\phi, \dot{\phi}]\dot{\phi} -
\mathcal{H}[\phi, \pi[\phi, \dot{\phi}]]  \]
Or inversely:
\[ \mathcal{H}[\phi,\pi] = \pi\dot{\phi}[\phi, \pi] -
\mathcal{L}[\phi, \dot{\phi}[\phi, \pi]],   
\frac{\partial \mathcal{L}[\phi, \dot{\phi}]}{\partial \dot{\phi} } = \pi\]

The Halmiltonian corresponds to a conserved quantity - the total energy of
the system - while the Lagrangian does not. The problem with Halmiltonian is
that they are not Lorentz invariant. It is the 0 component of a Lorentz
vector: $P^\mu = (H, \vec{P})$. And $\mathcal{H}$ is the 00 compnent of a
Lorentz tensor, the energy-momentum tensor $\mathcal{T}_{\mn}$. Halmiltonians
are great for non-relativistic systems, but for relativistic systems we will
almost exclusively use Lagrangians.

Time evolution is generated by a hamiltonian H.
$i\hbar\partial_{t}\ket{\phi} = H\ket{\phi}$

%%%%%%%%%%%%%%%%%%%%%%%%%%%%%%%%%%%%%%%%%%%%%%%%
\subsection{Norther's theorem}
If there is such a symmetry that depends on some parameter $\alpha$ that can
be taken small (continuous), than we find:
\[ 0 = \frac{\delta\mathcal{L}}{\delta\alpha} =
\displaystyle\sum_n\left\{ \left[\frac{\partial\mathcal{L}}{\partial{\phi_n}} - 
\partial_\mu\frac{\partial\mathcal{L}}{\partial(\partial_\mu\phi_n)}\right]\frac{\delta\phi_n}{\delta\alpha} 
+
\partial_\mu\left[\frac{\partial\mathcal{L}}{\partial(\partial_\mu\phi_n)}\frac{\delta\phi_n}{\delta\alpha}\right]
\right\} \]
If the EOM is satisfied, then it reduces to $\partial\mu J_\mu = 0$, where 
\begin{equation}
    \label{Norther current}
    J_\mu =
    \displaystyle\sum_n\frac{\partial\mathcal{L}}{\partial(\partial_\mu\phi_n)}\frac{\delta\phi_n}{\delta\alpha}
\end{equation}
A vector field $J_\mu$ that satisfies $\partial_\mu J_\mu = 0$ is called
\emph{conserved current}. The total charge Q, defined as: 
\[ Q = \int d^3xJ_0 \]
satisfies 
\[ \partial_{t}Q = \int d^3x\partial_{t}J_0 = \int
d^3x\vec{\nabla}\cdot\vec{J} = 0 \]
\textbf{Noether's theorem}: If a Lagrangian has a continuous symmetry then
there exists a current associate with that symmetry that is conserved when
the equations of motion are satisfied.
\begin{itemize}
    \item continuous
    \item the current is conserved \textit{on-shell}, that is, when the EOM
	are satisfied. (The field dies out at asymptotic boundary).
    \item It works for \textit{global symmetries}, parametrized by number
	$\alpha$, not only for \textit{local(gauge) symmetries} parametrized
	by functions $\alpha(x)$.
\end{itemize}

%%%%%%%%%%%%%%%%%%%%%%%%%%%%%%%%%%%%%%%%%%%%%%%%
\subsection{Energy-Momentum Tensor}
There is a very important case of Noether's theorem that applies to a global
symmetry of the action, not the Lagrangian.
\textit{global} space-time translation $\rightarrow$ energy-momentum tensor.  \\
\[ \phi(x) \rightarrow \phi(x+\xi) = \phi(x) + \xi^\nu\partial_\nu\phi(x) + \cdots \]
With infinitesimal change:
\[ \frac{\delta\phi}{\delta\xi^\nu} = \partial_\nu\phi, 
\frac{\delta\mathcal{L}}{\delta\xi^\nu} = \partial_\nu\mathcal{L}, 
\]
So:
\[ \delta S = \int d^4x\delta\mathcal{L} = \xi^\nu\int
d^4x\partial_\nu\mathcal{L} = 0 \]
which leads to:
\[ \partial_\nu\mathcal{L} =
\frac{\delta\mathcal{L}[\phi_n,\partial_\mu\phi_n]}{\delta\xi^\nu} = 
\partial_\mu\left(
\displaystyle\sum_n\frac{\partial\mathcal{L}}{\partial(\partial_\mu\phi_n)}\frac{\delta\phi_n}{\delta\xi^\nu}
\right)
    \]
Or equivalently
\[
\partial_\mu\left(
\displaystyle\sum_n\frac{\partial\mathcal{L}}{\partial(\partial_\mu\phi_n)}\partial_\nu\phi_n
- g_{\mn}\mathcal{L}
\right) = 0
    \]
The four symmetries have produced four Noether current, one for each $\nu$:
\begin{equation}
    \mathcal{T}_{\mn} = 
    \displaystyle\sum_n\frac{\partial\mathcal{L}}{\partial(\partial_\mu\phi_n)}\partial_\nu\phi_n
    - g_{\mn}\mathcal{L}
\end{equation}
The corresponding conserved current:
\[
    Q_\nu = \int d^3x\mathcal{T}_{0\nu} \]


Electron has an inherent two-valuedness called spin, while a photon has
an inherent two-valuedness called polarization.


%%%%%%%%%%%%%%%%%%%%%%%%%%%%%%%%%%%%%%%%%%%%%%%%%%%%%%%%%%%%%%%%%%%%%%%
\section{Lagrangian}
In QFT, we will use Lagrangian rather than Hamiltonian, because it is
Lorentz invariant. To construct Lorentz invariant Lagrangian from field, one
just need to add Lorentz invariant terms into it.

A vector field $A_\mu$ is just four scalars until we construct it with
$\partial_\mu$ in the Lagrangian, as in the $(\partial_{\mu}A_\mu)^2$ part
of $F_{\mn}^2$.

%%%%%%%%%%%%%%%%%%%%%%%%%%%%%%%%%%%%%%%%%%%%%%%%%%%%%%%%%%%%%%%%%%%%%%%
\section{Symmetry}
Our universe has a number of apparent symmetries that we would like our
quantum field theory to respect. One symmetry is that no place in space-time
seems any different from any other place. Thus, our theory should be
translation invariant: if we take all our fields $\phi(x)$ and replace them
by $\phi(x+a)$ for any 4-vector $a^\nu$, the observable should look the
same. Another symmetry is Lorentz invariance: physics should look the same
whether we point our measurement apparatus to the left or to the right, or
put it on a train. The group of translations and Lorentz translations is
called the \textbf{\Poincare{} group}, ISO(1,3) (the isometry group of
Minkowski space).

It is possible for two different groups to have the same algebra. For
example, the proper orthochronous Lorentz group($det(\Lambda)=+1,
\Lambda^0_0>1$) and the full Lorentz group($det(\Lambda)=\pm1,
\Lambda^0_0>1 \text{or} \Lambda^0_0<-1$)
have the same algebra, but the full Lorentz group has in addtion time
reversal and parity. The \textbf{proper} Lorentz group is the special
orthogonal group SO(1,3), which contains only the elements with determinant
1, so it excludes T and P.

Notice that different representations of the same group G can have different
dimensions. \textbf{Lie algebra} relations is the properties of G that are
inherent in all of its representations.
 
 Assume $D_R(g(\theta))$ as a representation of group G, and 
 \[
     D_R(g(\theta\sim0)) = \mathds{1} + i\theta_aT^a_R + \mathcal{O}(\theta^2)
     \]
 where $T^a_R \equiv -i\partial_{\theta_a}D_R(g(\theta))|_{\theta=0}$ 
 is the \textbf{generator} of G in the representation of R. So
 \[
     D_R(g(\theta)) = e^{i\theta_aT^a_R}
     \]
 Given two such elements $D_R(g(\theta_1)) = e^{i\theta^1_aT^a_R}$, 
 $D_R(g(\theta_2)) = e^{i\theta^2_aT^a_R}$, their product must give a third:
 \[
     D_R(g_1)D_R(g_2) = D_R(g_1g_2) = e^{i\theta^3_aT^a_R}
     \]
 Expanding the log of both hand sides to second order in the $\theta$(See
 Maggiore chapter 2.1), we get:
 \[
     \theta^3_a = \theta^1_a + \theta^2_a - \frac{1}{2}f^{bc}{}_a +\mathcal{O}(\theta^3)
     \]
 which implies that:
 \[
     [T^a, T^b] = if^{ab}{}_cT^c
     \]
 This is the \emph{Lie algebra} of G and the \emph{f} is the \emph{structure
 constants} of G. f does not depend on the representation. Note that the
 normalization of the $T^a$ is ambiguous, and rescaling T rescales f. A
 common convention is to choose an orthonormal basis:
 \[
     tr(T^aT^b) = \frac{1}{2}\delta^{ab}
     \]

 The Lie algebra is defined in the neighborhood of the identity element, but
 by conjugating by finite transformations, the tangent space to any point on
 the group has the same structure, so it determines the local structure. 
 It doesn't know about global, discrete issues, like disconnected components, 
 so different groups can have the same Lie algebra.

 A \emph{casimir} of the algebra is an operator made from the generators
 which commutes with all of them.

 The notation of "axis of rotation" is (d=3)-centric. More generally, 
 a rotaion is specified by a (2D) \emph{plane} of rotation. 
 In d=3, we can specify a plane by its normal
 direction, the one that's left out, $J^i \equiv \epsilon^{ijk}J^{jk}$, in
 terms of which the so(3) lie algebra is 
 \[
     [J^{ij}, J^{kl}] = i(\delta^{jk}J^{il}+\delta^{il}J^{jk}-\delta^{ik}J^{jl}-\delta^{jl}J^{ik})
     \]
 The vector representation is 
 \[
     \left(J^{ij}_{(1)}\right)^k{}_l = i(\delta^{ik}\delta^j_l - \delta^{jk}\delta^i_l)
     \]
 The spinor representation is 
 \[
     J^{ij}_{(\frac{1}{2})} = \epsilon^{ijk}\frac{1}{2}\sigma^k =
     \frac{i}{4}[\sigma^i, \sigma^j]
     \]
 For general \emph{d}, we can make a spinor representation of dimension k
 (k=2J+1) if we find d $k\times k$ matrices $\gamma^i$ which satisfy the
 \emph{Clifford algebra} $\{\gamma^i, \gamma^j\} = 2\delta^{ij}$.

 For (faithful) representations of \emph{non-compact} groups, 'unitary' and
 'finite-dimensional' are mutually exclusive.
%%%%%%%%%%%%%%%%%%%%%%%%%%%%%%%%%%%%%%%%%%%%%%%%
\subsection{Lorentz group} 

the symmetry group associated with \emph{special relativity}. \\

For scalar field, the physical content of Lorentz invariance is 
that nature has a symmetry under which scalar fields do not transform. 
Take, for example, the temperature of a fluid, which can vary from 
point to point. If we change reference frames, the labels for the 
points change, but the temperature at each point stays the same.

As for vector field, the difference is that the compnents of a vector field
at the point x transform into each other as well.

The simplest Lorentz-invariant operator that we can write down involving
derivatives is the d'Alembertian:
\[ \Box = \partial_\mu^2 = \partial^2_t - \partial^2_x - \partial^2_y -
\partial^2_z \]

Objects such as $v^2 = V_\mu V^\mu, \phi, 1, \partial_\mu V^\mu$ are
\emph{Lorentz invariant}, meaning they do not depend on the Lorentz frame at
all. While objects like: $V_\mu, F_\mn, \partial_\mu, x_\mu$ are
\emph{Lorentz covariant}, meaning they do change in different frames, but
precisely as the Lorentz transformation dictates.

The Lorentz group is sometimes called O(1,3). This is orghogonal (preserves
a metric) group corresponding to a metric with (1,3) signature.

The irreducible representations of the Lorentz group can be constructed from
irreducible representations of SU(2). 
\[
    J^+_i \equiv \frac{1}{2}(J_i + iK_i), 
    J^-_i \equiv \frac{1}{2}(J_i - iK_i)
\]
which satisfy
\[
    \begin{aligned}[]
	[J^+_i, J^+_j] &= i\epsilon_{ijk}J^+_k    \\
	[J^-_i, J^-_j] &= i\epsilon_{ijk}J^-_k    \\
	[J^+_i, J^-_j] &= 0 
    \end{aligned}
\]

%%%%%%%%%%%%%%%%%%%%%%%%
\subsubsection{Lorentz algebra}
Lorentz algebra, so(1,3)
\[
    \begin{aligned}[]
	[J_i, J_j] &= i\epsilon_{ijk}J_k    \\
	[J_i, K_j] &= i\epsilon_{ijk}K_k    \\
	[K_i, K_j] &= -i\epsilon_{ijk}J_k    \\
    \end{aligned}
\]
Note that $ [J_i, J_j] = i\epsilon_{ijk}J_k $ is also the algebra for
rotations, SO(3), and in fact the $J_i$ generate the 3D rotation subgroup of
the Lorentz group.

Similar to rotation group, we can generalize to other SO(1,d) by collecting
the generators into an antisymmetric matrix $J^{\mu}$ with components
$J^{ij} = \epsilon^{ijk}J^k, J^{0i} = K^I = -J^{i0}$(exactly as $\vec{E},
\vec{B}$ are collected into $F^{\mu}$). This object satisfies:
\[
    J^{\mn}=
    \begin{pmatrix}
	0   & K_1   & K_2   & K_3   \\
	-K_1	& 0 & J_3   & -J_2  \\
	-K_2	& -J_3	& 0 & J_1   \\
	-K_3	& J_2	& -J_1	& 0 \\
    \end{pmatrix}
    \]
\[
    [J^{\mn}, J^{\rho\sigma}]=
    i( \eta^{\nu\rho}J^{\mu\sigma}
    +\eta^{\mu\sigma}J^{\nu\rho} 
    -\eta^{\mu\rho}J^{\nu\sigma}
    -\eta^{\nu\sigma}J^{\mu\rho}
    )
    \]
The fundamental (d+1 Dimentional vector) representation matrices solving
this equation are 
\[
    (J^{\mn})^\rho{}_\sigma = i(\eta^{\nu\rho}\delta^{\mu}_\sigma -
    \eta^{\mu\rho}\delta^{\nu}_{\sigma})
    \]
\[
    [S^{\mn}, S^{\rho\sigma}]=
    i( g^{\nu\rho}S^{\mu\sigma}
    -g^{\mu\rho}S^{\nu\sigma}
    -g^{\nu\sigma}S^{\mu\rho}
    +g^{\mu\sigma}S^{\nu\rho} )
    \]
The Lie algebra is independent of any concrete representation, though we
derive them from 4-vector representation. They holds for any representation.

%%%%%%%%%%%%%%%%%%%%%%%%
\subsubsection{represention}
The Dirac representation of the Lorentz group is reducible; it is the direct
sum of a left-handed and a right-hadned spinor representation.

\subsubsection{$\gamma^5$}
\begin{itemize}
    \item $(\gamma^5)^2 = \mathcal{1}$
    \item $\left\{\gamma^5,\gamma^\mu\right\}=0$
    \item Extended Clifford algebra: $\left\{\gamma^M,\gamma^N\right\}=2g^{MN}$, with $\gamma^M =
	\gamma^0,\gamma^1,\gamma^2,\gamma^3,i\gamma^5$, and $g^{MN}=diag(1, -1, -1, -1, -1)$ 
\end{itemize}

%%%%%%%%%%%%%%%%%%%%%%%%%%%%%%%%%%%%%%%%%%%%%%%%
\subsection{Unitary group} 
the probability should add up to \emph{1}.
\textbf{U(N)} is defined by its N dimensional representation as $N\times N$
complex unitary matrices $\mathds{1} = M^{\dag}M = MM^{\dag}$. This one
doesn't arise as a spacetime symmetry, but is crucial in the study of gauge
theory. 

Unitary representation, which means under the representation, the group
element $\mathcal{P}$ has $\mathcal{P}^\dag\mathcal{P}=1$. But why the
unitarity depends on representation? shouldn't it be representation
independent.

%%%%%%%%%%%%%%%%%%%%%%%%
\subsubsection{Majorana represention}
\begin{equation}
    \gamma^{0} = 
    \begin{pmatrix}
	0   &	\sigma^2    \\
	\sigma^2    & 0     \\
    \end{pmatrix},
    \gamma^{1} = 
    \begin{pmatrix}
	i\sigma^3    & 0     \\
	0   &	i\sigma^3    \\
    \end{pmatrix},
    \gamma^{2} = 
    \begin{pmatrix}
	0   &	-\sigma^2    \\
	\sigma^2    & 0     \\
    \end{pmatrix},
    \gamma^{3} = 
    \begin{pmatrix}
	-i\sigma^1    & 0     \\
	0   &	-i\sigma^1    \\
    \end{pmatrix}
\end{equation}
The Majorana is another $(\frac{1}{2}, 0)\bigoplus (0, \frac{1}{2})$
representation of the Lorentz group that is phhysically equivalent to the
Weyl representation.
%%%%%%%%%%%%%%%%%%%%%%%%%%%%%%%%%%%%%%%%%%%%%%%%
\subsection{\Poincare \ Group} 
The group of translations and Lorentz transformation is called the
\textbf{\Poincare{} group}, 
ISO(1,3) (the isometry group of Minkowski space).

\textbf{Particle} can be defined as a set of states that mix only among
themselves under \Poincare{} transformations.

Particles transform under irreducible unitary representations of the
\Poincare{} group.

There are \emph{no finite-dimensional unitary representations of the
\Poincare{} group}

%%%%%%%%%%%%%%%%%%%%%%%%%%%%%%%%%%%%%%%%%%%%%%%%
\subsection{Gauge symmetry}
Under a gauge transformation, $\phi$ can transform as 
\[
    \phi\rightarrow e^{-i\alpha(x)}\phi
    \]
But the derivatives $|\partial_\mu \phi|^2$ is not invariant. we can make
the kinetic term gauge invariant using something called a covariant
derivative. Let
\[
    A_\mu \rightarrow A_\mu + \frac{1}{e}\partial_\mu \alpha
    \]
So
\[
    (\partial_\mu+ieA_\mu)\phi\rightarrow(\partial_\mu+ieA_\mu+i\partial_\mu\alpha)e^{-i\alpha(x)}\phi=e^{-i\alpha(x)}(\partial_\mu+ieA_\mu)\phi
    \]
This leads to the definition of \emph{covariant derivative}
\[
    D_\mu\phi\equiv(\partial_\mu+ieA_\mu)\phi\rightarrow{e^{-i\alpha(x)}D_\mu\phi}
    \]
%%%%%%%%%%%%%%%%%%%%%%%%%%%%%%%%%%%%%%%%%%%%%%%%%%%%%%%%%%%%%%%%%%%%%%%
\section{Field}
%%%%%%%%%%%%%%%%%%%%%%%%%%%%%%%%%%%%%%%%%%%%%%%%
\subsection{Scalar Field}
\[ \mathcal{L} = \frac{1}{2}\dot{\phi}^{2} -
\frac{1}{2}\vec{\nabla}\phi \cdot \vec{\nabla}\phi - \frac{1}{2}m^{2}\phi^{2} 
= \frac{1}{2}(\partial_{\mu}\phi\partial^{\mu}\phi - m^{2}\phi^{2}) 
=-\frac{1}{2}(\phi\Box\phi-m^2\phi^2)	\]
EOM:
$$ -\partial_{t}^{2}\phi + \nabla^{2}\phi - m^{2}\phi =
(\partial_{\mu}\partial^{\mu} + m^{2}) \phi = 0 $$
$$ \pmb{\phi}(\vec{x}) = \int
\frac{d^{d}k}{(2\pi)^{d}}\sqrt{\frac{\hbar}{2\omega_{k}}}(e^{i\vec{k}\cdot\vec{x}}a_{k}+e^{-i\vec{k}\vec{x}}a_{k}^{\dag})$$
$$ \pmb{\pi}(\vec{x}) = \frac{\partial{\mathcal{L}}}{\partial_{\mu}\phi} =
\frac{1}{i} \int \frac{d^{d}k}{(2\pi)^{d}}\sqrt{\frac{\hbar\omega_{k}}{2}}(e^{i\vec{k}\cdot\vec{x}}a_{k}-e^{-i\vec{k}\vec{x}}a_{k}^{\dag}) $$
$$ [\phi(\vec{x}), \pi(\vec{x'})] = i\hbar\delta^{d}(\vec{x} - \vec{x'})$$
$$ H =\sum_{n}(p_{n}\dot{q}_n) = \int dx(\pi(x)\dot{q}(x) -
\mathcal{L})  $$

%%%%%%%%%%%%%%%%%%%%%%%%%%%%%%%%%%%%%%%%%%%%%%%%
\subsection{Complex Scalar Field}
\[  \mathcal{L}=-\phi^*\Box\phi-m^2\phi^*\phi\]

%%%%%%%%%%%%%%%%%%%%%%%%%%%%%%%%%%%%%%%%%%%%%%%%
\subsection{\KG{} Field}
For a massive scalar (spin 0) and neutral (charge 0) field:
$$\mathcal{L} = \frac{1}{2} [(\partial_{\mu}\phi)(\partial^{\mu}\phi) -
m^{2} \phi^{2}]$$
The Euler-Lagrange formula:
$$ (\Box + m^{2})\phi = 0$$
This is the \KG{} equation. It was quantized by Pauli and Weisskopf in 1934.
The Klein-Gordon equation was historically rejected as a fundamental quantum
equation because it predicted negative probability density.

%%%%%%%%%%%%%%%%%%%%%%%%%%%%%%%%%%%%%%%%%%%%%%%%
\subsection{Dirac Field}
Dirac was looking for an equation linear in E or in $\partial_t$. For a
massive spinor (spin $1/2$) field the Lagrangian density is:
$$ \mathcal{L} = \bar{\psi}(i\gamma^{\mu}\partial_{\mu} - m)\psi; \quad
\bar{\psi} = \psi^{*}\gamma^{0} \quad \text{Dirac adjoint} $$
The $4 \times 4$ Dirac matrices $\gamma^{\mu} (\mu = 0,1,2,3)$ satisfy the
Clifford algebra:
\[ 
    \{\gamma^{\mu},\gamma^{\nu}\}=\gamma^{\mu}\gamma^{\nu}+\gamma^{\nu}\gamma^{\mu}=2g^{\mn} 
\]

Define:
\[
    \sigma^{\mn} \equiv \frac{i}{2}[\gamma^{\mu}, \gamma^{\nu}]
    \]
Any set of four matrices satisfying this equation can be combined to form a
4d representation of so(1,3) in the form
\[
    J^{\mn}_{Dirac} \equiv \frac{i}{4}[\gamma^\mu, \gamma^\nu]
    \]
The corresponding EOM is the Dirac equation:
$$ (i\gamma^{\mu}\partial_{\mu} - m) \psi =0;	\quad
i(\partial_{\mu}\bar{\psi})\gamma^{\mu} + m\bar{\psi} = 0$$
To be explicit, this is shorthand for
\[
    (i\gamma^\mu_{\alpha\beta}\partial_\mu-m\delta_{\alpha\beta})\psi_\beta = 0
    \]
    The Dirac ($\gamma$) matrices:
\[
    \gamma^{\mu} = 
    \begin{pmatrix}
	0   & \sigma^{\mu}  \\
	\bar{\sigma}^{\mu}  & 0	\\
    \end{pmatrix}
\]
\[
    \gamma^5\equiv{i}\gamma^0\gamma^1\gamma^2\gamma^3
    \]
where
\[
    \sigma^\mu\equiv(\mathds{1},\vec{\sigma}),
    \bar{\sigma}^\mu\equiv(\mathds{1},-\vec{\sigma}),
\]
\[ 
    \gamma^{0} = 
    \begin{pmatrix}
	0   & I	\\
	I   & 0	\\	
    \end{pmatrix}, \quad
    \gamma^{i} = 
    \begin{pmatrix}
	0   & \sigma^{i}    \\
	-\sigma^{i} & 0	    \\
    \end{pmatrix}
\]
Where \textbf{$\sigma$} is the Pauli matrices:
\begin{equation}
    \sigma^{1} = 
	\begin{pmatrix}
	    0	& 1 \\
	    1	& 0 
	\end{pmatrix},	\quad
    \sigma^{2} = 
	\begin{pmatrix}
	    0	& -i \\
	    i	& 0 
	\end{pmatrix},	\quad
    \sigma^{3} = 
	\begin{pmatrix}
	    1	& 0 \\
	    0	& -1 
	\end{pmatrix}
    \label{Pauli Matrices}
\end{equation}
\[  \vec{\sigma} = (\sigma_1, \sigma_2, \sigma_3)   \]
\[  \vec{\sigma}^\dag=\vec{\sigma}\]
Commutation relation is
\[  
    \{\sigma^i, \sigma^j\} = 2\delta^{ij}, 
    [\sigma^i, \sigma^j] = 2i\epsilon^{ijk}\sigma^k	\]
In the Weyl basis, the Dirac equation is:
\[
    \begin{pmatrix}
	-m  & i\sigma^\mu\partial_\mu	\\
	i\bar{\sigma}^\mu\partial_\mu	& -m	\\
    \end{pmatrix}
    \begin{pmatrix}
	\psi_L	\\
	\psi_R	\\
    \end{pmatrix} = 0
\]
Quantization of the Dirac field is achieved by replacing the spinors by
field operators and using the Jordand and Wigner quantization rules.
Heisenberg's EOM for the field operator $\hat{\psi}(\vec{x}, t)$
reads:
$$ i\partial_{t}\hat{\psi}(\vec{x}, t) = [
    \hat{\psi}(\vec{x}, t), \hat{H}]$$
There are both positive and negative eigenvalues in the energy spectrum. The
later are problematic in view of Einstein’s energy of a particle at rest 
$ E = mc^2 $. Dirac’s way out of the negative energy catastrophe was to 
postulate a Fermi sea of antiparticles. This genial assumption was not 
taken seriously until the positron was discovered in 1932 by Anderson.

\subsection{Weyl spinor}
\[  \sigma^\mu\partial_\mu\psi=\mathds{1}\partial_t\psi-\partial_i\sigma_i\psi=0    \]
Dirac equation for a Weyl spinor:
\[  \sigma^\mu\partial_\mu\psi=0    \]

%%%%%%%%%%%%%%%%%%%%%%%%%%%%%%%%%%%%%%%%%%%%%%%%
\subsection{Maxwell Field}
\begin{equation}
    \label{Maxwell Eqn}
\end{equation}

\[ 
\begin{aligned}
    \mathcal{L} &= -\frac{1}{4}F^2_{\mn} - j_{\mu}A^{\mu} 
    =-\frac{1}{4}(\partial_{\mu}A_\nu-\partial_{\nu}A_\mu)^2-j_{\mu}A^{\mu}\\
    &=-\frac{1}{2}(\partial_{\mu}A_\nu)^2+\frac{1}{2}(\partial_{\mu}A_\mu)^2-j_{\mu}A^{\mu}\\
    &=(E^2 - B^2)/2 - \phi V + \vec{j}\vec{A} 
\end{aligned}
    \]
Here we use
\[
    (\partial_{\mu}A_\nu)^2=(\partial_{\nu}A_\mu)^2
    \]
where $J_\mu$ is the external current:
\[ J_\mu(x) = \left\{ 
\begin{aligned}
    J_0(x) = \rho(x) \\
    J_i(x) = v_i(x)
\end{aligned}
\right.\]

Field strength tensor 
\[ F^{\mn} = \partial^{\mu}A^{\nu} - \partial^{\nu}A^{\mu} \] 
with components 
\[ F^{0i} = \partial^{0}A^{i} - \partial^{i}A^{0} = -E^{i}, 
\quad
F^{ij} = \partial^{i}A^{j} - \partial^{j}A^{i} = -\varepsilon^{ijk}B^{k} \]
EOM
$\frac{\partial\mathcal{L}}{\partial{A_\nu}}-\partial_\mu\frac{\partial\mathcal{L}}{\partial(\partial_{\mu}A_\nu)}=0$
\[-J_\nu-\partial_\mu(-\partial_{\mu}A_\nu)-\partial_\nu(\partial_{\mu}A_\mu)=0\]
which gives
\[\partial_{\mu}F_{\mn}=J_\nu\]
Lorentz gauge: $\partial_{\mu}A_{\mu}=0$
\[J_{\nu}=\partial_{\mu}(\partial_{\mu}A_\nu-\partial_{\nu}A_{\mu})=\Box{A_\nu}-\partial_{\nu}(\partial_{\mu}A_{\mu})
=\Box{A_{\nu}}\]
so
\[A_{\nu}(x)=\frac{1}{\Box}J_{\nu}(x)\]
\textbf{propagator}
\[\Pi_A=\frac{1}{\Box}\]
Note that the propagator has nothing to do with the source. In fact it is
entirely determined by the kinetic terms for a field.

%%%%%%%%%%%%%%%%%%%%%%%%%%%%%%%%%%%%%%%%%%%%%%%%
\subsection{Proca (Massive Vector Boson) Field}
$$ \mathcal{L} = -\frac{1}{4}F_{\mn}F^{\mn} +
\frac{1}{2}m^{2}A_{\mu}A^{\mu} - j_{\mu}A^{\mu} $$
EOM:
$$ \Box A^{\nu} - \partial^{\mu}(\partial_{\mu}A^{\mu}) + m^{2}A^{\nu} =
j^{\nu}$$

%%%%%%%%%%%%%%%%%%%%%%%%
\subsubsection{spinors}
\begin{description}
    \item[Dirac spinors] have both left- and right-handed compnents. Theny
	can be massive or massless.
    \item[Weyl spinors] are always massless and can be left- or
	right-handed. When embedded in Dirac spinors they satisfy the
	constraint $\gamma_5\psi=\pm\psi$.
    \item[Majorana spinors] are left- or right handed. When embedded in
	Dirac spinors they satisfy the constraint
	$\psi=\psi_C=-i\gamma_2\psi^*$
\end{description}

\section{Interaction}

time ordered amplitude: 
\[
    \begin{aligned}
	D_F(x,y) \equiv \bra{0}T\{\phi(x)\phi(y)\}\ket{0} = \bra{0}\phi(x)\phi(y)\ket{0}\theta(x^0-y^0) + \bra{0}\phi(y)\phi(x)\ket{0}\theta(y^0-x^0)     \\
	(\Box + m^2)D_F(x,y) = -i\hbar\delta^{(4)}(x-y)
    \end{aligned}
    \]

Let $\ket{\Omega}$ being the vacuum state of interacting QFT, ($\ket{\Omega} \neq \ket{0}$), then
\[
    \begin{aligned}
	&\partial_t\bra{\Omega}T\{\phi(x)\phi(x')\}\ket{\Omega}	    \\
	=& \partial_t[\bra{\Omega}\phi(x)\phi(x')\ket{\Omega}\theta(t-t') + \bra{\Omega}\phi(x')\phi(x)\ket{\Omega}\theta(t'-t)]   \\
	=& \bra{\Omega}(\partial_t\phi(x))\phi(x')\ket{\Omega}\theta(t-t') + \bra{\Omega}\phi(x')\partial_t\phi(x)\ket{\Omega}\theta(t'-t)]   \\
	&+ \bra{\Omega}\phi(x)\phi(x')\ket{\Omega}\delta(t-t') + \bra{\Omega}\phi(x')\phi(x)\ket{\Omega}\delta(t'-t)]   \\
	=& \bra{\Omega}T\{(\partial_t\phi(x))\phi(x')\}\ket{\Omega} + \bra{\Omega}[\phi(x), \phi(x')]\ket{\Omega}\delta(t-t')	\\
	=& \bra{\Omega}T\{(\partial_t\phi(x))\phi(x')\}\ket{\Omega} \\
    \end{aligned}
    \]

So, the second derivative:
\[
    \begin{aligned}
	&\partial_t^2\bra{\Omega}T\{\phi(x)\phi(x')\}\ket{\Omega}	    \\
	=& \partial_t\bra{\Omega}T\{(\partial_t\phi(x))\phi(x')\}\ket{\Omega}	    \\
	=& \bra{\Omega}T\{(\partial_t^2\phi(x))\phi(x')\}\ket{\Omega} + \bra{\Omega}[\partial_t\phi(x), \phi(x')]\ket{\Omega}\delta(t-t')	\\
	=& \bra{\Omega}T\{(\partial_t^2\phi(x))\phi(x')\}\ket{\Omega} -i\hbar\delta^{(4)}(t-t') \\
    \end{aligned}
    \]
So for free field ( $(\Box+m^2)\phi(x) = 0$):
\[
    \begin{aligned}
    (\Box+m^2)\bra{\Omega}T\{\phi(x)\phi(x')\}\ket{\Omega} = \bra{\Omega}T\{(\Box+m^2)\phi(x)\quad\phi(x')\}\ket{\Omega} = -i\hbar\delta^{(4)}(x-x')\\
    =& -i\hbar\delta^{(4)}(x-x')
    \end{aligned}
    \]
Define:
\begin{equation}
    \bra{\Omega}T\{\phi(x)\phi(x_2)\cdots\phi(x_n)\}\ket{\Omega} = \langle\phi(x)\phi(x_2)\cdots\phi(x_n)\rangle	\\
\end{equation}
So:
\[
    \Box\langle\phi(x)\phi(x_2)\cdots\phi(x_n)\rangle = \langle(\Box\phi(x))\phi(x_2)\cdots\phi(x_n)\rangle 
    -i\hbar\displaystyle\sum_{j=1}^{n}\delta^{(4)}(x-x_j)\langle\phi(x)\phi(x_2)\cdots\phi(x_{j-1})\phi(x_{j+1})\cdots\phi(x_n)\rangle
    \]
One more definition:
\begin{equation}
    \phi_x \equiv \phi(x), \qquad \phi_j \equiv \phi(x_j)
\end{equation}
So, in general:
\[
    \begin{aligned}
	&\Box\langle\phi_x\phi_2\cdots\phi_n\rangle = \langle(\Box\phi_x)\phi_2\cdots\phi_n\rangle + ...
	&= \langle(\frac{\partial\mathcal{L}_{int}}{\partial\phi_x}phi_2\cdots\phi_n\rangle - i\hbar\displaystyle\sum_{j=1}^{n}\delta^{(4)}(x-x_j)\langle\phi_x\phi_2\cdots\phi_{j-1}\phi_{j+1}\cdots\phi_n\rangle
    \end{aligned}
    \]

For interacting QFT: ($(\Box+m^2)\phi = \frac{\partial\mathcal{L}_int}{\partial\phi}$)


Define shorthands:
\begin{equation}
    \delta_{xi} \equiv \delta(x-x_i), \quad D_{xi} \equiv \langle\phi_x\phi_i\rangle
\end{equation}

So for free massless scalar:
\[
    \Box_xD_{x1} = -i\delta_{x1}
    \]

So:
\begin{equation}
    \begin{aligned}
	D_{12} = \langle\phi_1\phi_2\rangle &= \int d^4x\delta_{x1}\langle\phi_x\phi_x\rangle	\\
	&= i\int d^4x\Box_xD_{x1}\langle\phi_x\phi_2\rangle \\
	\text{IBP twice} &= i\int d^4xD_{x1}\Box_x\langle\phi_x\phi_2\rangle \\
	&= \int d^4xD_{x1}\delta_{x2} = D_{21}	\\
    \end{aligned}
\end{equation}

and 

\[
    \begin{aligned}
	\langle\phi_1\phi_2\phi_3\phi_4 &= \int d^4x\delta_{x1}\langle\phi_x\phi_2\phi_3\phi_4\rangle 
	= i\int d^4xD_{x1}\Box_x\langle\phi_x\phi_2\phi_3\phi_4\rangle	\\
	&= i\int d^4xD_{x1}(-i)(\delta_{x2}D_{34}+\delta_{x3}D_{24}+\delta_{x4}D_{23}) \\
	&= D_{21}D_{34} + D_{31}D_{24} + D_{41}D_{23}	\\
	&= D_{12}D_{34} + D_{13}D_{24} + D_{14}D_{23}	\\
    \end{aligned}
    \]
% Insert Feymann Diagram of the above expressoin.

Perturbation Theory in coupling constant $g<<1$: truncate at $O(g^2)$

Feymann Rules:
\begin{enumerate}
    \item For all external points, draw a fixed point with a line going out from it.
    \item 
	\begin{enumerate}
	    \item contract lines together -- Feymann Propagator
	    \item line-split: an interaction-vertex $\prop i\mathcal{L}'_{int}(\phi)$
	\end{enumerate}
    \item for given order in P.T. in coupling constant, sum of all diagrams with all lines contracted 
	and intergragte over position of all vertices.
\end{enumerate}

%%%%%%%%%%%%%%%%%%%%%%%%%%%%%%%%%%%%%%%%%%%%%%%%%%%%%%%%%%%%%%%%%%%%%%%%
\section{Terminology}
\begin{description}
    \item [free spinors]
    \item [Dirac spinors]
    \item [Weyl representation]
    \item [spinor representation]
    \item [$(\frac{1}{2}, \frac{1}{2})$ representation]
    \item [projective representation]
    \item [Lorentz generators]
	\[
	    S^{\mn} = \frac{i}{4}[\gamma^\mu, \gamma^\nu]
	\]
\end{description}


Lorentz group O(1,3); tensor representations $\phi,A_\mu, h_{\mn}$, spinor
represontations: Weyl spinors $\phi_L, \phi_R$. A Dirac spinor $\psi$
transforms in the reducible $(\frac{1}{x},0)\oplus(0, \frac{1}{2})$
representaion.
The next step towards quantizing a thory with spinors is to use these
Lorentz group representaions to generate irreducible unitray representaions
of the \Poincare{} group.



\chapter{Examples}

\section{Dimensional analysis}

\subsection{Black body radiation}
\[ I(\omega) \equiv \frac{1}{V}\frac{d}{d\omega}E(\omega)\]
This has units of $[Energy] \times [time] \times [distance]^{-3}$ that can
be constructed out of $\omega, k_{B}T$ and c. So easily:
\[ I(\omega) = const \times c^{-3}w^{2}k_{B}T \]

\section{First Principle}
\subsection{Special relativity field}
\[ \text{special relativity scalar field}\phi \xrightarrow{simplest possible
\\ Lorentz invariant} \Box\phi = 0 \]
One solution is: 
\[ \phi(x) = a_p(t)e^{i\vec{p}\cdot\vec{x}}\]
So we get:
\[ (\partial^2_t + \vec{p}\cdot\vec{p})a_p(t) = 0 \]
This is exactly the equation of motion of harmonic oscillator.
General solution:
\[ \phi(x,t) = \int \frac{d^{3}p}{(2\pi)^3} [a_p(t)e^{i\vec{p}\cdot\vec{x}}
+ a_p^*(t)e^{-i\vec{p}\cdot\vec{x}}]\]


% part 4: application
\part{Application}
%%%%%%%%%%%%%%%%%%%%%%%%%%%%%%%%%%%%%%%%%%%%%%%%%%%%%%%%%%%%%%%%%%%%%%%%
\section{Application}

%%%%%%%%%%%%%%%%%%%%%%%%%%%%%%%%%%%%%%%%%%%%%%%%
\subsection{time}
Definition of time:
\begin{itemize}
    \item sun
    \item pendulum --1929
    \item quartz -- 1967
    \item atomic beam resonance -- 1967
    \item atomic fountain -- 1989
    \item optical clock -- 2020 ?
\end{itemize}


% part 5: timeline
\part{Timeline of Important Physics}
\chapter{Timeline of Import Physics}

\section{HEP}
\begin{itemize}
    \item 1896, Zeeman effect: \textbf{spin}
    \item 1897, J.J. Thomson, \textbf{electron} from cathode rays
    \item 1917 (reported in 1919 and 1925), Rutherford proved that the hydrogen 
	nucleus is present in other nuclei, a result usually regarded as the discovery
	of \textbf{proton}
    \item 1922, Otto Stern and Walther Gerlach, seperation (\textbf{spin}) of silver atoms 
	(a single 5s electron) in a non-uniform magnetic field
    \item 1925, George Uhlenbeck and Samuel Goudsmit propose \textbf{electron spin} 
    \item 1928, \textbf{Dirac equation}; Quantum magnetism
    \item 1932, \textbf{Isospin}
    \item 1932, Chadwick, \textbf{neutron}
    \item 1932, Anderson, \textbf{positron} from cosmic rays
    \item 1933, Otto Stern, \textbf{proton magnetic moment}
    \item 1936, Anderson and Neddermeyer, \textbf{muon} from cosmic radiation
    \item 1946, Nuclear Magnetic Resonance (NMR)
    \item 1947, $\pi^\pm$
    \item 1947, \textbf{kaon} from cosmic rays
    \item 1950, $\pi^0$
    \item 1943, proton anomalous magnetic moment
    \item 1955, proton has a finite size \cite{PhysRev.98.217} -- elastic ep 
	scattering deviates from $1/\sin^4(\theta/2)$
    \item 1957, Pontecorvo suggested neutrino mixing (observed in 1998)
    \item 1961, Glashow proposed his model for electroweak unification
    \item 1961, first proton FF measurement
    \item 1968, Bjorken scaling \cite{PhysRev.179.1547}
    \item 1972, experimental evidence of parton \cite{PhysRevD.5.528}
    \item 1973, Neutral weak interaction from Gargamelle \cite{HASERT19741}
    \item 1974 (Nov), \textbf{c quark} ($J/\psi$ particle) was reported by 
	Samuel Ting's group at BNL (fixed target, $\sqrt{s} = 8\ GeV$, 
	observation: $J/\psi \rightarrow \mu^+\mu^-$) 
	and Burton Richter's group at SLAC ($e^+e^-$ ring, $\sqrt{s} = 3.1\ GeV$, 
	observation: $e^+e^- \rightarrow J/\psi$)
    \item 1975, $\tau$ lepton at Mark I (SPEAR/SLAC, $\sqrt{s} = 8 \ GeV$, 
	$e^+e^- \rightarrow \tau^+\tau^-$) \cite{PhysRevLett.35.1489}.
    \item 1975, partons (quarks) are spin-1/2 \cite{PhysRevLett.35.1609}
    \item Scaling violation $\Rightarrow$ structure function is NOT a function 
	of x only, it also depends on $Q^2$
	Quarks are radiating energy (probability increases with $Q^2$). What
	are they radiating -- gluons
    \item 1977, \textbf{b quark} from $Y(1S) = \ket{b\bar{b}}$ by Leon Lederman, 
	at FNAL using fixed target exp.
	$\sqrt{s} = 25 \ GeV$, $\epsilon \rightarrow \mu^+\mu^-$
    \item 1979, \textbf{gluon} was found at PETRA/DESY ($e^+e^-$ ring) with 
	$e^+e^- \rightarrow qqg$ (3-jet events), $\sqrt{s} = 30\ GeV$
    \item 1983, \textbf{W, Z} bosons, SPS/CERN, $\sqrt{s} = 900\ GeV$, $W \rightarrow l\bar{\nu}$,
	$Z \rightarrow l^+l^-$
    \item 1985, Flux-tube model for hadrons in QCD \cite{PhysRevD.31.2910}
    \item 1988, Giant magnetoresistance
    \item 1989, \textbf{3 neutrino generation}, LEP/CERN, $\sqrt{s} = 91\ GeV$, 
	observation: Z-boson lineshape measurement
    \item 1995, \textbf{top quark} at Tevatron/FNAL (D0 and CDF), $\sqrt{s} = 1960\ GeV$, 
	observation: 2 semileptonic t-quark decays
    \item 1998, neutrino mixing was first observed by Kamiokande
    \item 2012, \textbf{Higgs boson}, LHC/CERN (ATLAS and CMS), $\sqrt{s} = 8\ TeV$,
	$H\rightarrow \gamma\gamma$, $H\rightarrow Z^*Z \rightarrow 4l$ \cite{Aad_2012}\cite{201230}
\end{itemize}



\end{document}
