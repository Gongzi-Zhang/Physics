\chapter{Electramegnetic Mechanics}

%%%%%%%%%%%%%%%%%%%%%%%%%%%%%% Formulas  %%%%%%%%%%%%%%%%%%%%%%%%%%%%%%
\section{Formulas}
\[
    \vec{A}_{tr} = \vec{n}\times(\vec{n}\times\vec{A}) = \vec{A}-(\hat{n}\vec{A})\hat{n} = (\mathcal{1}-n\cdot n^T)\vec{A}
\]
let n being the column vector 
$n=\begin{pmatrix}
    x_1	\\
    x_2	\\
    x_3	\\
\end{pmatrix}$

%%%%%%%%%%%%%%%%%%%%%%%%%%%%%% Tensor  %%%%%%%%%%%%%%%%%%%%%%%%%%%%%%
\section{Tensor}
the stress tensor $T^{ij}$ is the force per area. The force per volumn $f^j$
i: the minus divergence of the stress tensor
\[ f^j = -\partial_iT^{ij} \]
Conservation law:
\[  \partial_tg^j+\partial_iT^{ij} = 0 \]
where $g^j$ is the momentum per volume. So $f^j=\partial_tg^j$, the force is
the time derivative of the momentum.
In electrostatic:
\[  f^j = \rho E^j \]
So:
\[  \rho E^j = \partial_iT_E^{ij} \]
To get 
\[  T_E^{ij} \equiv -E^iE^j + \frac{1}{2}E^2\delta^{ij} \]
The force:
\begin{equation*}
\begin{aligned}  
    F^j &= \int_v d^3\vec{r}\rho(\vec{r})E^j = \int_v d^3\vec{\nabla}\cdot\vec{E}E^j	\\
	&= \int_v d^3(\nabla_iE^i)E^j \\
	&= \int_v d^3 [\nabla_i(E^iE^j) - E^i\nabla_iE^j ] \\
	&= \int_v d^3 [\nabla_i(E^iE^j) - E^i\nabla_jE^i ] \\
	&= \int_v d^3 [\nabla_i(E^iE^j) - \frac{1}{2}\nabla_jE^2 ] \\
	&= \int_v d^3 [\nabla_i(E^iE^j) - \frac{1}{2}\nabla_iE^2\delta^{ij} ] \\
	&= \int_v d^3 \nabla_i [(E^iE^j) - \frac{1}{2}E^2\delta^{ij} ] \\
	&= \int_S dS n_i T_E^{ij} \\
	&= -\int_S dS n_i T_E^{ij} 
\end{aligned}  
\end{equation*}

For magnetism:
\[  T_B^{ij} = -\frac{1}{\mu}B^iB^j + \frac{1}{2\mu}B^2\delta^{ij} \]

So
\[
    \begin{aligned}
	f^j_{em} &= -\partial_iT_E^{ij} - \partial_iT_B^{ij} - \frac{1}{c}(\partial_tD\times B)^j
    &= (\nabla\cdot D)E^j + ((\nabla\times\vec{H})\times\vec{B})^j - al_tD\times B)^j 
    \end{aligned}
\]
\section{Maxwell Equation}
\begin{equation}
    \label{eqn:Maxwell}
    \left\{
    \begin{aligned}
	\nabla\cdot{E} &= \frac{\rho}{\epsilon}	\\
	\nabla\times{B} &= \frac{1}{c}\left(J + J_D + J_{ind}\right)	\\
	\nabla\cdot{B} &= 0  \\
	\nabla\times{E} &= -\frac{1}{c}\partial_{t}B	\\
    \end{aligned}
    \right.
    \longrightarrow
    \left\{
    \begin{aligned}
	\nabla\cdot{D} &= \rho	\\
	\nabla\times{H} &= \frac{1}{c}J +\frac{1}{c}\partial_{t}D    \\
	\nabla\cdot{B} &= 0  \\
	\nabla\times{E} &= -\frac{1}{c}\partial_{t}B	\\
    \end{aligned}
    \right.
\end{equation}

Symmetry of Maxwell Eqn:
\begin{equation}
    \label{eqn:symmetric_Maxwell}
    \left\{
    \begin{aligned}
	\nabla\cdot\vec{E} &= \frac{\rho}{\epsilon}	\\
	\nabla\times{B} - \frac{1}{c}\partial_t\vec{E}J &= \frac{\vec{J}}{c} 	\\
	\nabla\cdot\vec{B} &= 0  \\
	\nabla\times\vec{E} -\frac{1}{c}\partial_{t}\vec{B} &= 0       \\
    \end{aligned}
    \right.
\end{equation}

For $\rho = \vec{j} = 0$, then there is a symmetry between $\vec{E} \rightarrow \vec{B}$ 
and $\vec{B} \rightarrow -\vec{E}$.
\[
    \begin{aligned}
	\nabla\cdot\vec{E} &= 0	\\
	-\frac{1}{c}\partial_t\vec{E} + \nabla\times\vec{B} &= 0
    \end{aligned}
    \Longleftrightarrow
    \begin{aligned}
	\nabla\cdot\vec{B} &= 0	\\
	-\frac{1}{c}\partial_t\vec{B} - \nabla\times\vec{E} &= 0
    \end{aligned}
\]
Though we derive the $\vec{E}$,$\vec{B}$ symmetry from free conditions ($\rho = \vec{j} = 0$),
it \textbf{looks like ???} we can apply them in any conditions.

where 
\[J_D = \epsilon\partial_{t}E; \quad J_{induced} = c\chi_m^B(\nabla\times{B})\]
Note that the Maxwell equation is relationship about $\mathbf{E}$ and
$\mathbf{B}$, the $\vec{D}$ and $\vec{H}$ is introduced to help reduce the
equations to a more concised form. 

To solve it, using iterative method (in power of $1/c$)
\begin{enumerate}[label=\roman*)]
\item 0th order:
\[
    \begin{aligned}
	\nabla\cdot{E^{(0)}} &= \rho	\qquad	&\nabla\times{B^{(0)}} &= 0 \\
	\nabla\times{E^{(0)}} &= 0	\qquad	&\nabla\cdot{B^{(0)}}  &= 0 \\
    \end{aligned}
\]
\item 1th order:
\[
    \begin{aligned}
	\nabla\cdot{E^{(1)}} &= 0	\qquad	&\nabla\times{B^{(1)}} &= \frac{j}{c} + \frac{1}{c}\partial_{t}E^{(0)}	\\
	\nabla\times{E^{(1)}} &= 0	\qquad	&\nabla\cdot{B^{(1)}}  &= 0 \\
    \end{aligned}
\]
\item 2th order:
\[
    \begin{aligned}
	\nabla\cdot{E^{(2)}} &= 0	\qquad	&\nabla\times{B^{(2)}} &= 0 \\
	\nabla\times{E^{(2)}} &= -\frac{1}{c}\partial_{t}B^{(1)}
	\qquad	&\nabla\cdot{B^{(2)}}  &= 0 \\
    \end{aligned}
\]
\item 3th order \dots
\[
    \begin{aligned}
	\nabla\cdot{E^{(3)}} &= 0	\qquad	&\nabla\times{B^{(3)}} &= \frac{1}{c}\partial_{t}E^{(3)} \\
	\nabla\times{E^{(3)}} &= -\frac{1}{c}\partial_{t}B^{(1)}
	\qquad	&\nabla\cdot{B^{(3)}}  &= 0 \\
    \end{aligned}
\]
\end{enumerate}
So in the quasi-static approximation, we get 
\[
    \begin{aligned}
    E = E^{(0)} + E^{(2)} + \dots   \\
    B = B^{(1)} + B^{(3)} + \dots   \\
    \end{aligned}
    \]

\subsection{Boundary conditions}
\begin{equation}
    \left\{
	\begin{aligned}
	    \vec{e}_n \cdot (\vec{D}_2 - \vec{D}_1) &= \sigma    \\
	    \vec{e}_n \times (\vec{H}_2 - \vec{H}_1) &= \frac{\vec{\alpha}}{c}    \\
	    \vec{e}_n \cdot (\vec{B}_2 - \vec{B}_1) &= 0	\\
	    \vec{e}_n \times (\vec{E}_2 - \vec{E}_1) &= 0    \\
	\end{aligned}
	\right.
\end{equation}
Note that here $\sigma$ and $\vec{\alpha}$ are \textbf{free} charge and
\textbf{free} current. The induced charges and current make trivial
contribution due to infinitesimal integral volume (or area).
\subsection{Plain wave}
If $J = \rho = 0$, then
\[
    \begin{aligned}
    \nabla\times(\nabla\times{E}) &= \nabla(\nabla\cdot{E}) - \nabla^{2}E \\
	&=-\frac{1}{c}\partial_{t}(\nabla\times{B})
	=-\frac{1}{c^2}\cdot\epsilon\mu\cdot\partial^2_{t}E \\
    \end{aligned}
    \]
let 
\[ n^2=\epsilon\mu \]
So
\[
    \nabla^2{E}-\frac{n^2}{c^2}\partial^2_{t}E = 0 
    \]
Because $E = E_0e^{-i\omega{t}}$, so we get
\[
    \nabla^2{E}+\frac{n^2}{c^2}\omega^2E = 0 
    \]
\[
    E = E_0e^{i(kx-\omega{t})}, \quad k^2 = n^2\omega^2/c^2
    \]
let $B = B_0e^{i(kx-\omega{t})}$
\[
    \nabla\times{E} = -\frac{1}{c}\partial_{t}B = \frac{i\omega}{c}B
    \]
\[
    \vec{B}=\frac{c}{i\omega}\nabla\times{E}=\frac{c}{\omega}\vec{k}\times{\vec{E}}
    \]

\subsection{Metal}
\[
    \begin{aligned}
    \nabla\times{H}&=\frac{1}{c}(J+\partial_{t}D)   \\
    &=\frac{1}{c}(\sigma{E}-i\omega\epsilon{E})	\\
    &=\frac{\sigma-i\omega\epsilon}{c}E
    \end{aligned}
\]
So 
\[
    \nabla\times(\nabla\times{E})=-\nabla^2{E}=-\frac{1}{c}\nabla\times\partial_{t}B
    =-\frac{1}{c}\partial_{t}(\nabla\times{{\mu}H})
    =-\mu\frac{\sigma-i\omega\epsilon}{c^2}(-i\omega)E
    =k^2E
    \]
where,
\[
    k^2=\frac{\omega^2\mu(\epsilon+i\sigma/\omega)}{c^2}=\frac{\omega^2\mu\hat{\epsilon}}{c^2}
    \]
Helmholtz eqn:
\[
    (\nabla^2+\frac{\omega^2\mu\hat{\epsilon}}{c^2})E=0
    \]



\section{EM in relativity}

\begin{equation}
    F^{\mn} \equiv 
	\begin{pmatrix}
	    0	& E^x	& E^y	& E^z	\\
	    -E^x    & 0	& -B^z	& B^y	\\
	    -E^y    & B^z   & 0	& -B^x	\\
	    -E^z    & -B^y  & B^x   & 0	\\
	\end{pmatrix}
	= \partial^\mu A^\nu - \partial^\nu A^\mu
\end{equation}
Dual basis $F_{\mn}$, when lowering or raising the 0 index, change the sign; while
for other indecies, nothing happen:
\[
    \begin{aligned}
    F^{0i} = -F^{i0} = {F^0}_i = -{F_0}^i = -F_{0i} =  F_{i0} = -{F_i}^0 = {F^i}_0   \\
    F^{ij} = -F^{ji} = {F^i}_j =  {F_i}^j =  F_{ij} = -F_{ji} = -{F_j}^i = -{F^j}^i \\
    \end{aligned}
\]
So:
\begin{equation}
    F_{\mn} \equiv 
	\begin{pmatrix}
	    0	& -E_x	& -E_y	& -E_z	\\
	    E_x    & 0	& -B_z	& B_y	\\
	    E_y    & B_z   & 0	& -B_x	\\
	    E_z    & -B_y  & B_x   & 0	\\
	\end{pmatrix}
\end{equation}
\[
    F_{\mn}F^{\mn} = 2(\vec{B}^2 - \vec{E}^2)
\]

The Maxwell Eqn:
\begin{gather}
    \partial_\mu F^{\mn} = \Box A^\nu - \partial_\nu(\partial_mu A^\mu) =
    \Box A^\mu = 0, \notag\\
    \partial_\mu F^{\nu\rho} + \partial_\nu F^{\rho\mu} + \partial_\rho F^{\mn} = 0
\end{gather}

Define:
\begin{equation}
    \tilde{F}^{\mn} \equiv 
	\begin{pmatrix}
	    0	& B^x	& B^y	& B^z	\\
	    -B^x    & 0	& -E^z	& E^y	\\
	    -B^y    & E^z   & 0	& -E^x	\\
	    -B^z    & -E^y  & E^x   & 0	\\
	\end{pmatrix} 
	= \frac{1}{2}\epsilon^{\mn\ab}F_{\ab}
\end{equation}
where:
\[
    \epsilon^{\mn\ab} = 
    \lefg\{ 
	\begin{aligned}
	    +1	\quad	even permutation of 0, 1, 2, 3	\\
	    -1	\quad	odd permutation of 0, 1, 2, 3	\\
	    0	\quad	otherwise
	\end{aligned}
	\.
\]

\[
    F_{\mn}\tilde{F}^{\mn} = 2\vec{B}\cdot\vec{E}
\]

Define: 
\begin{equation}
    \begin{aligned}
    T_{[\mu, \nu]}  \equiv \frac{1}{2} (T_\mn - T_\nm)	\\
    T_{[\mu_1, \mu_2, \mu_3]} \equiv \frac{1}{3!} [(T_{\mu_1\mu_2\mu_3} - T_{\mn_1\mu_3\mu_2})
    - (T_{\mu_2\mu_1\mu_3} - T_{\mu_2\mu_3\mu_1})
    + (T_{\mu_3\mu_1\mu_2} - T_{\mu_3\mu_2\mu_1})]
    \end{aligned}
\end{equation}

Then the complete Maxwell Eqns:
\begin{equation}
    \begin{aligned}
	\partial_\mu F^{\mn} = A^{\nu}    \\
    \partial_\nu\tilde{F}^{\mn} = \frac{1}{2}\epsilon^{\mn\ab}\partial_{[\nu}F_{\ab]} = 0
    \end{aligned}
\end{equation}

\subsubsection{Transformation of $\vec{E}$ and $\vec{B}$}
in a boost along x:
\begin{equation}
    \begin{aligned}
	E'_{//} = E_{//}    \\
	E'_{\perp} = \gamma{E_{\perp} + \vec{\beta}\times\vec{B}}
    \end{aligned}
    \quad
    \begin{aligned}
	B'_{//} = B_{//}    \\
	B'_{\perp} = \gamma{B_{\perp} - \vec{\beta}\times\vec{E}}
    \end{aligned}
\end{equation}
