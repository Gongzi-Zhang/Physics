\chapter{Introduction}
% \section{Introduction}

EOM: one can derive it from action (the smallest action principle). As for the action, it should obey some symmetric rules, for example, in relativity, the action should be \emph{Lorentz invariant} and \emph{Gauge invariant}. 

In terms of $\dot{q} = \frac{\partial{q(t)}}{\partial t}$, one can also write $q(t) = \int \dot{q}dt$, so $\dot{q}$ and $q$ are equivalent, not one base on another, both can be independent.

The tools of QM: Perturbation theory (P.T. both time-dependent and time-independent)

Two keys objects in QM: operators and wavefunctions, if you keep operators invariant and wavefunction change with time, you get \emph{Schordinger Picture}; on the other hand, if you keep wavefunction invariant and change operators with time, you get \emph{Heisenburg Picture}. No matter in which picture, both operators and wavefunctions dependents on \emph{representations}, usual representations include r rep., k rep., Bloch state rep. etc.

QFT is just QM with an infinite number of Harmonic Oscillator (H.O.)

In QM (QFT), the basic idea behind renormalization -- infinities can appear in intermediate calculations, but they must drop out of physical observables.

