\chapter{Introduction}
% \section{Introduction}

\begin{itemize}
    \item EOM: one can derive it from action (the smallest action principle).
	As for the action, it should obey some symmetric rules, for example,
	in relativity, the action should be \emph{Lorentz invariant} and \emph{Gauge invariant}. 
    \item The tools of QM: Perturbation theory (P.T. both time-dependent and time-independent)
    \item Two keys objects in QM: operators and wavefunctions, if you keep 
	operators invariant and wavefunction change with time, you get 
	\emph{Schordinger Picture}; on the other hand, if you keep wavefunction 
	invariant and change operators with time, you get \emph{Heisenburg Picture}.
	No matter in which picture, both operators and wavefunctions dependents on 
	\emph{representations}, usual representations include r rep., k rep.,
	Bloch state rep. etc.
    \item QFT is just QM with an infinite number of Harmonic Oscillator (H.O.)
    \item In QM (QFT), the basic idea behind renormalization -- infinities can 
	appear in intermediate calculations, but they must drop out of physical 
	observables.
\end{itemize}
