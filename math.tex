\chapter{Math}

%%%%%%%%%%%%%%%%%%%%%%%%%%%%%%%%%%%%%%%%%%%%%%%%%%%%%%%%%%%%%%%%%%%%%%%%
\section{Definition}
%%%%%%%%%%%%%%%%%%%%%%%%
\subsubsection{Levi-Civita}
$\epsilon_{ijk}$ is an antisymmetric tensor, which is defined as: 
\[  \epsilon_{123} = 1, \epsilon_{ijj} = 0, \epsilon_{ijk} = -\epsilon_{jik} \]


%%%%%%%%%%%%%%%%%%%%%%%%%%%%%%%%%%%%%%%%%%%%%%%%%%%%%%%%%%%%%%%%%%%%%%%%
\section{Formulas}

%%%%%%%%%%%%%%%%%%%%%%%%%%%%%%%%%%%%%%%%%%%%%%%%
\subsection{Fourier transform}
\[
    \delta^3(\vec{x})=\int\frac{d^3k}{(2\pi)^3}e^{i\vec{k}\vec{x}}
    \]

%%%%%%%%%%%%%%%%%%%%%%%%%%%%%%%%%%%%%%%%%%%%%%%%
\subsection{Integral}
\[
    I=\int_{-\infty}^{\infty}dk\frac{e^{ikr}-e^{-ikr}}{k}
    \]
Note that the integrand does not blow up as $k \rightarrow 0$, so
\[
    I=\displaystyle\lim_{\delta\rightarrow{0}}\left[\int_{-\infty}^{\infty}dk\frac{e^{ikr}-e^{-ikr}}{k+i\delta}\right]
    \]
So we have pole at $k = -i\delta$. For $e^{ikr}$ we must close the contour
up to get exponential decay at large $\mathit{k}$. This misses the pole, so
this term gives zero. So
\[
    I=\int_{-\infty}^{\infty}dk\frac{-e^{-ikr}}{k+i\delta}=-(2\pi{i})(-e^{-\delta{r}})=2\pi{i}e^{-i\delta{r}}
    \]

%%%%%%%%%%%%%%%%%%%%%%%%%%%%%%%%%%%%%%%%%%%%%%%%%%%%%%%%%%%%%%%%%%%%%%%%
\section{Functions}

%%%%%%%%%%%%%%%%%%%%%%%%%%%%%%%%%%%%%%%%%%%%%%%%
\subsection{Interesting functions}
%%%%%%%%%%%%%%%%%%%%%%%%
\subsubsection{Weierstrass function}
\begin{equation}
    f(x) = \sum_{n=0}^{\infty}a^{n}cos(b^{n}\pi x)
\end{equation}
Continuous but not derivative everywhree.


%%%%%%%%%%%%%%%%%%%%%%%%%%%%%%%%%%%%%%%%%%%%%%%%
\subsection{Frequently used functions}

%%%%%%%%%%%%%%%%%%%%%%%%
\subsubsection{Legendre Polynomials and spherical harmonics}
\begin{equation}
    Y^m_l(\theta,\phi)=Ne^{im\phi}P^m_l(\cos\theta)
\end{equation}
which satisfies
\[
    r^2\nabla^2Y^m_l(\theta,\phi)=-l(l+1)Y^m_l(\theta,\phi)
    \]

Standard convention :
\[
    P^{-m}_l(\cos\theta)=(-1)^m\frac{(l-m)!}{(l+m)!}P^m_l(\cos\theta),
\]
In QM:
\[
    Y^m_l(\theta,\phi)=(-1)^msqrt{\frac{2l+1}{4\pi}\frac{(l-m)!}{(l+m)!}}P_{lm}(\cos\theta)e^{im\phi}
\]
Where $P_{lm}$ are associated Legendre polynomials without the
Condon-Shortley phase.

Though of different definition, all of them satisfy
\[ 
    Y_l^{m*}(\theta,\phi)=(-1)^{m}Y_{l}^{-m}(\theta,\phi) 
    \]


Associate Legendre Polynomials:

%%%%%%%%%%%%%%%%%%%%%%%%
\subsection{Bessel Function}
%%%%%%%%%%%%%%%%%%%%%%%%%%%%%%%%%%%%%%%%%%%%%%%%
\subsection{Differentive Equations}

%%%%%%%%%%%%%%%%%%%%%%%%
\subsubsection{First Order Equation}
\[ 
\frac{dV}{dt} + \eta(t)V(t) = f(t)  
\]
so:
\[
    \frac{1}{I}\frac{d(I(t)V(t))}{dt} = \frac{dV(t)}{dt} + \frac{I'(t)}{I(t)}V(t)
    \]
if: 
\[ 
    \frac{I'(t)}{I(t)} = \eta(t)	\rightarrow I(t) = exp(\int^{t}dt'\eta(t'))
    \]
so:
\[ \frac{d(I(t)V(t))}{I dt} = f(t)\]

%%%%%%%%%%%%%%%%%%%%%%%%
\subsubsection{Floquet theory}
\begin{equation}
    \ddot{x} + \omega_{0}^{2}(1+\mu cos(\nu t))x = 0
\end{equation}
Analogy to QM(Schrodinger Equation):	
\[
    -\frac{\hbar^{2}}{2m}{\psi''} + U(x)\psi = E\psi
    \]
where $U(x) = -U_{0}cos\frac{2\pi x}{a}$.
So let 
\[
    k^{2} = \frac{2mE}{\hbar^{2}}, u=\frac{U_0}{E}, \nu=\frac{2\pi}{a}
    \]
$u$ is small, we will get   
\[ 
    \psi''+k^{2}\psi = -k^{2}ucos(\nu x)\psi
    \]

%%%%%%%%%%%%%%%%%%%%%%%%%%%%%%%%%%%%%%%%%%%%%%%%
\subsection{Integrated functions}
%%%%%%%%%%%%%%%%%%%%%%%%
\subsubsection{$\Gamma$ function}
\[\Gamma(z) \equiv \int_{0}^{+\infty}dt e^{-t} t^{z-1} =
2\int_{0}^{+\infty}dx e^{-x^2}x^{2z-1}\]

\[\Gamma(z)\Gamma(1-z) = \frac{\pi}{sin\pi z}\]

%%%%%%%%%%%%%%%%%%%%%%%%%%%%%%%%%%%%%%%%%%%%%%%%%%%%%%%%%%%%%%%%%%%%%%%%
\section{Group}

\subsection{Representation of Group}
A \textbf{representation} of a group G on a vector space V over a field K is
a group homomorphism from G to GL(V), the 
