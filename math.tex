\chapter{Math}

%%%%%%%%%%%%%%%%%%%%%%%%%%%%%%%%%%%%%%%%%%%%%%%%%%%%%%%%%%%%%%%%%%%%%%%%
\section{Useful Equations}
\begin{itemize}
    \item Associated Laguerre polynomials:
	$$ L_n{}^m(x) = (-1)^m\frac{d^m}{dx^m}L_{n+m}(x) \qquad \text{for } x \ge 0$$
    \item Assocated Legendre polynomials:
	$$ P_l{}^m(x) = (1-x^2)^{m/2}\frac{d^m}{dx^m}P_{l}(x) \qquad \text{for } x \ge 0$$
    \item Beta function
	$$ B(x,y) = \frac{\Gamma(x)\Gamma(y)}{\Gamma(x+y)} $$
    \item Complete elliptic integral of the first kind:
	$$ K(k) = F(k,\frac{\pi}{2}) = \int_0^{\frac{\pi}{2}}\frac{d\theta}{\sqrt{1-k^2\sin^2\theta}}$$
    \item Complete elliptic integral of the second kind:
	$$ K(k,\frac{\pi}{2}) = \int_0^{\frac{\pi}{2}}\sqrt{1-k^2\sin^2\theta}\ d\theta$$
    \item Complete elliptic integral of the third kind:
	$$ \Pi(\nu,k,\frac{\pi}{2}) = \int_0^{\frac{\pi}{2}}\frac{d\theta}{(1-\nu\sin^2\theta)\sqrt{1-k^2\sin^2\theta}}$$
    \item Confluent hypergeometric function:
	$$ F(a,c,x)= \frac{\Gamma(c)}{\Gamma(a)}\sum_{n=0}^\infty\frac{\Gamma(a+n)x^n}{\Gamma(c+n)n!}$$
    \item Regular modified cylindrical Bessel function:
	$$ I_\nu(x) = i^{-\nu} J_\nu(ix) = \sum_{k=0}^{\infty}\frac{(x/2)^{\nu+2k}}{k!\Gamma(\nu+k+1)} \qquad \text{for } x \ge 0$$
    \item Cylindrical Bessel function of the first kind:
	$$ J_\nu(x) = \sum_{k=0}^{\infty}\frac{(-1)^k(x/2)^{\nu+2k}}{k!\Gamma(\nu+k+1)} \qquad \text{for } x \ge 0$$
    \item Irregular modified cylindrical Bessel function: 
	$$ K_\nu(x) = \frac{\pi}{2}i^{\nu+1}(J_\nu(ix) + iN_\nu(ix)) = 
	    \begin{cases}
		\frac{I_{-\nu}(x) - I_\nu(x)}{\sin\nu\pi}   \qquad \text{for } x\ge0 \text{ and } \nu \notin Z \\
		\displaystyle \frac{\pi}{2}\lim_{\mu\rightarrow\nu}\frac{I_{-\nu}(x) - I_\nu(x)}{\sin\mu\pi}	\qquad \text{for } x<0 \text{ and } \nu \in Z \\
	    \end{cases}
	    $$
    \item Cylindrical Neumann function of the first kind
	$$ N_\nu(x) =
	    \begin{cases}
		\frac{J_{\nu}(x)\cos\nu x - J_\mu(x)}{\sin\nu\pi} \qquad \text{for } x\ge0 \text{ and } \nu \notin Z \\
		\displaystyle \lim_{\mu\rightarrow\nu}\frac{J_{\mu}(x)\cos\mu\pi - J_{-\nu}(x)}{\sin\mu\pi} \qquad \text{for } x<0 \text{ and } \nu \in Z \\
	    \end{cases}
	    $$
	\item Incomplete elliptic integral of the first kind:
	    $$ F(k,\phi) = \int_0^{\phi}\frac{d\theta}{\sqrt{1-k^2\sin^2\theta}}	\qquad \text{for } |k| \le 1$$
	\item Incomplete elliptic integral of the second kind:
	    $$ E(k,\phi) = \int_0^{\phi}\sqrt{1-k^2\sin^2\theta}\ d\theta	\qquad \text{for } |k| \le 1$$
	\item Incomplete elliptic integral of the thrid kind:
	    $$ \Pi(k,\nu,\phi) = \int_0^{\phi}\frac{d\theta}{(1-\nu\sin^2\theta)\sqrt{1-k^2\sin^2\theta}}	\qquad \text{for } |k| \le 1$$
	\item Exponential integral:
	    $$ Ei(x) = -\int_{-x}^\infty \frac{e^{-t}}{t}dt $$
	\item Hermite polynomials:
	    $$ H_n(x) = (-1)^n e^{x^2} \frac{d^n}{dx^n}e^{-x^2} $$
	\item Hygergeometric series:
	    $$ F(a,b,c,x) = \frac{\Gamma(c)}{\Gamma(a)\Gamma(b)}\sum_{n=0}^{\infty}\frac{\Gamma(a+n)\Gamma{b+n}}{\Gamma(c+n)}\frac{x^n}{n!}$$
	\item Laguerre polynomials:
	    $$ L_n(x) = \frac{e^x}{n!} \frac{d^n}{dx^n}\left(x^n e^{-x} \right)	\qquad \text{for } x\ge 0$$
	\item Legendre polynomials:
	    $$ P_l(x) = \frac{1}{2^l l!} \frac{d^l}{dx^l}(x^2 - 1)^l	\qquad \text{for } |x| \le 1$$
	\item Riemann zeta function:
	    $$ Z(x) = 
	    \begin{cases}
		\sum_{k=1}^{\infty} k^{-x}  \qquad \text{for } x > 1\\
		2^x \pi^{x-1}\sin\left( \frac{x\pi}{2} \right)\Gamma(1-x)\zeta(1-x) \qquad \text{for } x < 1
	    \end{cases}
	    $$
	\item Spherical Bessel functions of the first kind:
	    $$ j_n(x) = \sqrt{\frac{\pi}{2x}} J_{n+1/2}(x)  \qquad \text{for } x \ge 0$$
	\item Spherical assocated Legendre function:
	    $$ Y_l^m(\theta, \phi) = (-1)^m \left[ \frac{2l+1}{4\pi} \frac{(l-m)!}{(l+m)!}\right]^{\frac{1}{2}} P_l^m(\cos\theta)e^{im\phi} \qquad \text{for } |m| \le l$$
	\item Spherical Neumann functions, Spherical Bessel functions of the second kind:
	    $$ n_n(x) = \left(\frac{\pi}{2x} \right)^{\frac{1}{2}} N_{n+\frac{1}{2}}(x)	\qquad \text{for } x \ge 0 $$
\end{itemize}
%%%%%%%%%%%%%%%%%%%%%%%%%%%%%%%%%%%%%%%%%%%%%%%%%%%%%%%%%%%%%%%%%%%%%%%%
Number of independent solution to a derivative equation ???
\section{Definition}

%%%%%%%%%%%%%%%%%%%%%%%%
\subsubsection{Levi-Civita}
$\epsilon_{ijk}$ is an antisymmetric tensor, which is defined as: 
\begin{equation}
    \label{eqn:math::Levi-Civita}
    \epsilon_{123} = 1, \epsilon_{ijj} = 0, \epsilon_{ijk} = -\epsilon_{jik}
\end{equation}

\subsubsection{Fourier Transform}
\begin{equation}
    \label{eqn:math::FT}
    \begin{gathered}
    \displaystyle \tilde{f}(k) = \int_{-\infty}^{\infty}dx e^{ikx}f(x)	\\
    \displaystyle f(x) = \int_{-\infty}^{\infty}\frac{dk}{2\pi}e^{-ikx}\tilde{f}(k)
    \end{gathered}
\end{equation}

If $f(x) = f(x+R)$, then 
\begin{equation}
    \tilde{f}(k) = \int dx e^{ikx}f(x) = e^{-ikR} \int d(x+R) e^{ik(x+R)}f(x+R) = e^{-ikR} \tilde{f}(k)
\end{equation}
So, $\tilde{f}(k)$ is non-zero only at point with $kR = 2n\pi$, or $k = \frac{2\pi}{R}n$.
\begin{equation}
    \displaystyle f(x) = \sum_{k} e^{-ikx}\tilde{f}(k)
\end{equation}

\section{Vector Operation}
\begin{equation}
    \nabla\times(\nabla\times\vec{A}) = \nabla(\nabla\vec{A}) - \nabla^2\vec{A} 
\end{equation}

Two point Fourier transformation:
\begin{equation}
    \displaystyle f(\bm{r,r'}) = \frac{1}{\Omega}\sum_{\bm{qq'}}e^{i\bm{q\cdot r}}f(\bm{q,q'})e^{-i\bm{q'\cdot r'}}
\end{equation}

%%%%%%%%%%%%%%%%%%%%%%%%%%%%%%%%%%%%%%%%%%%%%%%%%%%%%%%%%%%%%%%%%%%%%%%%
\section{Formulas}

%%%%%%%%%%%%%%%%%%%%%%%%%%%%%%%%%%%%%%%%%%%%%%%%
\subsection{Fourier transform}
\[
    \delta^3(\vec{x})=\int\frac{d^3k}{(2\pi)^3}e^{i\vec{k}\vec{x}}
    \]
%%%%%%%%%%%%%%%%%%%%%%%%%%%%%%%%%%%%%%%%%%%%%%%%
\subsection{Limitation}
For 
\[
    f(\omega) = \frac{\sin(N\omega)T}{\sin(\omega T)}   \quad (\omega > 0)
    \]
when $N\rightarrow\infty$, 
we can find that the dominant value lies in 
\[
    \omega = \frac{m\pi}{T} \qquad  m=1,2,3 \cdots
    \]
with $\omega=\frac{m\pi}{T}+\Delta\omega$, we can get:
\begin{equation}
    \displaystyle \lim_{N\rightarrow\infty}f(\omega)\approx\frac{(-1)^m\sin(N\Delta\omega T)}{(-1)^m\Delta\omega T}
\end{equation}
Note: $N\Delta\omega T$ is not infinity small, so we need to keep $\sin()$. \\
We also find that:
\begin{equation}
    \int_{-\epsilon}^{\epsilon}d(\Delta\omega)\frac{\sin(N\Delta\omega T)}{\Delta\omega T}
    =\frac{1}{T}\int_{-N\epsilon T}^{N\epsilon T} dx\frac{x}{x} = \frac{\pi}{T}
\end{equation}
So: 
\begin{equation}
    \displaystyle \lim_{N\rightarrow\infty}f(\omega)\approx\frac{(-1)^m\sin(N\Delta\omega T)}{(-1)^m\Delta\omega T}
    =\displaystyle \sum_{m=1}\frac{\pi}{T}\delta({\omega-\omega_m})
\end{equation}

Similarly:
\begin{equation}
    \displaystyle \lim_{N\rightarrow\infty}f_1(\omega)=\left(\frac{\sin(N\omega T)}{\omega T}\right)^2
    \approx\left(\frac{(-1)^m\sin(N\Delta\omega T)}{(-1)^m\Delta\omega T}\right)^2
\end{equation}

\begin{equation}
    \int_{-\epsilon}^{\epsilon}d(\Delta\omega)\left(\frac{\sin(N\Delta\omega T)}{\Delta\omega T}\right)^2
    =\frac{N}{T}\int_{-N\epsilon T}^{N\epsilon T} dx\frac{x}{x}
    =\frac{N\pi}{T}
\end{equation}
\begin{equation}
    f_1(\omega)=\displaystyle \sum_{m=1}\frac{N\pi}{T}\delta({\omega-\omega_m})
\end{equation}


\begin{equation}
    \displaystyle lim_{\alpha\rightarrow 0^{+}}\frac{1}{x \pm i\alpha} = P\left(\frac{1}{x}\right) \mp i\pi\delta(x)
\end{equation}
%%%%%%%%%%%%%%%%%%%%%%%%%%%%%%%%%%%%%%%%%%%%%%%%
\subsection{Integral}
\[
    I=\int_{-\infty}^{\infty}dk\frac{e^{ikr}-e^{-ikr}}{k}
    \]
Note that the integrand does not blow up as $k \rightarrow 0$, so
\[
    I=\displaystyle\lim_{\delta\rightarrow{0}}\left[\int_{-\infty}^{\infty}dk\frac{e^{ikr}-e^{-ikr}}{k+i\delta}\right]
    \]
So we have pole at $k = -i\delta$. For $e^{ikr}$ we must close the contour
up to get exponential decay at large $\mathit{k}$. This misses the pole, so
this term gives zero. So
\[
    I=\int_{-\infty}^{\infty}dk\frac{-e^{-ikr}}{k+i\delta}=-(2\pi{i})(-e^{-\delta{r}})=2\pi{i}e^{-i\delta{r}}
    \]

\subsubsection{Useful Integrals}
\[
    \int_0^\infty e^{-\alpha x}x^ndx = \frac{n!}{\alpha^{n+1}} 
    \]
%%%%%%%%%%%%%%%%%%%%%%%%%%%%%%%%%%%%%%%%%%%%%%%%%%%%%%%%%%%%%%%%%%%%%%%%
\section{Functions}

%%%%%%%%%%%%%%%%%%%%%%%%%%%%%%%%%%%%%%%%%%%%%%%%
\subsection{Interesting functions}
%%%%%%%%%%%%%%%%%%%%%%%%
\subsubsection{Weierstrass function}
\begin{equation}
    f(x) = \sum_{n=0}^{\infty}a^{n}cos(b^{n}\pi x)
\end{equation}
Continuous but not derivative everywhree.


%%%%%%%%%%%%%%%%%%%%%%%%%%%%%%%%%%%%%%%%%%%%%%%%
\subsection{Frequently used functions}

%%%%%%%%%%%%%%%%%%%%%%%%
\subsubsection{Legendre Polynomials and spherical harmonics}
\begin{equation}
    Y^m_l(\theta,\phi)=Ne^{im\phi}P^m_l(\cos\theta)
\end{equation}
which satisfies
\[
    r^2\nabla^2Y^m_l(\theta,\phi)=-l(l+1)Y^m_l(\theta,\phi)
    \]

Standard convention :
\[
    P^{-m}_l(\cos\theta)=(-1)^m\frac{(l-m)!}{(l+m)!}P^m_l(\cos\theta),
\]
In QM:
\[
    Y^m_l(\theta,\phi)=(-1)^msqrt{\frac{2l+1}{4\pi}\frac{(l-m)!}{(l+m)!}}P_{lm}(\cos\theta)e^{im\phi}
\]
Where $P_{lm}$ are associated Legendre polynomials without the
Condon-Shortley phase.

Though of different definition, all of them satisfy
\[ 
    Y_l^{m*}(\theta,\phi)=(-1)^{m}Y_{l}^{-m}(\theta,\phi) 
    \]


Associate Legendre Polynomials:

%%%%%%%%%%%%%%%%%%%%%%%%
\subsection{Bessel Function}
%%%%%%%%%%%%%%%%%%%%%%%%%%%%%%%%%%%%%%%%%%%%%%%%
\subsection{Differential Equations}

%%%%%%%%%%%%%%%%%%%%%%%%
\subsubsection{First Order Equation}
\[ 
\frac{dV}{dt} + \eta(t)V(t) = f(t)  
\]
so:
\[
    \frac{1}{I}\frac{d(I(t)V(t))}{dt} = \frac{dV(t)}{dt} + \frac{I'(t)}{I(t)}V(t)
    \]
if: 
\[ 
    \frac{I'(t)}{I(t)} = \eta(t)	\rightarrow I(t) = exp(\int^{t}dt'\eta(t'))
    \]
so:
\[ \frac{d(I(t)V(t))}{I dt} = f(t)\]

\subsubsection{inhomogeneous 2nd order differential equation}
\begin{equation}
    \frac{d^{2}x}{dt^2} + a(t)\frac{dx}{dt} + b(t)x = f(t)
\end{equation}
Its solution is given by
\begin{equation}
    \displaystyle x(t) = c_{1}\phi_{1}(t) + c_{2}\phi_{2}(t) + \int_{t_0}^{t}\frac{\phi_{1}(\xi)\phi_{2}(t) - \phi_{2}(\xi)\phi_{1}(t)}{W(\xi)} f(\xi)d\xi
\end{equation}
Where $\phi_{1}$ and $\phi_{2}$ are the 2 orthogonal solutions to the homogeneous 2nd differential equation, and $W(t)$ is the Wronskian.
\begin{equation}
    W(t) = 
	\begin{vmatrix}
	    \phi_1(t)	    & \phi_2(t) \\
	    \phi_1^{'}(t)   & \phi_2^{'}(t)	\\
	\end{vmatrix}
\end{equation}


\subsubsection{Pendulum Equation}
\begin{equation}
    \begin{gathered}
	\frac{d^2}{dz^2}\psi + ucos\psi = 0 \\
	\psi(z) = \psi(0) + \psi'(0)z - u\int_0^z dz_1 \int_0^{z_1}cos\psi(z_2)dz_2
    \end{gathered}
\end{equation}
%%%%%%%%%%%%%%%%%%%%%%%%
\subsubsection{Floquet theory}
\begin{equation}
    \ddot{x} + \omega_{0}^{2}(1+\mu cos(\nu t))x = 0
\end{equation}
Analogy to QM(Schrodinger Equation):	
\[
    -\frac{\hbar^{2}}{2m}{\psi''} + U(x)\psi = E\psi
    \]
where $U(x) = -U_{0}cos\frac{2\pi x}{a}$.
So let 
\[
    k^{2} = \frac{2mE}{\hbar^{2}}, u=\frac{U_0}{E}, \nu=\frac{2\pi}{a}
    \]
$u$ is small, we will get   
\[ 
    \psi''+k^{2}\psi = -k^{2}ucos(\nu x)\psi
    \]

%%%%%%%%%%%%%%%%%%%%%%%%%%%%%%%%%%%%%%%%%%%%%%%%
\subsection{Integrated functions}
%%%%%%%%%%%%%%%%%%%%%%%%
\subsubsection{$\Gamma$ function}
\[\Gamma(z) \equiv \int_{0}^{+\infty}dt e^{-t} t^{z-1} =
2\int_{0}^{+\infty}dx e^{-x^2}x^{2z-1}\]

\[\Gamma(z)\Gamma(1-z) = \frac{\pi}{sin\pi z}\]

%%%%%%%%%%%%%%%%%%%%%%%%%%%%%%%%%%%%%%%%%%%%%%%%%%%%%%%%%%%%%%%%%%%%%%%%
\section{Group}

\subsection{Representation of Group}
A \textbf{representation} of a group G on a vector space V over a field K is
a group homomorphism from G to GL(V), the 

%%%%%%%%%%%%%%%%%%%%%%%%%%%%%%%%%%%%%%%%%%%%%%%%%%%%%%%%%%%%%%%%%%%%%%%%
\section{Probability Distribution Functions}

%%%%%%%%%%%%%%%%%%%%%%%%%%%%%%%%%%%%%%%%%%%%%%%%
\subsection{Uniform}

%%%%%%%%%%%%%%%%%%%%%%%%%%%%%%%%%%%%%%%%%%%%%%%%
\subsection{Exponential}

%%%%%%%%%%%%%%%%%%%%%%%%%%%%%%%%%%%%%%%%%%%%%%%%
\subsection{Gauss}

%%%%%%%%%%%%%%%%%%%%%%%%%%%%%%%%%%%%%%%%%%%%%%%%
\subsection{BreitWigner}

%%%%%%%%%%%%%%%%%%%%%%%%%%%%%%%%%%%%%%%%%%%%%%%%
\subsection{Poisson}

%%%%%%%%%%%%%%%%%%%%%%%%%%%%%%%%%%%%%%%%%%%%%%%%%%%%%%%%%%%%%%%%%%%%%%%%
\section{Direvative Functions}

%%%%%%%%%%%%%%%%%%%%%%%%
\subsubsection{RC Oscillation}
\begin{equation}
    \dot{Q} + \frac{Q}{RC} = I(t) = I_0e^{-t/\tau}
\end{equation}
to get:
\begin{equation}
    Q(t) = A (e^{-t/\tau} - e^{-t/\tau_0})
\end{equation}
where: $$ \tau_0 = RC $$ 

If $$ \tau = \tau_0 $$, then the solution is different:
\begin{equation}
    Q(t) = I_0 te^{-t/\tau_0} + Ae^{-t/\tau_0}
\end{equation}
