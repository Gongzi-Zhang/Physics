\chapter{Definition}
Definition and Conclusion.

%%%%%%%%%%%%%%%%%%%%%%%%%%%%%%%%%%%%%%%%%%%%%%%%%%%%%%%%%%%%%%%%%%%%%%%%
\section{General}

%%%%%%%%%%%%%%%%%%%%%%%%%%%%%%%%%%%%%%%%%%%%%%%%
\subsection{index}
Entries $(\mathcal{L})_{\mn}$ of a matrix are labelled by rows ($\mu$) 
and columns ($\nu$).
You are \textbf{free} to move this row and column index up and down at 
will -- what matters is the order of index:
\[
    (\mathcal{L})_{\mn} = (\mathcal{L}^T)_{\mn} = {(\mathcal{L})^\mu}_\nu = {(\mathcal{L}^T)_\nu}^\mu = {L^\mu}_\nu
    \]
The much \textbf{preferred} placemnet of the indices surrounding the 
matrix is just a visual reminder of the individual entries ${L^\mu}_\nu$ 
which together form the matrix ($\mathcal{L}$) and ($\mathcal{L}^T$), 
and that's all.

The indices labelling ${L^\mu}_\nu$ can not be raised or lowered randomly,
but are raised and lowered with the metric tensor:
\[
    (g\mathcal{L})_{\mn} = g_{\mu\rho}{L^\rho}_{\nu} \equiv L_{\mn} 
    \]
and 
\[
    {(g\mathcal{L}g)_\mu}^\nu = g_{\mu\rho}{L^\rho}_\sigma g^{\sigma\nu} \equiv {L_\mu}^\nu
    \]

%%%%%%%%%%%%%%%%%%%%%%%%
\subsubsection{Vector Index}
\begin{equation}
    \begin{gathered}
	x^i = (x, y, z); \quad 
	x_i = 
	\begin{pmatrix}
	    x   \\
	    y   \\
	    z   \\
	\end{pmatrix}   \\
	x^\mu = (t, x, y, z); \quad 
	x_\mu = 
	\begin{pmatrix}
	    t	\\
	    x   \\
	    y   \\
	    z   \\
	\end{pmatrix}   \\
    \end{gathered}
\end{equation}



\subsection{Cross Section}
\begin{equation}
    \label{eqn:xsection}
    \sigma = \frac{\text{scattered power}}{\text{incident intensity}}
\end{equation}
quasi-periodic:
\begin{equation}
    \label{quasi-periodic}
    \phi(x+a) = e^{ika}\phi(x)
\end{equation}
There is a phase shift, so it is not rigorously periodic.


Rotation:
The algebraic characterization $R^TR=\mathcal{1}$ is a much more useful
definition of the group than the explicit form of the rotation matrices 
as a function of $\theta$.  \\
Another way to define the rotation group is as the set of linear 
transformations on $\mathcal{R}^n$ preserving the inner product 
$x^ix_i=\delta_{ij}x^ix^j$:
\begin{equation}
    \label{eqn:rotation}
    R_{ki}R_{lj}\delta_{kl} = (R^T)_{ik}\delta_{kl}R_{lj} = [(R^T)\mathcal{1}(R)]_{ij} = (R^TR)_{ij}=\delta_{ij}
\end{equation}


For an oscillating field $E(t) = E_0 e^{i(\vec{k}\cdot\vec{x}-\omega t}$, the time average is:
\begin{equation}
    <E> = \frac{1}{T}\int_0^TE(t)dt = \frac{1}{2}E_0e^{i\vec{k}\cdot\vec{x}}
\end{equation}

A plain wave $E = E_0 e^Pi(\vec{k}\cdot\vec{x}-\omega t)$ (such as the 
Fourier mode of a $\delta$ function) has \textbf{no} starting or ending 
point and it fulfills all the space. So understanding the scatter of a 
plain wave is some tricky thing.

Velocity: 
\begin{equation}
    \vec{v} = \nabla_{\vec{p}}E(|\vec{p}|)
\end{equation}

\textbf{Lagrangian (Hamiltonian)} is the Legendre transform of the Hamiltonian (Lagrangian)
\begin{equation}
    L(q, \dot{q}) = p(q, \dot{q})\dot{q} - H(q,p(q, \dot{q})), \quad 
    H(q, p) = p\dot{q}(q,p) - L(q,\dot{q}(q,p))
\end{equation}
where $p(q, \dot{q})$ is defined as $\frac{\partial H}{\partial p} = \dot{q}$
and $\dot{q}(q,p)$ is defined as $\frac{\partial L(q,\dot{q})}{\partial \dot{q}}=p$

%%%%%%%%%%%%%%%%%%%%%%%%%%%%%%%%%%%%%%%%%%%%%%%%
\subsection{Energy}
\begin{description}
    \item [Power]
	\begin{equation}
	    \frac{dP}{d\Omega} = \frac{dW}{dt\cdot d\Omega} = r^2\frac{dW}{dt dA}
	\end{equation}
\end{description}

%%%%%%%%%%%%%%%%%%%%%%%%%%%%%%%%%%%%%%%%%%%%%%%%
\subsection{Potential}
\begin{description}
    \item [Yukawa potential (screened Coulomb potential)]
	\begin{equation}
	    \label{eqn:Yukawa}
	    V = V_0\frac{e^{-\mu r}}{r}
	\end{equation}
\end{description}

%%%%%%%%%%%%%%%%%%%%%%%%%%%%%%%%%%%%%%%%%%%%%%%%
\subsection{Monte Carlo}
\begin{description}
    \item [Monte Carlo Method]
	A method to search for solutions to mathematical problem using a statistical sampling with random numbers.
\end{description}

%%%%%%%%%%%%%%%%%%%%%%%%%%%%%%%%%%%%%%%%%%%%%%%%
\subsection{Dirac Delta}
\begin{description}
    \item [Dimension] 
	Note that $\delta^3(x)$ and $\delta(x)$ have different dimensions.
	\begin{equation}
	    [\delta^3(x)] = L^{-3}, \qquad [\delta(x)] = L^{-1}
	\end{equation}
\end{description}

\subsection{Acronyms}
h.c.	\\
    Hermit conjugate	\\


%%%%%%%%%%%%%%%%%%%%%%%%%%%%%%%%%%%%%%%%%%%%%%%%%%%%%%%%%%%%%%%%%%%%%%%%
\section{CM}


%%%%%%%%%%%%%%%%%%%%%%%%%%%%%%%%%%%%%%%%%%%%%%%%%%%%%%%%%%%%%%%%%%%%%%%%
\section{QM}
Heisenberg EOM:
\begin{equation}
    \label{eqn:Heisenberg EOM}
    i\partial_t\phi(x) = [\phi, H]
\end{equation}

S-matrix
\begin{equation}
    \label{S-matrix}
    \bra{f}S\ket{i}_{heisernberg} = \bra{f;\infty}\ket{i;-\infty}_{Schrodinger}
\end{equation}

\subsubsection{Harmonic basis}
$\hat{\epsilon}_{-1} = \frac{\hat{x} - i\hat{y}}{\sqrt{2}}$
$\hat{\epsilon}_{0} = \hat{z}$
$\hat{\epsilon}_{1} = -\frac{\hat{x} + i\hat{y}}{\sqrt{2}}$

$\hat{\epsilon}_0$ refers to light linearly polarized in the z-direction
($\pi-$light). The polarizations $\hat{\epsilon_{\pm 1}}$ refer to 
circularly polarized light with $-1$ for left-handed ($\sigma^-$-light) 
and $+1$ for right-handed ($\sigma^+$-light)

\[
    \hat{\epsilon}_q \cdot \vec{r} = \sqrt{\frac{4\pi}{3}}rY_{1q}(\theta, \phi)
\]
%%%%%%%%%%%%%%%%%%%%%%%%%%%%%%%%%%%%%%%%%%%%%%%%
\subsection{Symmetry}

%%%%%%%%%%%%%%%%%%%%%%%%%%%%%%%%%%%%%%%%%%%%%%%%
\subsection{Perturbation Theory}

%%%%%%%%%%%%%%%%%%%%%%%%%%%%%%%%%%%%%%%%%%%%%%%%
\subsection{Scattering}
\begin{description}
    \item [Lippmann-Schwinger Equation]
	\begin{equation}
	    \label{eqn:qm:LSEqn}
	    \left\{
		\begin{aligned}
		    i\hbar\partial_t\psi^\dag(t,\vec{x}) &= H\psi^\dag(t, \vec{x})   \\
		    \psi^\dag(t=-\inf, x) &= \phi(t,x) = \frac{1}{(2\pi)^3/2}e^{-\frac{i}{\hbar}Et + i\vec{k}\cdot\vec{x}} 
		\end{aligned}
		\right.
	\end{equation}
\end{description}

\section{Relativity}

Postulates:
\begin{itemize}
    \item All inertial observers have the same equations of motion and the same physical laws.
    \item The speed of light is constant for all inertial frames.
\end{itemize}
Assuming internial frame K' moves with velocity $\vec{v} = v_0\hat{x}$ with respected to
rest internial frame K.
\subsection{Definition}
\begin{description}
    \item {4-vector}
	\begin{equation}
	    \label{eqn:sr::4vectors}
		x^\mu \equiv (ct, x^1, x^2, x^3) = (ct, \vec{x})
	\end{equation}

	In calculation, it's usually regarded as a column vector:
	\begin{equation}
	    x^\mu = 
	    \begin{pmatrix}
		ct	\\
		x^1	\\
		x^2	\\
		x^3	\\
	    \end{pmatrix}
	\end{equation}

    \item{4-velocity}
	\begin{equation}
	    \label{eqn:sr::4velocity}
	    U^\mu \equiv \frac{dx^\mu}{d\tau} = (\gamma c, \gamma \vec{v})
	\end{equation}

    \item[metric tensor] 
	\begin{equation}
	    \label{eqn:sr::metricTensor}
	    g^{\mn} = 
	    \begin{pmatrix}
		-1  & 0	& 0 & 0	\\
		0   & 1	& 0 & 0 \\
		0   & 0	& 1 & 0 \\
		0   & 0	& 0 & 1 \\
	    \end{pmatrix}
	\end{equation}
    \item [$\beta$, $\gamma$]
	\begin{equation}
	    \label{eqn:sm::betaGamma}
	    \begin{gathered}
		\beta \equiv \frac{v_0}{c} \\
		\gamma	\equiv \frac{1}{\sqrt{1-\beta^2}}	\Rightarrow \gamma \ge 1
	    \end{gathered}
	\end{equation}
	Play of $\beta$ and $\gamma$
	\begin{equation}
	    \begin{gathered}
		\gamma^2 - \beta^2\gamma^2 = \gamma^2(1-\beta^2) = 1	\\
		1-\beta^2 = (1+\beta)(1-\beta) \xRightarrow{\beta\sim 1} 2(1-\beta)  \\
		\beta \sim 0 \Rightarrow \gamma \sim 1  \\
	    \end{gathered}
	\end{equation}

    \item [Proper time]
	\begin{equation}
	    \label{eqn:sr::properTime}
	    d\tau = \frac{1}{\gamma}dt
	\end{equation}
\end{description}
\subsection{Transformation}
\begin{description}
    \item [Spacetime]
	\begin{equation}
	    \label{eqn:sr::spacetime}
	    \begin{pmatrix}
		ct' \\
		x'  \\
		y'  \\
		z'  \\
	    \end{pmatrix}
	    =
	    \begin{pmatrix}
		\gamma	& -\bg	&   0	& 0 \\
		-\bg	& \gamma    & 0	& 0 \\
		0   & 0	& 1 & 0	\\
		0   & 0	& 0 & 1	\\
	    \end{pmatrix}
	    \begin{pmatrix}
		ct \\
		x  \\
		y  \\
		z  \\
	    \end{pmatrix}
	\end{equation}

    \item [Velocity]
	\begin{equation}
	    \label{eqn:sr::velocity}
	    \begin{gathered}
		c = c	\\
		v'^x = \frac{v^x - v_0}{1-v^xv_0/c^2}	\\
		v'^y = \frac{v^y}{\gamma(1-v^xv_0/c^2)}	\\
		v'^z = \frac{v^z}{\gamma(1-v^xv_0/c^2)}	\\
	    \end{gathered}
	\end{equation}
\end{description}

\subsection{Examples}
A object moves in velocity $\vec{v} = v_0\hat{x}$, then in the 
object rest frame K', we measure its length as $l_0$; and the
proper time for an evnet as $d\tau$; then in the lab frame, what 
we meaure are $L$ and $dt$:
\begin{description}
    \item [length contract effect]
	\begin{equation}
	L = \frac{l_0}{\gamma}
	\end{equation}

    \item [time dilation]
	\begin{equation}
	    dt = \gamma d\tau
	\end{equation}
\end{description}

%%%%%%%%%%%%%%%%%%%%%%%%%%%%%%%%%%%%%%%%%%%%%%%%%%%%%%%%%%%%%%%%%%%%%%%%
\section{QFT}

%%%%%%%%%%%%%%%%%%%%%%%%%%%%%%%%%%%%%%%%%%%%%%%%%%%%%%%%%%%%%%%%%%%%%%%%
\section{Nucleus}
\begin{description}
    \item [Radiation length]
    \item [Mean free path]
    \item [$dE/dx$]
\end{description}

\begin{description}
    \item [Decay rate] 
	\begin{equation}
	    \label{eqn:nu::decayRate}
	    \Gamma = 1/\tau
	\end{equation}
\end{description}

%%%%%%%%%%%%%%%%%%%%%%%%%%%%%%%%%%%%%%%%%%%%%%%%%%%%%%%%%%%%%%%%%%%%%%%%
\section{HEP}
Fermi's golden rule: the transition rate between two states is proportional
to the matrix element squared:
\begin{equation}
    \label{Fermi's golden rule}
    \Gamma \approx |\mathcal{M}|^2\delta(E_f-E_i)
\end{equation}
The matrix element
\begin{equation}
    \label{matrix element}
    \mathcal{M} = \bra{f}H_{int}\ket{i}
\end{equation}
%%%%%%%%%%%%%%%%%%%%%%%%%%%%%%%%%%%%%%%%%%%%%%%%%%%%%%%%%%%%%%%%%%%%%%%%
\section{Miscellaneous}
\begin{description}
    %%%%%%%%% A %%%%%%%%%%% 
    \item [adjoint representation]
    \item [analyticity]
    \item [associativity of addition and multiplication]    
	$a+(b+c) = (a+b)+c$   , and $a\cdot(b\cdot c)=(a\cdot b)\cdot c$
    \item [bilinear]	In QFT, a bilinear term means it has exactly two
	fields. Such as:
	\[ \mathcal{L}_K \supset \frac{1}{2}\phi\Box\phi,
	\frac{1}{4}F^2_{\mn}, \frac{1}{2}m^2\phi^2,
	\frac{1}{2}\phi_1\Box\phi_2, \phi_1\partial_\mu A_\mu, \dots \]

    %%%%%%%%% B %%%%%%%%%%% 
    \item [Boltzmann distribution]  $n_{i} = Ne^{-\beta E_{i}}$

    %%%%%%%%% C %%%%%%%%%%% 
    \item [causality]
    \item [charge conjugation C] taking particles to antiparticles
	\[  C:\quad\psi\rightarrow-i\gamma_2\psi^*\equiv\psi_C	\]
	In the Weyl basis, $\gamma_2^*=-\gamma_2$ and $\gamma_2^T=\gamma_2$, so
	\[  C:\quad\psi^*\rightarrow-i\gamma_2\psi  \]
    \item [chirality] The handedness of a spinor is referred to as its
	chirality. The \textbf{left-handed} and \textbf{right-handed} 
	refer to the ($\frac{1}{2},0$) or (0,$\frac{1}{2}$) representations
	of the Lorentz group. This concept only exists for spinors, or more
	precisely for (A,B) representations of the Lorentz group with
	$A\neq B$. Almost always, chirality means that a theory is not
	symmetric between left-handed Weyl spinors $\psi_L$ and right-handed
	spinors $\psi_R$
    \item [cluster decomposition principle]
    \item [commutativity of addition and multiplication] 
	$a+b=b+a$ and $a\cdot b = b\cdot a$
    \item [contravariant vectors]   vectors with upper indices
    \item [covariant vectors]	vectors with lower indices

    %%%%%%%%% D %%%%%%%%%%% 
    \item [Dirac spinors]
    \item [Dirac Lagrangian] 
	\[
	    \mathcal{L}=\bar{\psi}(i\slashed{\partial}-m)\psi
	\]
    \item [Dirac equation]
	\[  
	    (i\slashed{D} -m)\psi =0	
	\]
    \item [Dirac Mass]
	\[  
	    \mathcal{L}_{\text{Dirac mass}} = m(\psi_L^\dag\psi_R + \psi^\dag_R\psi_L) 
	\]
    \item [Distributivity of multiplication over addition]
	$a\cdot(b+c)=(a\cdot b) + (a\cdot c)$

    %%%%%%%%% E %%%%%%%%%%% 
    \item [equipartion theorem] a body in thermal equilibrium should have
	energy equally distributed among all possible modes, (mode is a
	seperation of phase space), which means all modes have the same
	energy.

    %%%%%%%%% F %%%%%%%%%%% 
    \item [faithful representation] A representation in which each group
	element gets its own matrix is called a \emph{faithful
	representation}.
    \item [Fermi's golden rule]	$\Gamma \sim |\mathcal{M}|^{2}\delta(E_f - E_i)$
    \item [first principle]

    %%%%%%%%% G %%%%%%%%%%% 
    \item [Gauge transform] $\phi \rightarrow e^{-i\alpha}\phi$
    \item [good quantum number] which doesn't change along time. $[Q, H] = 0$
    \item [Grassmann Numbers] For Majorana masses to be non-trivial, fermion 
	compnents cannot be regular numbers, they must be anticommuting numbers. 
	Such things are called Grassmann numbers.

    %%%%%%%%% H %%%%%%%%%%% 
    \item [Helicity] spin projected on the direction of motion is called the
	helicity. $\hat{h}=\frac{\vec{\sigma}\cdot\vec{p}}{|\vec{p}|}$

    %%%%%%%%% L %%%%%%%%%%% 
    \item [Legendre transformation]
    \item [Levi-Civita] $\epsilon_{ijk} = \left\{
	\begin{aligned}
	    1\quad  normal order(123,231,312) \\
	    -1\quad reverse order(132,213,321)	\\
	    0\quad  otherwise	\\
	\end{aligned} \right.$
    \item [Lie algebra] the generators of the Lie group form an algebra
	called its Lie algebra. 
    \item [Lie groups]	Lie groups are a class of groups, including the
	Lorentz group, with an infinite number of elements but a finite
	number of generators.
    \item [lightlike]	$V^\mu V_\mu = 0$
    \item [Lippmann-Schwinger eqn] 
	In P.T.
	\[
	    H = H_0 + V (correction)
	    \]
	If 
	\[
	    H_0\ket{\phi} = E\ket{\phi}, \quad and \quad 
	    H\ket{\psi} = E\ket{\psi}
	    \]
	Then
	\[
	    \ket{\psi} = \ket{\phi} + \frac{1}{E-H_0}V\ket{\psi}
	    \]
	This is called the \textbf{Lippmann-Schwinger equation}. And the inverted object 
	is a kind of Green's function known as the \textbf{Lippmann-Schwinger kernel}:
	\[
	    \prod_{LS} = \frac{1}{E-H_0}
	    \]
	Define operator T, so that
	\[
	    V\ket{\psi} = T\ket{\phi}
	    \]
	So
	\[
	    \ket{\psi} = \ket{\phi} + \frac{1}{E-H_0}T\ket{\phi}
	    \]
	and 
	\[
	    T = V + V\frac{1}{E-H_0}T
	    \]
	Solve it perturbatively in V:
	\[
	    \begin{aligned}
		T &= V + V\frac{1}{E-H_0}V + V\frac{1}{E-H_0}V\frac{1}{E-H_0}V + \cdots
		  &= V + V\prod_{LS}V + V\prod_{LS}V\prod_{LS}V + \cdots
	    \end{aligned}
	    \]

    \item [little group] The representation of the full \Poincare{}
	group is induced by a representation of the subgorup of the
	\Poincare{} group that holds $p^\mu$ fixed, called the
	\emph{little group}. When $P_\mu$ is massive, the little group is
	SO(3); when $p_\mu$ is massless, the little group is ISO(2).
    \item [locality]
    \item [Lorentz group] this is the generalization of the rotation group
	to include both rotations and boosts.

    %%%%%%%%% M %%%%%%%%%%% 
    \item [Majorana spinor] A spinor whose antiparticle is itself. 
    \item [Majorana masses]
	\[
	    \mathcal{L}=i\psi^\dag_L\sigma_\mu\partial_\mu\psi_L+i\frac{m}{2}(\psi^\dag_L\sigma_2\psi^*_L-\psi^T_L\sigma_2\psi_L)
	\]
	The mass terms in this Lagrangian are called \textbf{Majorana masses}.
    \item [Majorana fermions]
	in Dirac spinors:
	\[
	    \psi=\begin{pmatrix}
		\psi_L	\\
		i\sigma_2\psi^*_L   \\
	    \end{pmatrix}
	\]

    %%%%%%%%% P %%%%%%%%%%% 
    \item [pseudo scalar] particles with odd \textbf{parity}

    %%%%%%%%% Q %%%%%%%%%%% 
    \item [quantize]	promote x and p as operators and impose the
	canonical commutation relations:    $[x, p] = i$
    \item [quantum process]  time evolution of an open quantum system ???

    %%%%%%%%% R %%%%%%%%%%% 
    \item [Rarita-Schwinger field]  spin-$\frac{3}{2}$
    \item [representation] A set of objects that mix under a transformation
	group is called a representation of the group, though technically
	the matrix embedding is the representation. A representation is a 
	particular embedding of group elements into operators that act on a 
	vector space. For finite-dimensional representations, this means an 
	embedding of the $g_i$ into matrices. 

    %%%%%%%%% S %%%%%%%%%%% 
    \item [second quantization]	canonical quantization of relativistic
	fields, 
	\[  H_0 = \int \frac{d^{3}p}{(2\pi)^3}\omega_p(a^{\dag}_p a_p +
	\frac{1}{2}) \] 
	First quantization refer to the discrete mode, for
	example, of a particle in a box. Second quantization refers to the
	integer numbers of excitations of each of these modes. There are two
	features in second quantization:
	\begin{enumerate}
	    \item We have many quantum mechanical systems - one for each
		$\vec{p}$ - all at the same time.
	    \item We interpret the nth excitation of the $\vec{p}$ harmonic
		oscillator as having n \emph{particles}.
	\end{enumerate}
    \item [S-matrix]
    \item [spacelike]	$V^\mu V_\mu < 0$
    \item [SO(n)] the group of nD rotations ($det(R) = 1$)
	
    %%%%%%%%% T %%%%%%%%%%% 
    \item [time dilation] In relativity, time dilation is a difference in 
	the elapsed time measured by two observers.
    \item [timelike]	$V^\mu V_\mu > 0$

    %%%%%%%%% U %%%%%%%%%%% 
    \item [unitary] $\Lambda^{\dag}\Lambda = 1$

    %%%%%%%%% W %%%%%%%%%%% 
    \item [Weyl spinor] Irreducible unitary spin-$\frac{1}{2}$
	representations of the \Poincare{} group.

    %%%%%%%%% $\gamma$ %%%%%%%%%%% 
    \item [$\gamma$ matrix]
	In the Weyl basis:
	\[  \gamma_\mu = 
	    \begin{pmatrix}
		0   & \sigma_\mu    \\
		\bar{\sigma}_\mu  & 0	\\
	    \end{pmatrix} 
	\]
    \item [$\gamma^5$]
	\[
	    \gamma^5 \equiv i\gamma^0\gamma^1\gamma^2\gamma^3
	\]
	In the Weyl representation:
	\[
	    \gamma^5 = 
	    \begin{pmatrix}
		-\mathcal{1}	&   \\
		    & \mathcal{1}   \\
	    \end{pmatrix}
	\]
\end{description}
