\chapter{Quantum Mechanics}
%%%%%%%%%%%%%%%%%%%%%%%%%%%%%%%%%%%%%%%%%%%%%%%%
\subsection{Harmonic Oscillator}
\[
    \bra{n'}x\ket{n}=\sqrt{\hbar/2m\omega}(\sqrt{n+1}\delta_{n',n+1}+\sqrt{n}\delta_{n',n-1})
    \]
For $V(r)=\frac{1}{2}m\omega^2r^2=\frac{1}{2}kr^2$
\[
    \phi_0(r)=[(\frac{\alpha}{\sqrt{\pi}})^{1/2}e^{-\frac{1}{2}\alpha^2x^2}]
    [(\frac{\alpha}{\sqrt{\pi}})^{1/2}e^{-\frac{1}{2}\alpha^2y^2}]
    [(\frac{\alpha}{\sqrt{\pi}})^{1/2}e^{-\frac{1}{2}\alpha^2z^2}]
    \]
where $\alpha=(\frac{mk}{\hbar^2})^{1/4}=(\frac{m\omega}{\hbar})^{1/2}$
%%%%%%%%%%%%%%%%%%%%%%%%%%%%%%%%%%%%%%%%%%%%%%%%%%%%%%%%%%%%%%%%%%%%%%%%

\section{Rotation}

%%%%%%%%%%%%%%%%%%%%%%%%%%%%%%%%%%%%%%%%%%%%%%%%
\subsection{Vector}
In QM, we demand that the expectation value of a vector operator transforms
under rotation like a classical vector.
\begin{gather*}
    \ket{\alpha}\rightarrow\ket{\alpha}^R=D(R)\ket{\alpha}\\
    ^{R}\bra{\alpha}V_i\ket{\alpha}^R=\bra{\alpha}D^{\dag}V_{i}D\ket{\alpha}=R_{ij}\bra{\alpha}V_j\ket{\alpha}\\
    D^{\dag}V_{i}D=R_{ij}V_j
\end{gather*}
With infinitesimal transform, we will find:
\begin{equation}
    [J_i,V_j]=i\hbar\epsilon_{ijk}V_k
\end{equation}
Let $V_0 = V_Z$ and $V_{\pm} = \frac{\mp V_x -iV_y}{\sqrt{2}}$, 
then
\[
    V_0 = [J_-, V_+], \quad V_- = [J_-, V_0], \qaud 
\]

%%%%%%%%%%%%%%%%%%%%%%%%%%%%%%%%%%%%%%%%%%%%%%%%
\subsection{Tensor}
We define spherical tensor of rank k with $(2k+1)$ components labelled by q
as
\begin{equation}
    D^{\dag}(R)T^{(k)}_{q}D(R)=\displaystyle\sum_{q^\prime=-k}^{k}D^{(k)*}_{qq^\prime}T^{(k)}_{q^\prime}
\end{equation}
%%%%%%%%%%%%%%%%%%%%%%%%%%%%%%%%%%%%%%%%%%%%%%%%
\subsection{Spin-$1/2$} 
\[ 
D(R_{\hat{n}}(\phi)) = e^{-i\frac{\phi}{\hbar}\hat{n}\cdot\vec{S}} = e^{-i\frac{\phi}{2}\hat{n}\cdot\vec{\sigma}} = 
cos(\frac{\phi}{2}) - i \hat{n} \cdot \vec{\sigma} sin(\frac{\phi}{2}) = 
\begin{pmatrix}
    \cos(\frac{\phi}{2}) - in_{z}\sin(\frac{\phi}{2}) & (-i n_{x} - n_{y})\sin(\frac{\phi}{2}) \\ 
    (-i n_{x} + n_{y})\sin(\frac{\phi}{2}) &  \cos(\frac{\phi}{2}) + in_{z}\sin(\frac{\phi}{2})  
\end{pmatrix}
\]

%%%%%%%%%%%%%%%%%%%%%%%%%%%%%%%%%%%%%%%%%%%%%%%%
\subsection{Spin-1}
\begin{equation}
    S_x=\frac{\hbar}{\sqrt{2}}
    \begin{pmatrix}
	0   & 1	& 0 \\
	1   & 0	& 1 \\
	0   & 1	& 0 \\
    \end{pmatrix}\quad
    S_y=\frac{\hbar}{\sqrt{2}}
    \begin{pmatrix}
	0   & -i    & 0 \\
	i   & 0	& -i \\
	0   & i	& 0 \\
    \end{pmatrix}\quad
    S_z=\hbar
    \begin{pmatrix}
	1   & 0	& 0 \\
	0   & 0	& 0 \\
	0   & 0	& -1 \\
    \end{pmatrix}\quad
\end{equation}

%%%%%%%%%%%%%%%%%%%%%%%%
\subsubsection{Wigner-Echart theorem}
The matrix elements of tensor operators with respect to angular momentum
eigenstates.
\begin{equation}
    \bra{\alpha^\prime,j^{\prime}m^\prime}T^{(k)}_{q}\ket{\alpha,jm}
    =\frac{\bra{jk;mq}\ket{jk;j^{\prime}m^\prime}}{\sqrt{2j+1}}{\bra{\alpha^\prime,j^\prime}T^{(k)}\ket{\alpha,j}}
\end{equation}

%%%%%%%%%%%%%%%%%%%%%%%%
\subsubsection{Zeeman-effect}
This is effect only in \textbf{weak magnetism}

%%%%%%%%%%%%%%%%%%%%%%%%
\subsubsection{Paschen-Back effect}
This happen with \textbf{strong magnetism}($\frac{\Phi}{\Phi_0}\gg\alpha^2\sim10^{-4}$).


%%%%%%%%%%%%%%%%%%%%%%%%%%%%%%%%%%%%%%%%%%%%%%%%%%%%%%%%%%%%%%%%%%%%%%%%
\section{Symmetry}

%%%%%%%%%%%%%%%%%%%%%%%%%%%%%%%%%%%%%%%%%%%%%%%%
\subsection{Parity} 
Parity of Spherical Harmonics:
\[
    r\rightarrow{}r,\quad\theta\rightarrow\pi-\theta,\quad\phi\rightarrow\phi+\pi
    \]
So
\[
    Y_{ll}(\theta,\phi)\xrightarrow{P}Y_{ll}(\pi-\theta,\phi+\pi)=e^{il\pi}e^{il\phi}\sin^l(\pi-\theta)=(-1)^{l}Y_{ll})\theta,\phi
    \]
Because $L_\_$ is an \emph{axial vector}, so $P(L_\_)=1$.
\[
    L_\_Y_{ll}\xrightarrow{P}(-1)^{l}L_\_Y_{ll}
    \]
To get
\[
    Y_{lm}(\theta,\phi)\xrightarrow{P}Y_{lm}(\pi-\theta,\phi+\pi)=(-1)^{l}L_\_Y_{l,m+1}(\theta,\phi)
    \]

\subsection{Time reversal}
\[
    \ket{jm}\xrightarrow{T}(-1)^m\ket{j,-m}
    \]
