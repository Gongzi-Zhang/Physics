\chapter{Quantum Mechanics}
%%%%%%%%%%%%%%%%%%%%%%%%%%%%%%%%%%%%%%%%%%%%%%%%
\subsection{Harmonic Oscillator}
\[
    \bra{n'}x\ket{n}=\sqrt{\hbar/2m\omega}(\sqrt{n+1}\delta_{n',n+1}+\sqrt{n}\delta_{n',n-1})
    \]
For $V(r)=\frac{1}{2}m\omega^2r^2=\frac{1}{2}kr^2$
\[
    \phi_0(r)=[(\frac{\alpha}{\sqrt{\pi}})^{1/2}e^{-\frac{1}{2}\alpha^2x^2}]
    [(\frac{\alpha}{\sqrt{\pi}})^{1/2}e^{-\frac{1}{2}\alpha^2y^2}]
    [(\frac{\alpha}{\sqrt{\pi}})^{1/2}e^{-\frac{1}{2}\alpha^2z^2}]
    \]
where $\alpha=(\frac{mk}{\hbar^2})^{1/4}=(\frac{m\omega}{\hbar})^{1/2}$

%%%%%%%%%%%%%%%%%%%%%%%%%%%%%%%%%%%%%%%%%%%%%%%%%%%%%%%%%%%%%%%%%%%%%%%%
\section{Rotation}
Rotation of \textbf{Hilbert} space:
\begin{equation}
    [J_i, J_j] = i\hbar\epsilon_{ijk}J_k
\end{equation}

\begin{equation}
    J_{\pm} = J_x \pm iJ_y
\end{equation}

\begin{equation}
    J_{\pm}\ket{j,m} = \sqrt{(j\mp m)(j\pm m+1)}\hbar\ket{j,m\pm 1}
\end{equation}

\begin{equation}
    \label{Pauli matrixes}
    \sigma_x = 
    \begin{pmatrix}
	0   &	1   \\
	1   &	0   \\
    \end{pmatrix}   \quad
    \sigma_y = 
    \begin{pmatrix}
	0   &	-i   \\
	i   &	0   \\
    \end{pmatrix}   \quad
    \sigma_z = 
    \begin{pmatrix}
	1   &	0   \\
	0   &	-1   \\
    \end{pmatrix}   
\end{equation}


General rotation for spin:
\begin{equation}
    \begin{aligned}
	D(R_i(\varphi)) &= e^{-\frac{i\varphi}{\hbar}\hat{n}\cdot\frac{\hbar}{2}\vec{\sigma}} = e^{-\frac{i\varphi}{2}\hat{n}\cdot\vec{\sigma}} \\
	&= \displaystyle \sum_{k=0}^{\infty}\frac{(-i\varphi/2)^k}{k!}(\sigma_n)^k   \qquad  \sigma_n^2=1    \\
	&= \displaystyle \sum_{[even]}\frac{(-i\varphi/2)^k}{k!}\cdot\mathds{1} + \displaystyle \sum_{[odd]}\frac{(-i\varphi/2)^k}{k!}\cdot\sigma_n \\
	&= \cos\frac{\varphi}{2} -i\sigma_n\sin\frac{\varphi}{2}
    \end{aligned}
\end{equation}

Euler Representation:
\begin{equation}
    \begin{aligned}
    R(\ab\gamma) &= R_{z'}(\gamma)\cdot R_{y'}(\beta)\cdot R_{z}(\alpha)	\\
	    &= R_{y'}(\beta)R_z(\beta)R_{y'}^{-1}(\beta)\cdot R_{y'}(\beta)\cdot R_z(\alpha)	\\
	    &= [R_{z}(\alpha)R_y(\beta)R_{z}(\gamma)]\cdot R_{z}(\gamma)\cdot R_z(\alpha)	\\
	    &= R_z(\alpha)R_y(\beta)R_z(\gamma)	\\
    \end{aligned}
\end{equation}

\begin{equation}
    \ket{\varphi}_R = \ket{R\varphi} = D(R)\ket{\varphi}
\end{equation}

\begin{equation}
    D^{j}_{mm'}(R) \equiv \bar{jm}D(R)\ket{jm'}
\end{equation}

%%%%%%%%%%%%%%%%%%%%%%%%%%%%%%%%%%%%%%%%%%%%%%%%
\subsection{Vector}
In QM, we demand that the expectation value of a vector operator transforms
under rotation like a classical vector.
\begin{gather*}
    \ket{\alpha}\rightarrow\ket{\alpha}^R=D(R)\ket{\alpha}\\
    ^{R}\bra{\alpha}V_i\ket{\alpha}^R=\bra{\alpha}D^{\dag}V_{i}D\ket{\alpha}=R_{ij}\bra{\alpha}V_j\ket{\alpha}\\
    D^{\dag}V_{i}D=R_{ij}V_j
\end{gather*}
With infinitesimal transform, we will find:
\begin{equation}
    [J_i,V_j]=i\hbar\epsilon_{ijk}V_k
\end{equation}
Let $V_0 = V_Z$ and $V_{\pm 1} = \frac{\mp V_x -iV_y}{\sqrt{2}}$, 
then
\[
    \begin{aligned}[]
	[J_\pm, V_\mp] = \sqrt{2}V_0 , \quad [J_\pm, V_0] = \sqrt{2}V_\pm \\
	[J_\pm, V_\pm] = 0, \quad [J_0, V_q] = qV_q
    \end{aligned}
\]
Where
\begin{equation}
    J_\pm \equiv J_x \pm iJ_y	\quad J_{\pm 1} = \mp \frac{1}{\sqrt{2}}J_\pm, \quad \text{and} \quad J_0 = J_z
\end{equation}
\begin{equation}
    \vec{J}\cdot\vec{V} = J_xV_x + J_yV_y + J_zV_z = J_+V_+ + J_0V_0 + J_-V_- = J_0V_0 - J_{+1}V_{-1} - J_{-1}V_{+1}
\end{equation}

The Projection Theorem:
\begin{equation}
    \bra{\alpha',jm'}V_q\ket{\alpha,jm} = \frac{\bra{\alpha',jm}J\cdot V\ket{\alpha,jm}}{\hbar^2j(j+1)}\bra{jm'}J_q\ket{jm}
\end{equation}



%%%%%%%%%%%%%%%%%%%%%%%%%%%%%%%%%%%%%%%%%%%%%%%%
\subsection{Tensor}
We define spherical tensor of rank k with $(2k+1)$ components labelled by q
as
\begin{equation}
    (T^k_q)^R \equiv D^{\dag}(R)T^{(k)}_{q}D(R)=\displaystyle\sum_{q^\prime=-k}^{k}D^{(k)*}_{qq^\prime}T^{(k)}_{q^\prime}
\end{equation}

\begin{equation}
    \begin{gathered}
	[J_{\pm}, T^k_q] = \hbar\sqrt{k(k+1) -q(q\pm 1)}T^k_{q\pm 1}    \\
	[J_z, T^k_q] = \hbar qT^k_q
    \end{gathered}
\end{equation}
%%%%%%%%%%%%%%%%%%%%%%%%%%%%%%%%%%%%%%%%%%%%%%%%
\subsection{Spin-$1/2$} 
\[ 
D(R_{\hat{n}}(\phi)) = e^{-i\frac{\phi}{\hbar}\hat{n}\cdot\vec{S}} = e^{-i\frac{\phi}{2}\hat{n}\cdot\vec{\sigma}} = 
cos(\frac{\phi}{2}) - i \hat{n} \cdot \vec{\sigma} sin(\frac{\phi}{2}) = 
\begin{pmatrix}
    \cos(\frac{\phi}{2}) - in_{z}\sin(\frac{\phi}{2}) & (-i n_{x} - n_{y})\sin(\frac{\phi}{2}) \\ 
    (-i n_{x} + n_{y})\sin(\frac{\phi}{2}) &  \cos(\frac{\phi}{2}) + in_{z}\sin(\frac{\phi}{2})  
\end{pmatrix}
\]

%%%%%%%%%%%%%%%%%%%%%%%%%%%%%%%%%%%%%%%%%%%%%%%%
\subsection{Spin-1}
\begin{equation}
    S_x=\frac{\hbar}{\sqrt{2}}
    \begin{pmatrix}
	0   & 1	& 0 \\
	1   & 0	& 1 \\
	0   & 1	& 0 \\
    \end{pmatrix}\quad
    S_y=\frac{\hbar}{\sqrt{2}}
    \begin{pmatrix}
	0   & -i    & 0 \\
	i   & 0	& -i \\
	0   & i	& 0 \\
    \end{pmatrix}\quad
    S_z=\hbar
    \begin{pmatrix}
	1   & 0	& 0 \\
	0   & 0	& 0 \\
	0   & 0	& -1 \\
    \end{pmatrix}\quad
\end{equation}

%%%%%%%%%%%%%%%%%%%%%%%%
\subsubsection{Wigner-Echart theorem}
The matrix elements of tensor operators with respect to angular momentum
eigenstates.
\begin{equation}
    \bra{\alpha^\prime,j^{\prime}m^\prime}T^{(k)}_{q}\ket{\alpha,jm}
    =\frac{\bra{jk;mq}\ket{jk;j^{\prime}m^\prime}}{\sqrt{2j+1}}{\bra{\alpha^\prime,j^\prime}T^{(k)}\ket{\alpha,j}}
\end{equation}

%%%%%%%%%%%%%%%%%%%%%%%%
\subsubsection{Zeeman-effect}
This is effect only in \textbf{weak magnetism}

%%%%%%%%%%%%%%%%%%%%%%%%
\subsubsection{Paschen-Back effect}
This happen with \textbf{strong magnetism}($\frac{\Phi}{\Phi_0}\gg\alpha^2\sim10^{-4}$).


%%%%%%%%%%%%%%%%%%%%%%%%%%%%%%%%%%%%%%%%%%%%%%%%%%%%%%%%%%%%%%%%%%%%%%%%
\section{Symmetry}

%%%%%%%%%%%%%%%%%%%%%%%%%%%%%%%%%%%%%%%%%%%%%%%%
\subsection{Parity} 
Parity of Spherical Harmonics:
\[
    r\rightarrow{}r,\quad\theta\rightarrow\pi-\theta,\quad\phi\rightarrow\phi+\pi
    \]
So
\[
    Y_{ll}(\theta,\phi)\xrightarrow{P}Y_{ll}(\pi-\theta,\phi+\pi)=e^{il\pi}e^{il\phi}\sin^l(\pi-\theta)=(-1)^{l}Y_{ll})\theta,\phi
    \]
Because $L_\_$ is an \emph{axial vector}, so $P(L_\_)=1$.
\[
    L_\_Y_{ll}\xrightarrow{P}(-1)^{l}L_\_Y_{ll}
    \]
To get
\[
    Y_{lm}(\theta,\phi)\xrightarrow{P}Y_{lm}(\pi-\theta,\phi+\pi)=(-1)^{l}L_\_Y_{l,m+1}(\theta,\phi)
    \]

\subsection{Time reversal}
\[
    \ket{jm}\xrightarrow{T}(-1)^m\ket{j,-m}
    \]


%%%%%%%%%%%%%%%%%%%%%%%%%%%%%%%%%%%%%%%%%%%%%%%%
\section{Young Tableau}
A way to describe 

%%%%%%%%%%%%%%%%%%%%%%%%%%%%%%%%%%%%%%%%%%%%%%%%%%%%%%%%%%%%%%%%%%%%%%%%
\section{Scattering}
In the case of an electron of mass $m_e$ scatters off a proton of mass $m_p$. 
From non-relativistic QM, the xsection should be given by the Born approximation:
\begin{equation}
    \left(\frac{d\sigma}{d\Omega}\right)_{Born} = \frac{m_e^2}{4\pi^2}|\tilde{V}(\vec{k})|^2
\end{equation}
Where the Fourier transform of the potential is given by:
\[
    \tilde{V}(\vec{k}) = \int d^3xe^{-i\vec{k}\vec{x}}V(\vec{x})
    \]
$\vec{k}$ is the difference in the electron momentum before and after scattering, 
sometimes called the \emph{momentum transfer}. For Coulomb potential, 
$V(x) = \frac{e^2}{4\pi|\vec{x}|}$, $\tilde{V}(\vec{k}) = \frac{e^2}{\vec{k}^2}$, 
So
\[
    \left(\frac{d\sigma}{d\Omega}\right)_{Born} = \frac{m_e^2}{4\pi^2}\left(\frac{e^2}{\vec{k}^2}\right)^2
    \]
