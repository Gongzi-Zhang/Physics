\textbf{逻辑}

\textbf{规律}:\\
    从大量同类事实中总结出的一般性道理。

\textbf{理论}
    从逻辑推倒得出的一般性结论,是按照已有的实证知识、经验、事实、法则、认知以及经过验证的假说,经由一般化与演绎推理等方法,进行合乎逻辑的推论性总结。
    理论不考查其是否经过检验,只考察是否合乎逻辑。而定理是经过检验(一定条件下)确定为真的。


\textbf{axiom}:	\\
    公理(公设),指不需要证明的道理。公: 公共的,即适用于一切场合。

\textbf{Theorem}:
    定理。指在一定条件下成立的理论,是经过受逻辑限制的证明为真的陈述。基于公理由逻辑推倒而得。

\textbf{Law}:
    定律。指在一定条件下成立的规律,是由不变的事实规律所归纳出的结论,是对客观事实的一种表达形式,是通过大量具体的客观事实经验累积归纳而成的结论。
规律不考察其中涉及的原理,只总结事务之间的联系。

\textbf{Definition}:
    定义。指在一定条件下成立的涵义,是抽象的、普遍的想法、观念或充当指明实体、事件或关系的范畴或类的实体。
