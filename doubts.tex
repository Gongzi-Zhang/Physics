\newcommand{\DOF}{degree of freedom}

\section{Doubts}
Some doubt about physics

\subsection{Questions}
\begin{itemize}
    \item gravitational redshift: will light interact with gravity? When shine 
	a beam of light into the sky, the light doesn't slow down, but gravity 
	does take away some of its energy? (how? light has no mass) -- but this 
	i observed in S2 (a bright, giant star that orbits the black hole 
	SgrA* every 16 years) in 2018
    \item Does mass continuous? Does it has a minimum unit?
\end{itemize}

\subsection{Classical Physics}
\begin{description}[style=nextline]
    \item [Newton's Second Law]	Why it is relationship between acceleration
	and force, rather than velocity and force?
    \item [Action] $S = \int Ldt$, why L, but not other quantities, e.g. H.
    \item [Lagrangian] What's the intuitive physical meaning of 
	$L = E_{k} - E_{p}$
\end{description}

\subsection{Statistical Physics}
\begin{description}[style=nextline]
    \item Now that we have smallest energy principle, does the assumption that
	each state is equivalent in SM violate the smallest energy principle?
    \item [Enthalpy, Free energy, Gibbs Free energy] What do H,F,G stand for? 
	Physical meaning?
    \item [Entropy] How do we know the entropy defined in $\Delta \Gamma =
	\frac{d\Gamma}{dE}\Delta E = e^{S}$ is the same one as we defined in
	$dE = TdS - PdV$ ?
\end{description}


\section{Notation}
\subsection{Electrodynamics}

The reason a conductor puts boundary conditions on the EM field is that the
electrons move around to compensate for an applied field. But there is a
limit on how fast the electrons can move. The resulting cutoff frequency is
called the \textit{plasma frequency}
\begin{description}[style=nextline]
    \item [Maxwell Eqn] 
	\begin{equation}
	    \left.
	    \begin{aligned}
		\vec{\nabla} \times (\vec{\nabla} \varphi) &= 0 \\
		\vec{\nabla} \cdot (\vec{\nabla} \times \vec{A}) &= 0
	    \end{aligned}
	    \right\} 
	    \Rightarrow 
	    \left.
	    \begin{aligned}
		\vec{\nabla} \cdot \vec{E} &= \rho \\
		\vec{\nabla} \times \vec{B} &=
		\frac{1}{c} \vec{j} + \frac{1}{c}\partial_{t}\vec{E}   \\
		\vec{\nabla} \cdot \vec{B} &= 0	\\
		-\vec{\nabla} \times \vec{E} &=
		\frac{1}{c} \partial_{t}\vec{B}   \\
	    \end{aligned}
	    \right\}	
	\end{equation}
	With $\vec{E}$ and $\vec{B}$ still connected
	with each other, there is one \textbf{DOF}.
\end{description}

Now that EM wave can only transport transversed wave ($\vec{k}\cdot\vec{E} =0$), then
how does the static electric potential $\varphi$ could produce $\vec{E}$ in
the direction of $\vec{r}$? So the formation of a static electric field is 
different from the propagation of EM wave.


\subsection{Statistical Physics}
\begin{description}[style=nextline]
    \item [Partisian func] When count from particle (one particle can occupy
	how many states), remember the $\frac{1}{N!}$ factor! When count from
	state (how many particles in one state), don't need $\frac{1}{N!}$.
\end{description}

\subsection{Quantum Mechanics}
\begin{description}[style=nextline]
    \item [Baker-Campbell-Hausdorff]	
	if $[A,B] = \mathbb{C}$, then:	\\
	$$ e^{A}e^{B} = e^{B}e^{A}e^{[A,B]}, \qquad 
	e^{A}e^{B} = e^{A+B+\frac{1}{2}[A,B]}, \qquad
	e^{A+B} = e^{B+A} $$
    \item [Angular Momentum] Rep. of J in the Hilbert Space are discrete
	(finite) $\Leftarrow$ subspace.
	$$Y^m_l(\theta,\phi) = \bra{\theta,\phi}\ket{l,m} =
	\sqrt{\frac{2l+1}{4\pi}\frac{(l-m)!}{(l+m)!}}P^m_l(cos\theta)e^{im\phi} $$
	Where $P_l^m$ is the associate Legendre polynomial:
    \item [Harmonic Oscillator]	What's the physical meaning of the energy
	eigenstate $\ket{\phi}$, what's its relationship to $\ket{x}$ state?
\end{description}
