\newcommand{\SR}{special relativity}
%%%%%%%%%%%%%%%%%%%%%%%%%%%%%% Relativity %%%%%%%%%%%%%%%%%%%%%%%%%%%%%%
\section{Special Relativity}
Two conditions of \SR{}. 
\begin{itemize}
    \item general physical rules, which means in different flames, we can 
	observe the same EOM.
    \item constant c
\end{itemize}
let $X^\mu=(ct, \vec{x})$, so in order to make c constant, we have
\[
    -(ct')^2+\vec{x'}62 = (ct)^2 +\vec{x}^2
\]
The \textbf{simplest} way to preserve the length is to construct a \textbf{linear map}:
\[
    X'^\mu = {(\mathcal{L})^\mu}_{\nu}X^\nu
\]
Where ${(\mathcal{L})^\mu}_{\nu} = {(\mathcal{L}^T)_\nu}^{\mu} = {\Lambda^\mu}_\nu$.
Note that for a matrix, what important is the order of index, not their position, so
\[
    {\mathcal{L}^{\mu}}_{\nu}={\mathcal{L}_{\mu}}^{\nu}={(\mathcal{L}^T)_{\nu}}^{\mu}={(\mathcal{L}^T)^{\nu}}_{\mu}
\]
Let 
\[
    A = \begin{pmatrix}
	A^0 \\
	A^1 \\
	A^2 \\
	A^3 \\
    \end{pmatrix}\quad
    B = \begin{pmatrix}
	B^0 \\
	B^1 \\
	B^2 \\
	B^3 \\
    \end{pmatrix}
\]
and 
\[
    A_\mu  \equiv g_{\mn}A^\nu
\]
where 
\begin{equation}
    g_{\mn} = diag(-1, 1, 1, 1)
\end{equation}
\[
    g^{\mn}g_{\nu\sigma}={g^{\mu}}_\sigma={\delta^{\mu}}_\sigma 
\]
So, we can define:
\[
    A\cdot B \equiv -A^0B^0 + A^1B^1 + A^2B^2 + A^3B^3 = A^\mu B_\mu = A^\mu g_{\mn}B^\nu
\]
Inserting $\mathcal{L}\mathcal{L}^T$ into the above equation:
\[
    \begin{aligned}
	A_\mu B^\mu= A^\mu g_{\mn}B^\nu = (\mathcal{L}^{-1}A')^{\mu}g_{\mn}(\mathcal{L}^{-1})^\nu \\
	= (\mathcal{L}^{-1})^\mu_{\nu}A'^{\nu}g_{\mn}(\mathcal{L}^{-1})^\nu_{\rho}B'^\rho    \\
	= A'^{\nu}(\mathcal{L}^{-1})^\mu_{\nu}g_{\mn}(\mathcal{L}^{-1})^\nu_{\rho}B'^\rho    \\
	= A'^{\nu}({\mathcal{L}^{-1}}^T)_\nu^{\mu}g_{\mn}(\mathcal{L}^{-1})^\nu_{\rho}B'^\rho    \\
	= A'^{\nu}g'_{\nu\rho}B'^\rho
    \end{aligned}
\]
In \SR{}, we require that:
\[
    g'_{\nu\rho}=({\mathcal{L}^{-1}}^T)_\nu^{\mu}g_{\mn}(\mathcal{L}^{-1})^\nu_{\rho}=g_{\nu\rho}
\]
So we get
\[
    {\mathcal{L}^{-1}}^Tg\mathcal{L}^{-1}=g \rightarrow {\mathcal{L}^{-1}}^T=g\mathcal{L}g \rightarrow \mathcal{L}^{-1}=g\mathcal{L}^Tg
\]
With these conditions, we can easily find out $\mathcal{L}$. For 2D-coordinate $(ct, x)$, we have
\[
    \mathcal{L}(v) = 
	\begin{pmatrix}
	    \gamma  &	-\beta\gamma	\\
	    -\beta\gamma    & \gamma	\\
	\end{pmatrix}
\]
Where $\beta=\frac{v}{c}$ and $\gamma=\frac{1}{\sqrt{1-\beta^2}}$. 
Usually, we would like to use another quantity in \SR{}, which is rapidity:
\[
    \tanh{y}\equiv\frac{v}{c} \rightarrow y=\frac{1}{2}ln(\frac{1+\beta}{1-\beta})
\]
And correspondingly
\[
    \gamma=\cosh{y}, \quad \beta\gamma\sinh{y}
\]
