\chapter{Special Relativity}
%%%%%%%%%%%%%%%%%%%%%%%%%%%%%% Relativity %%%%%%%%%%%%%%%%%%%%%%%%%%%%%%
Basic assumption:
\begin{itemize}
    \item Laws of physics should be invariant under special relativity 
	transformations: rotations and velocity boosts	\eqnum\label{SM:assumption1}
    \item Speed of light is constant	\eqnum\label{SM:assumption2}
\end{itemize}

According to \ref{SM:assumption1} 
(the distance travelled by light within time
interval $dt$ is $dl = \sqrt{dx_idx_i}$, therefore $c^2dt^2 - dx_idx_i = c^2dt'^2 - dx'_idx'_i = 0$), 
generalize it, we will get the length $ds^2$ and it should be invariant under
transformations:
\begin{equation}
    ds^2 = c^2dt^2 - dx_idx_i = c^2dt'^2 - dx'_i dx'_i
\end{equation}
define Lorentz 4-vector $dx^\mu = (dt, dx^1, dx^2, dx^3)$ and the metrix tensor:
$$ g_{\mu\nu} = diag(1,-1, -1, -1) $$
then $ds^2 = g_{\mu\nu} dx^\mu dx^\nu$
so:
\begin{equation}
    x^\mu x_\mu = x^\mu g_{\mu\nu}x^\nu = x'^\mu x'_\mu
\end{equation}
let the transformation be: $x'^\mu = \Lambda^\mu_\nu x^\nu$ or $x' = \Lambda x$
then:
\begin{equation}
    x^T g x = x'^T g x' = (\Lambda x)^T g (\Lambda x) = x^T (\Lambda^T g \Lambda) x
\end{equation}
we get:
\begin{equation}
    \label{SR:LG}
 g = \Lambda^T g \Lambda
\end{equation}
if $g = E$ where $E = diag(1, 1, 1,1 )$ is the identity matrix, then we get:
$$ E = \Lambda^T \Lambda \Rightarrow \Lambda^T = \Lambda^{-1}$$
This defines the special orthogonal matrices (SO(4)), which must have $det\Lambda = \pm 1$

\subsubsection{Lorentz Group, SO(3, 1)}
However, our metric tensor g is not the identity matrix, but rather a mixed 
metrix with 3 -1 entries and 1 +1 entry. We call all elements that satisfy
\ref{SR:LG} as Lorentz Group SO(3, 1).

Here are a few examples of elements in SO(3, 1):
\begin{equation*}
    \Lambda_R = 
    \begin{pmatrix}
	1   & 0	\\
	0   & R
    \end{pmatrix}
    \quad \text{where R are the 3x3 rotation matrix SO(3)}
\end{equation*}

\begin{equation*}
    \Lambda_{B_x} = 
    \begin{pmatrix}
	\cosh\eta   & -\sinh\eta    & 0	& 0 \\
	-\sinh\eta   & \cosh\eta    & 0	& 0 \\
	0   & 0 & 1 & 0 \\
	0   & 0 & 0 & 1 \\
    \end{pmatrix}
    \text{velocity boost in the x direction}
\end{equation*}
where $\cosh\eta = \gamma$ and $\sinh\eta = \beta\gamma$, with $\gamma = \frac{1}{\sqrt{1-v^2/c^2}}$
and $\beta = v/c$

\section{Relativistic generalization from Classical results}
The power of radiation is (cgs units):
\[
    P = \frac{2}{3}\frac{q^2a^2}{c^3}
\]
Genelize it to \textbf{covariant} form:
\[
    P = \frac{2}{3}\frac{q^2}{m^2c^3}|\dot{\vec{P}}|^2
\]
The power P should be \emph{Lorentz invariant}, so $|\dot{\vec{P}}|^2$ 
should include the Lorentz scalar by taking the inner product of the 
four-acceleration $a^\mu = dp^\mu/d\tau$ with itself, so:
\[
    P = -\frac{2}{3}\frac{q^2}{m^2c^3}\frac{dp_\mu}{d\tau}\frac{dp^\mu}{d\tau}
\]
where:
\[
    \frac{dp_\mu}{d\tau}\frac{dp^\mu}{d\tau} = \beta^2 \left( \frac{dp_0}{d\tau} \right)^2 - \left( \frac{d\vec{p}}{d\tau} \right)^2
\]

The above inner product can also be written in terms of $\vec{\beta}$ and
its time derivative:
\[
    P = \frac{2q^2\gamma^6}{3c}\left[ (\dot{\vec{\beta}})^2 - (\vec{\beta} \times \dot{\vec{\beta}})^2 \right]
\]



%%%%%%%%%%%%%%%%%%%%%%%%%%%%%%%%%%%%%%%%%%%%%%%%%%%%%%%%%%%%%%%%%%%%%%%%
\section{Relativity}
Two conditions of \SR{}. 
\begin{itemize}
    \item general physical rules, which means in different flames, we can 
	observe the same EOM.
    \item constant c
\end{itemize}
let $X^\mu=(ct, \vec{x})$, so in order to make c constant, we have
\[
    -(ct')^2+\vec{x'}^2 = (ct)^2 +\vec{x}^2
\]
The \textbf{simplest} way to preserve the length is to construct a \textbf{linear map}:
\[
    X'^\mu = {(\mathcal{L})^\mu}_{\nu}X^\nu
\]
Where ${(\mathcal{L})^\mu}_{\nu} = {(\mathcal{L}^T)_\nu}^{\mu} = {\Lambda^\mu}_\nu$.
Note that for a matrix, what important is the order of index, not their position, so
\[
    {\mathcal{L}^{\mu}}_{\nu}={\mathcal{L}_{\mu}}^{\nu}={(\mathcal{L}^T)_{\nu}}^{\mu}={(\mathcal{L}^T)^{\nu}}_{\mu}
\]
Let 
\[
    A = \begin{pmatrix}
	A^0 \\
	A^1 \\
	A^2 \\
	A^3 \\
    \end{pmatrix}\quad
    B = \begin{pmatrix}
	B^0 \\
	B^1 \\
	B^2 \\
	B^3 \\
    \end{pmatrix}
\]
and 
\[
    A_\mu  \equiv g_{\mn}A^\nu
\]
where 
\begin{equation}
    g_{\mn} = diag(-1, 1, 1, 1)
\end{equation}
\[
    g^{\mn}g_{\nu\sigma}={g^{\mu}}_\sigma={\delta^{\mu}}_\sigma 
\]
So, we can define:
\[
    A\cdot B \equiv -A^0B^0 + A^1B^1 + A^2B^2 + A^3B^3 = A^\mu B_\mu = A^\mu g_{\mn}B^\nu
\]
Inserting $\mathcal{L}\mathcal{L}^T$ into the above equation:
\[
    \begin{aligned}
	A_\mu B^\mu= A^\mu g_{\mn}B^\nu = (\mathcal{L}^{-1}A')^{\mu}g_{\mn}(\mathcal{L}^{-1})^\nu \\
	= (\mathcal{L}^{-1})^\mu_{\nu}A'^{\nu}g_{\mn}(\mathcal{L}^{-1})^\nu_{\rho}B'^\rho    \\
	= A'^{\nu}(\mathcal{L}^{-1})^\mu_{\nu}g_{\mn}(\mathcal{L}^{-1})^\nu_{\rho}B'^\rho    \\
	= A'^{\nu}({\mathcal{L}^{-1}}^T)_\nu^{\mu}g_{\mn}(\mathcal{L}^{-1})^\nu_{\rho}B'^\rho    \\
	= A'^{\nu}g'_{\nu\rho}B'^\rho
    \end{aligned}
\]
In \SR{}, we require that:
\[
    g'_{\nu\rho}=({\mathcal{L}^{-1}}^T)_\nu^{\mu}g_{\mn}(\mathcal{L}^{-1})^\nu_{\rho}=g_{\nu\rho}
\]
So we get
\[
    {\mathcal{L}^{-1}}^Tg\mathcal{L}^{-1}=g \rightarrow {\mathcal{L}^{-1}}^T=g\mathcal{L}g \rightarrow \mathcal{L}^{-1}=g\mathcal{L}^Tg
\]
With these conditions, we can easily find out $\mathcal{L}$. For 2D-coordinate $(ct, x)$, we have
\[
    \mathcal{L}(v) = 
	\begin{pmatrix}
	    \gamma  &	-\beta\gamma	\\
	    -\beta\gamma    & \gamma	\\
	\end{pmatrix}
\]
Where $\beta=\frac{v}{c}$ and $\gamma=\frac{1}{\sqrt{1-\beta^2}}$. 
Usually, we would like to use another quantity in \SR{}, which is rapidity:
\[
    \tanh{y}\equiv\frac{v}{c} \rightarrow y=\frac{1}{2}ln(\frac{1+\beta}{1-\beta})
\]
And correspondingly
\[
    \gamma=\cosh{y}, \quad \beta\gamma\sinh{y}
\]



%%%%%%%%%%%%%%%%%%%%%%%%%%%%%%%%%%%%%%%%%%%%%%%%%%%%%%%%%%%%%%%%%%%%%%%%
\section{General Relativity}
Einstein's relativity thoery says that no info travels faster than the speed the light. 
On the other hand, Neuton's familiar law of gravity says the force of gravity
acts instantaneously on distant bodies. To resolve this parabox, Einstein 
proposed that matter bends and warps space and time, giving rise to gravity. 

GR successfully explain the movement of Mercury, whose orbit is not a perfect ellipse.
Each time it revolves around the sun, it always comes back very slightly ahead of 
the ellipse. The effect is extremely small, scientists had noted this effect
before Einstein, but could not account for it with Neutro's thoery of gravity. The
anomalous 43 arcseconds/century was explained by Einstein's theory.

